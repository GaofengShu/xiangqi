\documentclass[UTF8]{ctexbook}

%\usepackage{graphicx,booktabs}%插入图表功能
\usepackage[top=1in , bottom =1in , left=1.25in , right =1.25in]{geometry}
\usepackage{cchess}
\usepackage{subfig}
\usepackage{wrapfig}

\newcommand{\rc}[2]{\piece{#1}{#2}{r}} \newcommand{\bc}[2]{\piece{#1}{#2}{R}}
\renewcommand{\rm}[2]{\piece{#1}{#2}{n}} \newcommand{\bm}[2]{\piece{#1}{#2}{N}}
\newcommand{\rx}[2]{\piece{#1}{#2}{b}} \newcommand{\bx}[2]{\piece{#1}{#2}{B}}
\newcommand{\rs}[2]{\piece{#1}{#2}{g}} \newcommand{\bs}[2]{\piece{#1}{#2}{G}}
\newcommand{\rj}[2]{\piece{#1}{#2}{k}} \newcommand{\bj}[2]{\piece{#1}{#2}{K}}
\newcommand{\rp}[2]{\piece{#1}{#2}{c}} \newcommand{\bp}[2]{\piece{#1}{#2}{C}}
\newcommand{\rb}[2]{\piece{#1}{#2}{p}} \newcommand{\bb}[2]{\piece{#1}{#2}{P}}


\title{象棋胜和定式}
\author{金启昌\quad 杨\quad 典}
\date{1999 年 7 月  \\ \vspace{14cm} 北京体育大学出版社
}

\begin{document}

\maketitle

\thispagestyle{empty}
\begin{center}
{\zihao{2}\textbf{前言} }
\end{center}
\vspace{3cm}

象棋对弈的全过程,一般要经历开局(布局)、中局、残局三个阶段。本丛书专门论述残局实战技巧和胜和规律。

残局,作为全局中的结束阶段,一局棋的胜、负、和往往到此阶段决出结果。残晑中双方的对峙形势,往往是棋手在中局甚至布局时确定战略方针的依据。但残局又具有与其它阶段不同的特点和胜和规律。残局子力少,局势相对较为筒明,利于锻炼各种能力。所以,古往今来的众多棋手都把残局作为必修的基本功。

为帮助广大象棋爱好者丰富残局的知识和经验,掌握残局的攻守技巧和胜和规律,尽快提高驾驭残局的能力,特策划编写出版了这套《象棋残局战法丛书》。

《象棋残局战法丛书》共分5册:《象棋残局攻杀技巧》、《象棋残局妙手精华》、《象棋残局名局战术》、《象棋残局胜和定式》、《象棋残局生死棋型》.前3册战例均选自国内外重大象棋比赛的名手对局,实战性和实用性皆较强。《胜和定式》大部分来自实用残局;《生死棋型》是把常见的江湖残局归纳整理,进一步升华。

残局是以多种多样的战术手段为主体的战斗,“残局妙手”往往是在胜和要点的关键时刻起决定性作用,它是棋手们呕心呖血的艺术结局,是残局中各种各样的战术的精华,颇能引人入胜、引人入迷。本丛书以实战性和实用性为宗旨,以实战技巧和胜和规律为主线,选用了1000余例精妙残局,精心归纳整理,就象棋残局的不同攻杀特点和胜和规律,及与此有关的战略战术运用要点,分别详加阐述和印证,完整地展示了象棋残局研究的全貌,是一部难得的上佳之作。

由于写作时间仓促,书中不当之处,恳请读者、方家见谅并不吝指正。

如果你通过本丛书的学习而有所裨益,笔者将感到欣慰。

~\\

编著者

1999年春于北京 \footnote{2019年10月20日,舒高峰使用 \LaTeX{} 及 cchess 宏包整理而成。}


\clearpage
\tableofcontents
\thispagestyle{empty}


\clearpage
\chapter{胜局定式篇}
\setcounter{page}{1}
\section{兵类}
\subsection{高兵巧胜单士}
\begin{wrapfigure}{r}{5cm}
\centering
\vspace{-1.5cm}
\smallboard
\begin{position}
\rj{f}{1} \rb{f}{7}

\bj{f}{10} \bs{d}{10}
\end{position}
\caption{} \label{高兵巧胜单士} 
\end{wrapfigure}
(图\ref{高兵巧胜单士}),单高兵对单士,正常形势下成为和局。当形势特殊时,单高兵攻单士属于巧胜局。此局巧胜要诀:(一)兵逼九宫胁将;(二)帅占中路,小兵灭士。

1. 兵四进一,士4进5

2. 兵四进一,将6平5

3. 帅四平五,将5平4

4. 兵四平五

红胜。

\subsection{高兵相巧胜单士}
\begin{wrapfigure}{r}{5cm}
\centering
\vspace{-1.5cm}
\smallboard
\begin{position}
\rj{f}{1} \rx{e}{3} \rb{e}{7}

\bj{e}{10} \bs{d}{10}
\end{position}
\caption{} \label{高兵相巧胜单士} 
\end{wrapfigure}
(图\ref{高兵相巧胜单士}),此局巧胜要诀:(一)兵入中官控将,形成左(右)兵右(左)帅的势态;(二)等待黑方落后,兵从无士一边进入象腰。

1.兵五进一,士4进5 \qquad 2.帅四平五,士5退4

3.兵五平四,将5平6 \qquad 4.帅五平四,士4进5

5.兵四进一,将6平5 \qquad 6.帅四平五,士5退6

7.帅五平六

形成“左帅右兵”,已成胜式。

7. ..............,士6进5

8. 相五退三,士5退6

9. 帅六进一,..............

落相、进帅,运用等着入局。

9. ..............,士6进5

10.帅六平五

红胜。
\subsection{高兵仕巧胜单士}
\begin{wrapfigure}{r}{5cm}
\centering
\vspace{-1.5cm}
\smallboard
\begin{position}
\rj{e}{1} \rs{e}{2} \rb{f}{7}

\bj{e}{10} \bs{d}{10}
\end{position}
\caption{}\label{高兵仕巧胜单士} 
\end{wrapfigure}

(图\ref{高兵仕巧胜单士}),此局巧胜要诀与上局类同,其中红方仕、兵亦有定的作用,帅借仕(兵)遮头,起控制或助攻之妙用。

1.兵四进一,将5进1

如改走士4进5(如将5平6,则帅五平四,士4进5,兵四进一,将6平5,帅四平五红胜),则兵四进一,士5进6,帅五平六,士6退5,仕五退四,士5进6,帅六进一,士6退5,帅六平五,红胜。

2.帅五平六!将5退1 \qquad 3.兵四进一,士4进5

4.仕五退四,士5进4 \qquad 5.帅六进一,将5平4

6.兵四平五

红胜。

\subsection{高兵巧胜双士}
\begin{wrapfigure}{r}{5cm}
\centering
\vspace{-1.5cm}
\smallboard
\begin{position}
\rj{f}{1} \rb{e}{8}

\bj{f}{10} \bs{f}{8} \bs{d}{8}
\end{position}
\caption{}\label{高兵巧胜双士} 
\end{wrapfigure}
(图\ref{高兵巧胜双士}),此局巧胜要诀:(一)小兵破士(左、右士);(二)用帅走闲等着,得士或使黑方欠行。

【第一种胜法】

1.兵五平四,将6平5 \qquad 2.兵四进一,将5平4

3.帅四平五,将4进1 \qquad 4.帅五平六,将4退1

5.兵四平五

红胜。

【第二种胜法】

1.兵五平六,将6平5 \qquad 2.兵六进一,士6退5

3.帅四平五,将5平6

4.兵六乎五

红胜。

\subsection{高低兵巧胜士象}
(图\ref{高低兵巧胜士象}),此局巧胜要诀:(一)首着红帅牵制将、;(二)后兵插进,控将得士。

1.帅五平六!将4平5

如改走将4进1,则后兵进一,象9进7,帅六进一,象7退9,后兵平五,红胜。

2.后兵进一,象9进7 \qquad 3.帅六进一,士4退5

4.帅六平五,象7退5 \qquad 5.前兵平五,将5进1

6.兵四平五

红胜。

\begin{figure}[!htbp]
\centering
	\begin{minipage}{0.45\textwidth}
	\centering
		\smallboard
		\begin{position}
		\rj{e}{1} \rb{f}{7} \rb{f}{9}
		
		\bj{d}{10} \bs{d}{8} \bx{i}{8}
		\end{position}
	\caption{} \label{高低兵巧胜士象} 
	\end{minipage}
	\begin{minipage}{0.45\textwidth}
	\centering
		\smallboard
		\begin{position}
		\rj{f}{1} \rb{e}{7} \rb{c}{10}
		
		\bj{e}{10} \bs{d}{10} \bs{f}{10}
		\end{position}
	\caption{}\label{高底兵巧胜双士} 
	\end{minipage}
\end{figure}

\subsection{高底兵巧胜双士}
(图\ref{高底兵巧胜双士}),此局巧胜要诀:(一)冲中兵;(二)用帅控将。

1.兵五进一,士6进5 \qquad 2.帅四进一!士5退6

3.兵五平六,士4进5 \qquad 4.兵六进一

或走帅四平五!红胜。


\subsection{高低兵胜双士}
(图\ref{高低兵胜双士}),属例胜(稳胜)局。

1.兵七进一!将4退1

如改走将4进1,则帅四平五!以后再兵四平五,红胜。

2.帅四平五,士5进4 \qquad 3.兵四进一,士4退5

4.兵四平五,将4平5 \qquad 5.兵五进一!士6进5

6.兵七平六

红胜。

\begin{figure}[!htbp]
\centering
	\begin{minipage}{0.45\textwidth}
	\centering
		\smallboard
		\begin{position}
		\rj{f}{1} \rb{f}{7} \rb{c}{8}
		
		\bj{d}{9} \bs{e}{9} \bs{f}{10}
		\end{position}
	\caption{} \label{高低兵胜双士} 
	\end{minipage}
	\begin{minipage}{0.45\textwidth}
	\centering
		\smallboard
		\begin{position}
		\rj{f}{1} \rb{f}{9} \rb{c}{8}
		
		\bj{d}{9} \bx{a}{8} \bx{e}{8}
		\end{position}
	\caption{}\label{双低兵巧胜双象} 
	\end{minipage}
\end{figure}

\subsection{双低兵巧胜双象}
(图\ref{双低兵巧胜双象}),此局巧胜要诀:用帅控制黑方中象,分兵掠象破局。

1.帅四平五!象1进3 \qquad 2.兵七平八!象3退1

3.兵八平九

红胜。

\subsection{高低兵胜双象}
(图\ref{高低兵胜双象})高低兵攻双象,属例胜局。

1.兵二平三,将6平5 \qquad 2.兵四平五,象5退3

3.帅五平四!象3进5 \qquad 4.兵三平四,将5平4

5.帅四平五,象5进7 \qquad 6.兵五平六,象7退5

7.帅五平六

伏兵六进一冲兵杀,红胜。

\begin{figure}[!htbp]
\centering
	\begin{minipage}{0.45\textwidth}
	\centering
		\smallboard
		\begin{position}
		\rj{e}{1} \rb{f}{7} \rb{h}{9}
		
		\bj{f}{9} \bx{e}{8} \bx{c}{6}
		\end{position}
	\caption{} \label{高低兵胜双象} 
	\end{minipage}
	\begin{minipage}{0.45\textwidth}
	\centering
		\smallboard
		\begin{position}
		\rj{e}{1} \rb{e}{6} \rb{f}{9}
		
		\bj{e}{10} \bx{e}{8} \bs{e}{9} \bs{d}{10}
		\end{position}
	\caption{}\label{高低兵巧胜单缺象} 
	\end{minipage}
\end{figure}

\subsection{高低兵巧胜单缺象}
(图\ref{高低兵巧胜单缺象}),此局巧胜要诀:用高兵胁象,然后向低兵靠拢,控象、士禁毙入局。

1.兵五进一,象5退3 \qquad 2.兵五平四,象3进5

3.帅五进一!..............

等着!控制黑方士象。

3. ..............,象5退3 \qquad 4.后兵进一,象3进1

5.前兵平三!..............

分兵是获胜的关键着法。亦可走后兵平三!

5. ..............,象1进3

如改走将5平6,则兵四进一再兵三进一,红胜。

6.兵四进一,象3退5 \qquad 7.兵三进一!象5进3

8.兵三平四

红胜。
\subsection{高低兵巧胜单缺士}
\begin{wrapfigure}{r}{5cm}
\centering
\vspace{-1.5cm}
\smallboard
\begin{position}
\rj{e}{1} \rb{f}{7} \rb{d}{9}

\bj{e}{10} \bs{e}{9} \bx{e}{8} \bx{g}{10}
\end{position}
\caption{}\label{高低兵巧胜单缺士} 
\end{wrapfigure}
(图\ref{高低兵巧胜单缺士}),此局巧胜要诀:分兵左右夹击,控制黑士,形成“兵前帅后”入局。

1.兵四平三!士5进4

如改走象5进3,则兵四进一,象3退5,兵四进一,士5进6,帅五平六,红胜。

2.兵三进一,将5平6

3.兵三平四,象5进7

4.帅五平四!象7退5

5.兵四进一

红胜。
\subsection{高低兵巧胜士象全(一)}
(图\ref{高低兵巧胜士象全(一)}),此局巧胜要诀:妙用高兵掠黑方双象。

1.兵九平八!象1进3 \qquad 2.兵八平七,象3进1

3.兵七平八,象1退3 \qquad 4.兵八进一,象3进1

5.帅五进一!..............

等着!逼黑方动象。

5. ..............,象1退3 \qquad 6.兵八平九,象3进1

7.兵九进一

红胜。
\begin{figure}[!htbp]
\centering
	\begin{minipage}{0.45\textwidth}
	\centering
		\smallboard
		\begin{position}
		\rj{e}{1} \rb{a}{6} \rb{d}{9}
		
		\bj{e}{10} \bs{e}{9} \bs{f}{10} \bx{c}{10} \bx{a}{8}
		\end{position}
	\caption{} \label{高低兵巧胜士象全(一)} 
	\end{minipage}
	\begin{minipage}{0.45\textwidth}
	\centering
		\smallboard
		\begin{position}
		\rj{e}{1} \rb{b}{7} \rb{f}{9}
		
		\bj{d}{9} \bs{e}{9} \bs{d}{8} \bx{e}{8} \bx{g}{6}
		\end{position}
	\caption{}\label{高低兵巧胜士象全(二)} 
	\end{minipage}
\end{figure}
\subsection{高低兵巧胜士象全(二)}
(图\ref{高低兵巧胜士象全(二)}),此局巧胜要诀:首着出帅控六路,冲兵破士。

1.帅五平六!..............

获胜要着!否则黑将4退1,立即成和。

1. ..............,5进6 \qquad 2.兵八进一,将4退1

3.兵八平七,将4平5 \qquad 4.兵七平六,士6退5

5.兵六进一,十5进4 \qquad 6.帅六平五,象7退9

7. 帅五平四

红胜。

\subsection{双高兵胜单炮}
\begin{wrapfigure}{r}{5cm}
\centering
\vspace{-1.5cm}
\smallboard
\begin{position}
\rj{e}{1} \rb{e}{7} \rb{f}{7}

\bj{e}{9} \bp{e}{10}
\end{position}
\caption{}\label{双高兵胜单炮} 
\end{wrapfigure}
(图 \ref{双高兵胜单炮}),属例胜局。

1.帅五进一!..............

进帅等着!迫使黑方动将。

1. ..............,将5平4

如改走炮5平9,则兵五进一,将5平4,兵四进一,炮9进1,兵四进一,红胜。

2.兵五平六,炮5平4 \qquad 3.兵六平七!炮4平5

4.兵四进一,炮5进1 \qquad 5.兵七进一,炮5退1

6.帅五平四,炮5平9 \qquad 7.兵四平五,炮9进1

8.兵七平六,将4退1 \qquad 9.兵五进一

再帅四平五,红胜。
\subsection{双高兵巧胜炮象}
\begin{wrapfigure}{r}{5cm}
\centering
\vspace{-1.5cm}
\smallboard
\begin{position}
\rj{e}{1} \rb{f}{7} \rb{d}{7}

\bj{f}{10} \bx{i}{8} \bp{e}{6}
\end{position}
\caption{} \label{双高兵巧胜炮象} 
\end{wrapfigure}
(图15),此局巧胜要诀:双兵逼宫,用帅牵黑炮、将。

1.兵六进一,象9进7

另有两种着法:①将6进1,兵六平五,象9进7,兵四进一,将6退1,兵五进一,红胜;②炮5退4,兵四进一,象9进7,帅五平四,炮5平1,兵四进一,将6平5,兵六进一,红胜。

2.兵四进一,将6平5 \qquad 3.兵六进一,炮5进2

4.兵四进一,象7退5 \qquad 5.帅五平六,炮5平4

6.兵六平五,将5平4 \qquad 7.兵四进一

红胜。


\subsection{双高兵巧胜炮士}
(图\ref{双高兵巧胜炮士}),此局巧胜要诀:冲中兵巧吃黑士。

1.兵五进一!士5进6 \qquad 2.帅四平五,炮6平5

如改走炮6平8,则兵五平六!将4退1,兵七进一,士6退5,兵七平六,将4平5,前兵平五,将5平6,兵六平五,红胜。

3.兵五平四,炮5进1 \qquad 4.帅五平四,炮5进2

5.兵四平五,炮5平8 \qquad 6.兵七进一,将4退1

7.兵五进一

红胜。
\begin{figure}[!htbp]
\centering
	\begin{minipage}{0.45\textwidth}
	\centering
		\smallboard
		\begin{position}
		\rj{f}{1} \rb{e}{7} \rb{c}{8}
		
		\bj{d}{9} \bs{e}{9} \bp{f}{10}
		\end{position}
	\caption{}\label{双高兵巧胜炮士} 
	\end{minipage}
	\begin{minipage}{0.45\textwidth}
	\centering
		\smallboard
		\begin{position}
		\rj{d}{2} \rb{d}{7} \rb{g}{9}
		
		\bj{e}{9} \bm{e}{6}
		\end{position}
	\caption{} \label{高低兵巧胜单马} 
	\end{minipage}
\end{figure}

\subsection{高低兵巧胜单马}
(图\ref{高低兵巧胜单马}),双高兵攻单马是例胜,高低兵属巧胜。此局巧胜要诀:双兵逼将,用帅控马助攻。

1.帅六平五!将5平4

如改走将5进1(如将5退1,则兵三平四!),则帅五退一,将5平6,兵六平五,红胜。

2.兵三平四,马5进4 \qquad 3.帅五平六,马4退5

如改走马4进6,则帅六退一,马6退4,兵六平五,将4退1,兵五进一,将4平5,兵五进一再兵四进一,红胜。

4.兵六进!..............

弃兵困马,妙手入局。

4. ..............,马5退4 \qquad 5.帅六退一,将4退1

6.兵四平五

黑方欠行,红胜。
\subsection{双低兵巧胜马士}
(图\ref{双低兵巧胜马士}),此局巧胜要诀:用帅助攻,双兵胁士。

1.帅四平五!马7进5

如改走马7退5,则兵三平四,士5进6,帅五进一(或帅五平六),士6退5,兵六平五,将5平4,兵四进一,马5退7,帅五退一,红胜。

2.兵三平四,马5进4 \qquad 3.帅五退一,马6退5

4.兵六平五,将5平4 \qquad 5.兵四进一,马5退6

6.帅五进一,马6进5 \qquad 7.兵四平五

红胜。

\begin{figure}[!htbp]
\centering
	\begin{minipage}{0.45\textwidth}
	\centering
		\smallboard
		\begin{position}
		\rj{f}{2} \rb{d}{9} \rb{g}{9}
		
		\bj{e}{10} \bs{e}{9} \bm{g}{7}
		\end{position}
	\caption{} \label{双低兵巧胜马士} 
	\end{minipage}
	\begin{minipage}{0.45\textwidth}
	\centering
		\smallboard
		\begin{position}
		\rj{f}{1} \rb{d}{9} \rb{g}{9}
		
		\bj{e}{10} \bx{c}{6} \bx{g}{6} \bm{e}{7}
		\end{position}
	\caption{}\label{双低兵巧胜马双象} 
	\end{minipage}
\end{figure}
\subsection{双低兵巧胜马双象}
(图\ref{双低兵巧胜马双象}),此局巧胜要诀:用帅拴马,弃兵闷杀。

1.帅四平五!将5平6

如改走象7退5,则兵三平四,马5退7,帅五平六,再兵六进一,杀局。

2.兵六平五,象7退9 \qquad 3.兵三平四!马5退6

4.兵五进一

红胜。
\subsection{三低兵巧胜单缺象}
\begin{wrapfigure}{r}{5cm}
\centering
\vspace{-1.5cm}
\smallboard
\begin{position}
\rj{e}{1} \rb{a}{9} \rb{d}{9} \rb{g}{9}

\bj{f}{10} \bs{e}{9} \bs{d}{10} \bx{a}{8}
\end{position}
\caption{}\label{三低兵巧胜单缺象} 
\end{wrapfigure}
(图\ref{三低兵巧胜单缺象}),此局巧胜要诀:冲底兵破双士,利用等着入局。

1.兵九平八,象1进3 \qquad 2.兵八平七,象3退1

3.帅五进一!象1进3 \qquad 4.兵七进一!象3退5

5.兵七平六!士5退4 \qquad 6.兵六进一,象5进3

7.兵六平五

红胜。

\subsection{三兵胜卒单缺士(一)}
(图\ref{三兵胜卒单缺士(一)}),红有高兵属例胜局。

1.兵七平六,卒5进1 \qquad 2.兵三进一,象3退1

3.兵三平四,士5退4 \qquad 4.兵六进一!..............

弃兵破士,引离黑将,妙手入局。

4. ..............,将5平4 \qquad 5.前兵平五

红胜。

\begin{figure}[!htbp]
\centering
	\begin{minipage}{0.45\textwidth}
	\centering
		\smallboard
		\begin{position}
		\rj{e}{1} \rb{f}{7} \rb{g}{8}  \rb{c}{9}
		
		\bj{e}{10} \bs{e}{9} \bx{e}{8} \bx{c}{6} \bb{e}{4}
		\end{position}
	\caption{} \label{三兵胜卒单缺士(一)} 
	\end{minipage}
	\begin{minipage}{0.45\textwidth}
	\centering
		\smallboard
		\begin{position}
		\rj{e}{2} \rb{d}{7} \rb{f}{9}  \rb{c}{9}
		
		\bj{d}{10} \bs{f}{8} \bx{a}{8} \bx{g}{6} \bb{e}{4}
		\end{position}
	\caption{}\label{三兵胜卒单缺士(二)} 
	\end{minipage}
\end{figure}

\subsection{三兵胜卒单缺士(二)}
(图\ref{三兵胜卒单缺士(二)}),此局属例胜局,首着平中兵是获胜的关键。

1.兵六平五!..............

如误走兵六进一,则将4平5,兵七平六,象1进3,帅五平四,卒5平6,前兵平七,象3退1,兵六进一,象7退5,和局。

1. ..............,将4平5 \qquad 2.兵七平六,卒5平4

3.帅五平四,卒4平5 \qquad 4.兵五平四

捉死黑士,红胜。
\subsection{三兵胜士象全}
\begin{wrapfigure}{r}{5cm}
\centering
\vspace{-1.5cm}
\smallboard
\begin{position}
\rj{e}{1} \rb{c}{7} \rb{d}{7} \rb{f}{7}

\bj{d}{9} \bs{e}{9} \bs{f}{10} \bx{c}{6} \bx{g}{10}
\end{position}
\caption{}\label{三兵胜士象全} 
\end{wrapfigure}
(图\ref{三兵胜士象全}),属例胜局。

1.兵四平三,象7进5 \qquad 2.兵三进一,象5进7

3.兵三进一,象3退1 \qquad 4.兵三平四!..............

冲兵卡肋,是获胜的第一步骤。

4. ..............,象1进3 \qquad 5.兵七进一,象7退5

6.帅五平六!象5进7 \qquad 7.兵六进一!..............

冲高兵叫将,是获胜的第二步骤。

7. ..............,将4退1 \qquad 8.兵六进一,将4平5

附图所示,红方有两种胜法,为获胜的第三步骤。

【第一种胜法】

9.兵六平五!..............

破士,获胜精妙着法。

9. ..............,士6进5 \qquad 10.帅六平五,象3退5

11.兵七进一,象7退9 \qquad 12.兵七平六,士5进4

\begin{wrapfigure}{r}{5cm}
\centering
%\vspace{-1.5cm}
\smallboard
\begin{position}
\rj{d}{1} \rb{c}{8} \rb{d}{9} \rb{f}{9}

\bj{e}{10} \bs{e}{9} \bs{f}{10} \bx{c}{6} \bx{g}{6}
\end{position}
\caption*{附图}\label{三兵胜士象全:附图} 
\end{wrapfigure}
13.帅五平四

红胜。

【第二种胜法】

9.帅六平五,..............

如误走兵七进一,则士5进4,成为和局。但如红方有相则必胜,读者不妨自演。

9. ..............,象7退5 \qquad 10.兵七进一,士5进4

11.帅五平四,6进5 \qquad 12.兵六平五!士4退5

13.兵七平六,5进6 \qquad 14.帅四平五

再帅五平六,红胜。

\subsection{三兵胜炮双象}
\begin{wrapfigure}{r}{5cm}
\centering
\vspace{-1.5cm}
\smallboard
\begin{position}
\rj{e}{1} \rb{d}{9} \rb{g}{7} \rb{h}{7}

\bj{f}{9} \bx{e}{8} \bx{c}{6} \bp{b}{8}
\end{position}
\caption{}\label{三兵胜炮双象} 
\end{wrapfigure}
(图\ref{三兵胜炮双象}),双高兵一低兵攻炮双象,属例胜局。

1.兵三平四,炮2进1 \qquad 2.兵二平三,炮2平1

3.帅五平四,炮1退1 \qquad 4.兵三进一!..............

弃兵引离黑炮,精巧入局。

4. ..............,炮1平7

如改走将6退1,则兵六平五,红胜。

5.兵四进一,将6退1 \qquad 6.兵四进一,将6平5

7.兵四进一

红胜。

\subsection{三兵巧胜炮双士}
\begin{wrapfigure}{r}{5cm}
\centering
\vspace{-1.5cm}
\smallboard
\begin{position}
\rj{e}{1} \rb{d}{6} \rb{e}{7} \rb{c}{9}

\bj{d}{10} \bs{e}{9} \bs{f}{8} \bp{d}{8}
\end{position}
\caption{}\label{三兵巧胜炮双士} 
\end{wrapfigure}
(图\ref{三兵巧胜炮双士}),此局巧胜要诀:双高兵推进腹地,巧妙破士。

1.兵五进一,炮4平1 \qquad 2.兵六进一,炮1进2

3.兵五平四!士5进6

形成高低兵巧胜炮士之局。

4.兵六进一,炮1平5

如改走士6退5,则兵七平六,将4平5,兵六平五,将5平6,兵六平五,红胜。

5.兵七平六,将4平5 \qquad 6.后兵平五,炮5平8

7.帅五半四,炮8退4 \qquad 8.兵五平四,炮8平6

9.兵四平五,炮6进4 \qquad 10.兵五进一,将5平4

11.兵六进一

红胜。

\subsection{三兵巧胜炮士象}
(图\ref{三兵巧胜炮士象}),此局巧胜要诀:高兵从左翼进击。

1.兵六进一!炮5平4

\begin{wrapfigure}{r}{5cm}
\centering
\vspace{-1.5cm}
\smallboard
\begin{position}
\rj{e}{1} \rb{d}{6} \rb{b}{8} \rb{f}{9}

\bj{d}{9} \bs{e}{9} \bx{e}{8} \bp{e}{10}
\end{position}
\caption{}\label{三兵巧胜炮士象} 
\end{wrapfigure}

如改走士5进6,则帅五平四,炮5平4,兵六平五,象5进3,兵八平七,炮4平1,兵五平六,炮1进1,兵六进一,将4退l,帅四平五,伏兵七进一,红胜。

2.兵六平七,炮4平5 \qquad 3.兵八平七,士5进6

4.帅五平四,炮5平1

如改走炮5平9,则后兵平六,象5进7,兵六进一,将4退1,帅四平五,炮9进2,帅五平六,红胜。

5.后兵平六,象5进7 \qquad 6.帅四平五,炮1进1

7.兵六进一,将4退1 \qquad 8.兵七进一!象7退5

9.兵四进一

再兵七平六杀,红胜。

\subsection{三兵胜马双象}
\begin{wrapfigure}{r}{5cm}
\centering
\vspace{-1.5cm}
\smallboard
\begin{position}
\rj{e}{1} \rb{e}{6} \rb{c}{9} \rb{f}{9}

\bj{d}{10} \bx{e}{8} \bx{a}{8} \bm{g}{8}
\end{position}
\caption{}\label{三兵胜马双象} 
\end{wrapfigure}
(图\ref{三兵胜马双象}),一高兵两低兵,属例胜局。

1.兵五进一!..............

弃中兵,妙!获胜关键着法。

1. ..............,马7进5

如改走象5进7,则兵五平六再进一,红胜。

2.兵四平五,象5进7 \qquad 3.兵七平六!马5退4

4.兵五进一

弃兵闷杀,红胜。

\subsection{三兵巧胜马双士}
\begin{wrapfigure}{r}{5cm}
\centering
\vspace{-1.5cm}
\smallboard
\begin{position}
\rj{e}{1} \rb{e}{7} \rb{c}{7} \rb{c}{9}

\bj{d}{10} \bs{e}{9} \bs{f}{10} \bm{f}{8}
\end{position}
\caption{}\label{三兵巧胜马双士} 
\end{wrapfigure}
(图\ref{三兵巧胜马双士}),此局巧胜要诀:三兵有两种胜法:(一)弃兵困毙,黑方欠行;(二)弃兵破士入局。

【第一种胜法】

1.前兵进一,将4平5

如改走将4进1,则兵七进一,红胜。

2.兵五平四,马6进8 \qquad 3.后兵进一,马8进7

4.后兵平六!..............

靠兵弃兵,妙!算准入局。

4. ..............,马7退6 \qquad 5.兵六进一,马6进5

6.帅五进一,马5退4 \qquad 7.兵七平六

红胜。

【第二种胜法】

1.兵五进一,马6进5 \qquad 2.前兵进一!将4平5

3.兵七进一,士5退4 \qquad 4.兵五平四,士6进5

5.兵四进一,马5退7 \qquad 6.前兵平六!将5平4

7.兵四平五

红胜。

\subsection{三兵巧胜马士象}

(图\ref{三兵巧胜马士象}),此局巧胜要诀:冲兵逼马,以兵换象,弃兵困毙。

1.兵六进一,马3进1 \qquad 2.兵七平六,士5进6
3.后兵进一,马1退3 \qquad 4.后兵平五!马3退5

如改走马3退4,则兵五进一,士6退5,兵四平五,将5平6,兵五平六,红胜。

5.帅五进一!6退5 \qquad 6.兵四平五,将5平6

7.兵六进一,马5退3 \qquad 8.帅五退一,马3进4

9.兵六平五

红胜。

\begin{figure}[!htbp]
\centering
	\begin{minipage}{0.45\textwidth}
	\centering
		\smallboard
		\begin{position}
		\rj{e}{1} \rb{d}{6} \rb{c}{9} \rb{f}{9}
		
		\bj{e}{10} \bs{e}{9} \bx{e}{8} \bm{c}{7}
		\end{position}
		\caption{}\label{三兵巧胜马士象} 
	\end{minipage}
	\begin{minipage}{0.45\textwidth}
	\centering
		\smallboard
		\begin{position}
		\rj{d}{1} \rb{d}{9} \rb{e}{9} \rb{g}{7}
		
		\bj{f}{10} \bc{b}{8}
		\end{position}
		\caption{}\label{三兵巧胜单车}
	\end{minipage}
\end{figure}
\subsection{三兵巧胜单车}

(图\ref{三兵巧胜单车}),此局巧胜要诀:首着高兵靠肋,运用帅等着过渡移至四路,使黑单车难以防范,形成“三英战吕布”。

1.兵三平四!..............

如误走帅六平五,则车2平5,帅五平六,车5平6,帅六进一,车6平4,帅六平五,车6平5,帅五平六,车5平6,和局。

1. ..............,车2平5 \qquad 2.帅六进一,车5平9

3.帅六平五,车9平5 \qquad 4.帅五平四,车5进1

如改走车5平4,则兵六进一!车4退2,兵四进一,红胜。

5.兵四进一!车5平6 \qquad 6.帅四平五车6退1

7.兵五进一,将6进1 \qquad 8.兵六平五

红胜。

%==================================================================
\section{炮类}
\subsection{炮仕胜双士}
(图\ref{炮仕胜双士}),属例胜局。此局取胜要诀:用炮等着,控将得士。

1.仕五进六,将5平6 \qquad 2.炮六平四,将6平5

3.炮四平三!..............

等着!准备控制黑将。

3. ..............,将5平6 \qquad 4.炮三平六,将6平5

5.炮六平四!将5平4 \qquad 6.炮四平五,..............

以上红方连续四步平炮,耐人寻味,是获胜的要着。

6. ..............,士5进4 \qquad 7.炮五平六,士6退5

8.炮六退一!将4平5 \qquad 9.炮六进七

得,红胜。


\begin{figure}[!htbp]
\centering
	\begin{minipage}{0.45\textwidth}
	\centering
		\smallboard
		\begin{position}
		\rj{e}{1} \rs{e}{2} \rp{d}{2}
		
		\bj{e}{10} \bs{e}{9} \bs{f}{8}
		\end{position}
	\caption{}\label{炮仕胜双士} 
	\end{minipage}
	\begin{minipage}{0.45\textwidth}
	\centering
		\smallboard
		\begin{position}
		\rj{f}{2} \rs{e}{2} \rp{d}{1}
		
		\bj{e}{10} \bb{f}{4}
		\end{position}
	\caption{}\label{炮仕巧胜高卒} 
	\end{minipage}
\end{figure}
\subsection{炮仕巧胜高卒}
(图\ref{炮仕巧胜高卒}),此局巧胜要诀:首着平炮打卒,抢中制胜。

1.炮六平四!..............

如误走炮六平五,则将5平4,炮五平四,卒6平5,帅四进一,将4进1,炮四平五,卒5平4,帅四平五,卒4平5,帅五平四,卒5平4,和局。

1. ..............,卒6平7 \qquad 2.帅四进一,将5平4

如改走卒7进1,则帅四退一,将5进1,炮四平七,将5退1,炮七进二,将5进1,仕五进六,卒7进1,帅四进一,将5退1,仕六退五,将5退1,炮七退二,将5进1,炮七平五,将5平4,帅四平五,红胜。

3.帅四平五,卒7平6 \qquad 4.仕五进六,卒6平5

5.帅五退一,将4进1 \qquad 6.炮四平六,将4平5

7.炮六平五

红胜。

\subsection{炮仕巧胜双低卒}
\begin{wrapfigure}{r}{5cm}
\centering
\vspace{-1.5cm}
\smallboard
\begin{position}
\rj{d}{3} \rs{f}{3} \rp{h}{5}

\bj{f}{10} \bb{c}{2} \bb{g}{2}
\end{position}
\caption{}\label{炮仕巧胜双低卒} 
\end{wrapfigure}
(图\ref{炮仕巧胜双低卒}),此局巧胜要诀:运炮伏击卒,使黑卒远离中,抢中夺胜。

1.炮二平一,卒7平8 \qquad 2.炮一平九,卒3平2

3.帅六平五,卒8平7 \qquad 4.炮九平一,卒7平8

5.炮一退四,卒8平7 \qquad 6.帅五退一,卒7进1

7.炮一进一

红胜。

\subsection{炮双仕巧胜双底卒}
(图\ref{炮双仕巧胜双底卒}),此局巧胜要诀:采用进炮阻将法,逼黑将进中,调炮胜。

1.帅五平四!将6进1 \qquad 2.炮四进二!将6退1

3.炮四进一!将6平5 \qquad 4.炮四平七,将5进1

5.炮七退八,将5退1 \qquad 6.炮七平五,将5平4

7.帅六平五

红胜。

\begin{figure}[!htbp]
\centering
	\begin{minipage}{0.45\textwidth}
	\centering
		\smallboard
		\begin{position}
		\rj{e}{3} \rs{e}{2} \rs{d}{3} \rp{f}{6}
		
		\bj{f}{10} \bb{f}{2} \bb{g}{2}
		\end{position}
	\caption{}\label{炮双仕巧胜双底卒} 
	\end{minipage}
	\begin{minipage}{0.45\textwidth}
	\centering
		\smallboard
		\begin{position}
		\rj{f}{3} \rs{e}{2} \rs{d}{3} \rp{d}{1}
		
		\bj{e}{9} \bb{e}{4} \bx{e}{8}
		\end{position}
	\caption{}\label{炮双仕巧胜高卒象} 
	\end{minipage}
\end{figure}
\subsection{炮双仕巧胜高卒象}
(图\ref{炮双仕巧胜高卒象}),属例胜局。

1.炮六平五!卒5平4

如改走卒5平6,帅四平五,将5平4,仕五进四,将4平5,炮五进一,红胜。(同正变)

2.帅四退一,卒4平3 \qquad 3.炮五平六,将5退1

4.帅四进一,将5进1 \qquad 5.帅四平五,将5平6

6.仕五进四,将6平5 \qquad 7.炮六平五,卒3平4

8.炮五进一!

进炮等着!红必得象,胜定。
\subsection{炮双仕巧胜士象}
\begin{wrapfigure}{r}{5cm}
\centering
\vspace{-1.5cm}
\smallboard
\begin{position}
\rj{d}{2} \rs{e}{2} \rs{f}{3} \rp{g}{1}

\bj{f}{10} \bs{d}{8} \bx{i}{8}
\end{position}
\caption{}\label{炮双仕巧胜士象} 
\end{wrapfigure}

(图\ref{炮双仕巧胜士象}),此局巧胜要诀:首着炮镇中路,用帅拴黑、将,形成胜局。否则是和局。

1.炮三平五!象9进7 \qquad 2.帅六进一象7退9

如改走将6进1,则炮五平四,将6平5,炮四平六,得士,红胜。

3.帅六平五,士4退5 \qquad 4.炮五平四,将6平5

5.仕五进六,将5平4 \qquad 6.炮四平六,将4平5

7.炮六平五

红胜。
\subsection{炮双仕巧胜单马}
(图\ref{炮双仕巧胜单马}),此局巧胜要诀:帅控中路,用炮制马。

1.帅六平五!马3进5

如改走马3进4踩仕,则炮九退六,马4进3,帅五进一,将6进1,炮九平八!再进一,红胜。

2.炮九退六,马5进7 \qquad 3.炮九平三,将6进1

4.炮三进一

红胜。

\begin{figure}[!htbp]
\centering
	\begin{minipage}{0.45\textwidth}
	\centering
	\smallboard
	\begin{position}
	\rj{d}{2} \rs{d}{3} \rs{f}{3} \rp{a}{7}
	
	\bj{f}{10} \bm{c}{5}
	\end{position}
	\caption{}\label{炮双仕巧胜单马} 
	\end{minipage}
	\begin{minipage}{0.45\textwidth}
	\centering
	\smallboard
	\begin{position}
	\rj{e}{3} \rs{d}{3} \rs{e}{2} \rp{g}{5}
	
	\bj{d}{8} \bp{e}{9}
	\end{position}
	\caption{}\label{炮双仕巧胜单炮} 
	\end{minipage}
\end{figure}
\subsection{炮双仕巧胜单炮}
(图\ref{炮双仕巧胜单炮}),此局由于黑将位不佳,红方退炮控中路取胜。

1.炮三退四,炮5进52.炮三平六,将4平5

3.仕五进四,将5平64.炮六平四,将6平5

5.炮四平五

红胜。

\subsection{双炮胜双士}
\begin{wrapfigure}{r}{5cm}
\centering
\vspace{-1.5cm}
\smallboard
\begin{position}
\rj{f}{3} \rp{f}{1} \rp{g}{2}

\bj{d}{9} \bs{e}{9} \bs{f}{8}
\end{position}
\caption{}\label{双炮胜双士} 
\end{wrapfigure}
(图\ref{双炮胜双士}),属例胜局。

1.炮四平六,将4退1 \qquad 2.炮三平六,将4平5

3.帅四平五!..............

进帅控制黑士活动,必要的等着。

3. ..............,将5平6 \qquad 4.前炮平四,将6平5

5.炮四进二!..............

高炮逼将、控中,获胜关键着法。

5. ..............,将5平4 \qquad 6.炮四平五!将4进1

7.炮五平六

重炮杀,红胜。


\subsection{双炮双象胜双象}
(图\ref{双炮胜双士}),属例胜局。

1.炮七平五!将5平6 \qquad 2.帅六平五!..............

红炮打将进帅控制双象,紧要之着。

2. ..............,将6退1 \qquad 3.炮三平四,将6进1

4.炮四进五!..............

塞象腰得象,获胜关键之着。

4. ..............,象5退3 \qquad 5.炮五平四

红胜。

\begin{figure}[!htbp]
\centering
	\begin{minipage}{0.45\textwidth}
	\centering
	\smallboard
	\begin{position}
	\rj{d}{3} \rp{c}{1} \rp{g}{2} \rx{c}{5} \rx{g}{5}
	
	\bj{e}{9} \bx{e}{8} \bx{g}{6}
	\end{position}
	\caption{}\label{双炮双象胜双象} 
	\end{minipage}
	\begin{minipage}{0.45\textwidth}
	\centering
		\smallboard
		\begin{position}
		\rj{e}{3} \rp{c}{2} \rp{h}{3} \rx{c}{5} \rx{g}{5}
		
		\bj{e}{10} \bs{e}{9} \bs{f}{10} \bx{e}{8} \bx{g}{10}
		\end{position}
	\caption{}\label{双炮双象胜士象全} 
	\end{minipage}
\end{figure}
\subsection{双炮双象胜士象全}
(图\ref{双炮双象胜士象全}),属例胜局。

1.帅五平六!士5进6

出帅控将,紧着。黑如改走象5进7,则炮七平五,士5进4,炮二平五,将5平4,后炮平六,6进5,炮五退二,伏帅六平五,红得士胜定。

2.炮二进七,象5进7 \qquad 3.炮七平四,士6退5

4.炮四平三,象7退9

如改走象7退5,则炮三平五,黑方欠行。

5.炮二退一!象7进5 \qquad 6.炮三平五

红胜。
\subsection{双炮仕胜士象全}
\begin{wrapfigure}{r}{5cm}
\centering
\vspace{-1.5cm}
\smallboard
\begin{position}
\rj{e}{3} \rs{e}{2} \rp{d}{1} \rp{d}{2}

\bj{e}{10} \bs{e}{9} \bs{f}{8} \bx{e}{8} \bx{g}{6}
\end{position}
\caption{}\label{双炮仕胜士象全} 
\end{wrapfigure}
(图\ref{双炮胜双士}),属例胜局。

1.仕五进四,象7退9

如改走士5退4,则后炮平五,士6退5,炮六平九,士5进4,炮五进七!象7退5,炮九平五,得象,形成炮仕必胜双的局面。

2.后炮平五,象9进7 \qquad 3.帅五平六!..............

出帅控将,紧要之着。

3. ..............,将5平6 \qquad 4.炮六平四!象5退7

5.炮四退一!象7进9 \qquad 6.帅六平五,象9退7

7.炮五进八

得士,红方胜定。

7. ..............,将6进1 \qquad 8.炮五平八,象7退5

9.炮八退七,象7进9 \qquad 10.炮四进七!象9进7

不能走将6进1,则炮八平四杀。

11.炮八平五,象5进3 \qquad 12.炮四退一,将6退1

13.炮五平四,象3退5 \qquad 14.前炮平五

红胜。

\subsection{双炮巧胜单炮}
(图\ref{双炮胜双士}),此局巧胜要诀:利用空头炮制胜。

1.炮八平四!炮1平2

如不动炮,则炮九退四,红胜。

2.炮九平八,炮2平3 \qquad 3.炮八平七,炮3平4

4.炮七平六,炮4平5 \qquad 5.炮六退四,将6进1

6.炮六平四,将6平5 \qquad 7.炮四平五

红胜。

\begin{figure}[!htbp]
\centering
	\begin{minipage}{0.45\textwidth}
	\centering
	\smallboard
	\begin{position}
	\rj{e}{2} \rp{a}{5} \rp{b}{5}
	
	\bj{f}{10} \bp{a}{8}
	\end{position}
	\caption{}\label{双炮巧胜单炮} 
	\end{minipage}
	\begin{minipage}{0.45\textwidth}
	\centering
		\smallboard
		\begin{position}
		\rj{f}{3} \rp{e}{1} \rp{g}{2}
		
		\bj{d}{9} \bm{f}{5}
		\end{position}
	\caption{}\label{双炮胜单马} 
	\end{minipage}
\end{figure}

\subsection{双炮胜单马}
(图\ref{双炮胜单马}),此局属例胜局,但要注意马换炮则成和局。

1.炮三平四!马6退5

另有两种应法:①马6退8,帅六平五,马8进7,炮四进二,马7退6,炮四平六,红胜。②马6退4,帅四平五,将4退1(如马4进3,则帅五退一,将4进1,炮四进二再平六,亦红胜),炮四平六,马4进3,炮六进二,马3进2,炮五平七,马2进4,炮六退二,马4退6,帅五退一,红胜。

2.帅四平五,马5进4 \qquad 3.帅五退一,将4退1

4.炮四进一,将4进1 \qquad 5.炮五平六,马4进6

6.帅五进一,将4退1 \qquad 7.炮四退一,马6进7

8.炮六进一,马7退8 \qquad 9.炮四退一,马8退6

10.帅五退一,马6进7 \qquad 11.炮四进二,马7进6

12.炮六退一,马6退7 \qquad 13.帅五退一

再炮四平六重炮杀,红胜。

\subsection{双炮巧胜马双士}
\begin{wrapfigure}{r}{5cm}
\centering
\vspace{-1.5cm}
\smallboard
\begin{position}
\rj{e}{2} \rp{g}{7} \rp{h}{6}

\bj{e}{10} \bs{e}{9} \bs{d}{10} \bm{f}{8}
\end{position}
\caption{}\label{双炮巧胜马双士} 
\end{wrapfigure}
(图\ref{双炮巧胜马双士}),此局巧胜要诀:利用炮帅拴马法取胜。

1.炮三平七!将5平6 \qquad 2.帅五平四,士5进4

3.炮七进一!..............

打马紧着!控制黑马。

3. ..............,士4进5 \qquad 4.炮二平五!将6进1

5.帅四进一!将6退1 \qquad 6.炮七退七,将6进1

7.炮七平四,将6退1 \qquad 8.炮五退四,将6进1

9.帅四平五,马6进7 \qquad 10.炮五平四

重炮杀,红胜。

\subsection{双炮巧胜炮双士}
\begin{wrapfigure}{r}{5cm}
\centering
\vspace{-1.5cm}
\smallboard
\begin{position}
\rj{d}{2} \rp{e}{2} \rp{g}{6}

\bj{d}{10} \bs{f}{8} \bs{f}{10} \bp{d}{9}
\end{position}
\caption{}\label{双炮巧胜炮双士} 
\end{wrapfigure}
(图\ref{双炮巧胜炮双士}),此局巧胜要诀:类同上局。

1.炮三进二!..............

进炮下二路,控制黑方上士。

1. ..............,将4平5\qquad 2.帅六进一,炮4进1

3.炮三退八,将5进1 \qquad 4.炮三平五,将5平4

如改走将5平6,则帅六平五,再炮五平四,红胜。

5.前炮平六,士6进5 \qquad 6.帅六平五炮4平3

7.炮五平六

重炮杀,红胜。

\subsection{双炮仕胜炮双士}
(图\ref{双炮仕胜炮双士}),属例胜局。

l.炮七进二!炮2进7 \qquad 2.炮七平五!..............

海底叫将,扰乱黑,好棋!

2. ..............,士5进6

如改走士5进4,则仕六退五,将5退1,炮五平九,炮2平5,炮六平四,得士红方胜定。

3.炮六平四,炮2退5 \qquad 4.帅四进一,将5退1

5.炮五平九,将5进1 \qquad 6.仕六退五,炮2平1

7.炮四平五,将5平4 \qquad 8.帅四平五,士6进5

9.炮五平六,将4退1 \qquad 10.仕五进六,士5进4

11.帅五平四,炮1进5 \qquad 12.帅四退一,炮1退6

13.帅四退一,炮1进8 \qquad 14.炮六进一,炮1退8

15.帅四平五,炮1进1 \qquad 16.炮九平五!炮1进5

17.炮五退五,炮1平2 \qquad 18.炮五平六,士4退5

19.仕六退五

重炮杀,红胜。


\begin{figure}[!htbp]
\centering
	\begin{minipage}{0.45\textwidth}
	\centering
		\smallboard
		\begin{position}
		\rj{f}{2} \rs{d}{3} \rp{d}{1} \rp{c}{8}
		
		\bj{e}{8} \bs{e}{9} \bs{f}{10} \bp{b}{10}
		\end{position}
		\caption{}\label{双炮仕胜炮双士} 
	\end{minipage}
	\begin{minipage}{0.45\textwidth}
	\centering
		\smallboard
		\begin{position}
		\rj{d}{3} \rs{f}{3} \rp{d}{1} \rp{e}{4}
		
		\bj{f}{10} \bs{e}{9} \bs{d}{8} \bm{e}{6}
		\end{position}
		\caption{}\label{双炮仕胜马双士}  
	\end{minipage}
\end{figure}
\subsection{双炮仕胜马双士}
(图\ref{双炮仕胜马双士}),此局属例胜局,红方用炮兑马或破士,均可取胜。

1.炮六平四!士5进6

如改走马5退6,则炮四进七兑马,红方胜定。

2.炮五退二,马5进3 \qquad 3.帅六平五,马3退4

4.炮五退一,士4退5

如改走将6进1,则炮五平六!马4进5,炮四进七!将6进1,炮六平四,将6平5,帅四平五,红胜。

5.帅五平六,马4进3 \qquad 6.帅六退一,马3进2

7.帅六进一,马2退4 \qquad 8.炮五进三,马4退5

9.炮五进五

破士,红胜。

\subsection{双炮仕相全胜马双象}
(图\ref{双炮仕相全胜马双象}),属例胜局。

1.炮五平三!马7进6 \qquad 2.帅五退一,象7进9

如改走马6进7,则炮三进一!象7进9,炮九平五,象9进7,炮五平三,马7退6,前炮进四打象,胜法同正变。

3,炮九平五,象9进7 \qquad 4.炮三进五!象5进7

5,炮五平六

红胜。

\begin{figure}[!htbp]
\centering
	\begin{minipage}{0.45\textwidth}
	\centering
		\smallboard
		\begin{position}
		\rj{e}{3} \rs{d}{3} \rs{f}{3} \rp{e}{1} \rp{a}{1} \rx{c}{5} \rx{g}{5}
		
		\bj{d}{10} \bx{e}{8} \bx{g}{10} \bm{g}{7}
		\end{position}
		\caption{}\label{双炮仕相全胜马双象} 
	\end{minipage}
	\begin{minipage}{0.45\textwidth}
	\centering
		\smallboard
		\begin{position}
		\rj{e}{1} \rs{d}{3} \rs{f}{3} \rp{a}{2} \rp{h}{2} \rx{e}{3} \rx{g}{5}
		
		\bj{e}{10} \bx{e}{8} \bx{g}{6} \bp{b}{2}
		\end{position}
		\caption{}\label{双炮仕相全巧胜炮双象}  
	\end{minipage}
\end{figure}
\subsection{双炮仕相全巧胜炮双象}
(图\ref{双炮仕相全巧胜炮双象}),此局巧胜要诀:困炮攻象取胜,着法曲折。

1.炮二平八!象5进3 \qquad 2.炮九退一!将5进1

3.相三退一,将S退1 \qquad 4.相一退三,将5进1

5.仕六退五,将5平4 \qquad 6.帅五平四,将4平5

7.帅四进一,将5退1 \qquad 8.仕五退六,将5平6

9.相五退七,将6平5 \qquad 10.仕四退五,将5进1

11.仕五进六,将5退1 \qquad 12.帅四进一,将5进l

13.相三进五,将5退1 \qquad 14.相五进七,将5进1

15.相七进五,将5退1 \qquad 16.仕六退五,将5进1

l7.仕五退四,将5退1 \qquad 18.仕六进五,将5进1

19.仕五进六,将5退1 \qquad 20.炮九进一,象7退5

21.炮九进二!..............

红方先用双炮困住黑炮,一系列调整帅、仕、相,着法精妙。现在高炮准备展开中路攻势。

21. ..............,炮8平9 \qquad 22.炮九平五,象5退7

23.相七退九!炮1退1 \qquad 24.相五进三,将5进1

25.炮五退三,将5退1 \qquad 26.炮八平三!象7进9

27.相三退五

黑必丢象,红胜。

\section{炮兵类}
\subsection{炮底兵巧胜双士}
(图\ref{炮底兵巧胜双士}),此局巧胜要诀:炮兵禁将。

1.炮八进六!..............

要着!防止黑动士。

1. ..............,将6退1 \qquad 2.兵二平三,将6进1

3.炮八平七

困毙,红胜。

\begin{figure}[!htbp]
\centering
	\begin{minipage}{0.45\textwidth}
	\centering
		\smallboard
		\begin{position}
		\rj{e}{2} \rp{b}{3} \rb{h}{10}
		
		\bj{f}{9} \bs{d}{8} \bs{f}{8}
		\end{position}
		\caption{}\label{炮底兵巧胜双士} 
	\end{minipage}
	\begin{minipage}{0.45\textwidth}
	\centering
		\smallboard
		\begin{position}
		\rj{e}{2} \rs{d}{3} \rx{e}{3} \rp{e}{1} \rb{e}{10}
		
		\bj{e}{9} \bs{d}{8} \bx{i}{8}
		\end{position}
		\caption{}\label{炮底兵仕相胜士象}  
	\end{minipage}
\end{figure}
\subsection{炮底兵仕相胜士象}
(图\ref{炮底兵仕相胜士象}),属例胜局。

1.炮五平六!将5平4

如改走将5进1,则帅五平四,士4退5,炮六平五,将5平4,炮五进八得士,红胜。

2.相五进三,象9进7 \qquad 3.炮六平五,象7退9

4.炮五平三!士4退5 \qquad 5.帅五退一,士5进4

6.炮三进三,士4退5 \qquad 7.炮三退二,士5进4

8.兵五平四,..............

以上红运炮、分兵,意在控制黑象。

8. ..............,士4退5 \qquad 9.炮三平六,士5进4

10.帅五平六!象9进7 \qquad 11.炮六平三,象7退9

12.仕六退五!将4平5 \qquad 13.仕五进四,将5进1

14.相三退五,象9进7 \qquad 15.帅六进一,象7退9

16.帅六进一,将5平6 \qquad 17.炮三平六,士4退5

18.炮六平四,将6平5 \qquad 19.炮四平五,将5平6

20.炮五进八

得士,红胜。

\subsection{炮低兵巧胜单象}
\begin{wrapfigure}{r}{5cm}
\centering
\vspace{-1.5cm}
\smallboard
\begin{position}
\rj{e}{2} \rp{e}{3} \rb{b}{8}

\bj{f}{8} \bx{g}{6}
\end{position}
\caption{}\label{炮低兵巧胜单象} 
\end{wrapfigure}
(图\ref{炮低兵巧胜单象}),此局巧胜要诀:炮兵控象。

1.炮五平九!将6退1 \qquad 2.炮九进五,将6退1

3.兵八平七,将6进1 \qquad 4.兵七平六,将6退1

5.炮九退一,将6进1 \qquad 6.炮九平四!将6退1

7.兵六平五,将6平5 \qquad 8.帅五平六,象7退9

9.兵五平四,将5进1

如改走象9退7,则兵四进一,象7进9,炮四平二!黑欠行,红胜。

10.炮四平二!象9退7 \qquad 炮二进二,象7进5

12.帅六平五

得象,红胜

\subsection{炮低兵巧胜士象}
(图\ref{炮低兵巧胜士象}),此局巧胜要诀:运炮控将、象取胜。

1.炮二进七!象7进9 \qquad 2.炮二平八,象9进7

3.炮八退二!将4退1 \qquad 4.兵四平五!

黑必失象,红胜。

\begin{figure}[!htbp]
\centering
	\begin{minipage}{0.45\textwidth}
	\centering
		\smallboard
		\begin{position}
		\rj{d}{1} \rp{h}{3} \rb{f}{9}
		
		\bj{d}{9} \bs{d}{8} \bx{g}{10}
		\end{position}
		\caption{}\label{炮低兵巧胜士象}  
	\end{minipage}
	\begin{minipage}{0.45\textwidth}
	\centering
		\smallboard
		\begin{position}
		\rj{d}{3} \rp{i}{2} \rb{d}{9}
		
		\bj{e}{8} \bs{e}{9} \bs{d}{10}
		\end{position}
		\caption{}\label{炮低兵巧胜双士}  
	\end{minipage}
\end{figure}
\subsection{炮低兵巧胜双士}
(图\ref{炮低兵巧胜双士}),此局巧胜要诀:运炮海底控中,搏士取胜。

1.炮一进八!土5进4 \qquad 2.炮一平五!..............

平中炮控士,抢夺中帅,乃获胜关键之着。

2. ..............,将5平6 \qquad 3.帅六平五,士4退5

4.炮五平一,土5进4 \qquad 5.炮一退三,将6退1

6.炮一平九,士4退5

如改走将6退1,则炮九进二,士4进5,炮九平五,红胜。

7.炮九进二!将6进1 \qquad 8.炮九平五!将6退1

9.炮五退二,士4进5 \qquad 10.炮五平九,士5进4

11.炮九进二,将6退1 \qquad 12.炮九乎八

红胜。

\subsection{炮低兵相巧胜单象}
(图\ref{炮低兵相巧胜单象}),此局巧胜要诀:运炮捉象而胜。


\end{document}
..............

\begin{wrapfigure}{r}{5cm}
\centering
\vspace{-1.5cm}
\smallboard
\begin{position}
\rj{f}{1} \rb{e}{8}

\bj{f}{10} \bs{f}{8} \bs{d}{8}
\end{position}
\caption{}\label{高兵巧胜双士} 
\end{wrapfigure}

\begin{figure}[!htbp]
\centering
	\begin{minipage}{0.45\textwidth}
	\centering
		\smallboard
		\begin{position}
		\rj{e}{1} \rb{f}{7} \rb{h}{9}
		
		\bj{f}{9} \bx{e}{8} \bx{c}{6}
		\end{position}
	\caption{} \label{高低兵胜双象} 
	\end{minipage}
	\begin{minipage}{0.45\textwidth}
	\centering
		\smallboard
		\begin{position}
		\rj{e}{1} \rb{e}{6} \rb{f}{9}
		
		\bj{e}{10} \bx{e}{8} \bs{e}{9} \bs{d}{10}
		\end{position}
	\caption{}\label{高低兵巧胜单缺象} 
	\end{minipage}
\end{figure}

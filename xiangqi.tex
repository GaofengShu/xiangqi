\documentclass[UTF8]{ctexbook}

%\usepackage{graphicx,booktabs}%插入图表功能
\usepackage[top=1in , bottom =1in , left=1.25in , right =1.25in]{geometry}
\usepackage{cchess}
\usepackage{subfig}
\usepackage{wrapfig}

\newcommand{\rc}[2]{\piece{#1}{#2}{r}} \newcommand{\bc}[2]{\piece{#1}{#2}{R}}
\renewcommand{\rm}[2]{\piece{#1}{#2}{n}} \newcommand{\bm}[2]{\piece{#1}{#2}{N}}
\newcommand{\rx}[2]{\piece{#1}{#2}{b}} \newcommand{\bx}[2]{\piece{#1}{#2}{B}}
\newcommand{\rs}[2]{\piece{#1}{#2}{g}} \newcommand{\bs}[2]{\piece{#1}{#2}{G}}
\newcommand{\rj}[2]{\piece{#1}{#2}{k}} \newcommand{\bj}[2]{\piece{#1}{#2}{K}}
\newcommand{\rp}[2]{\piece{#1}{#2}{c}} \newcommand{\bp}[2]{\piece{#1}{#2}{C}}
\newcommand{\rb}[2]{\piece{#1}{#2}{p}} \newcommand{\bb}[2]{\piece{#1}{#2}{P}}


\title{象棋胜和定式}
\author{金启昌\quad 杨\quad 典}
\date{1999 年 7 月  \\ \vspace{14cm} 北京体育大学出版社
}

\begin{document}

\maketitle

\thispagestyle{empty}
\begin{center}
{\zihao{2}\textbf{前言} }
\end{center}
\vspace{3cm}

象棋对弈的全过程,一般要经历开局(布局)、中局、残局三个阶段。本丛书专门论述残局实战技巧和胜和规律。

残局,作为全局中的结束阶段,一局棋的胜、负、和往往到此阶段决出结果。残晑中双方的对峙形势,往往是棋手在中局甚至布局时确定战略方针的依据。但残局又具有与其它阶段不同的特点和胜和规律。残局子力少,局势相对较为筒明,利于锻炼各种能力。所以,古往今来的众多棋手都把残局作为必修的基本功。

为帮助广大象棋爱好者丰富残局的知识和经验,掌握残局的攻守技巧和胜和规律,尽快提高驾驭残局的能力,特策划编写出版了这套《象棋残局战法丛书》。

《象棋残局战法丛书》共分5册:《象棋残局攻杀技巧》、《象棋残局妙手精华》、《象棋残局名局战术》、《象棋残局胜和定式》、《象棋残局生死棋型》.前3册战例均选自国内外重大象棋比赛的名手对局,实战性和实用性皆较强。《胜和定式》大部分来自实用残局;《生死棋型》是把常见的江湖残局归纳整理,进一步升华。

残局是以多种多样的战术手段为主体的战斗,“残局妙手”往往是在胜和要点的关键时刻起决定性作用,它是棋手们呕心呖血的艺术结局,是残局中各种各样的战术的精华,颇能引人入胜、引人入迷。本丛书以实战性和实用性为宗旨,以实战技巧和胜和规律为主线,选用了1000余例精妙残局,精心归纳整理,就象棋残局的不同攻杀特点和胜和规律,及与此有关的战略战术运用要点,分别详加阐述和印证,完整地展示了象棋残局研究的全貌,是一部难得的上佳之作。

由于写作时间仓促,书中不当之处,恳请读者、方家见谅并不吝指正。

如果你通过本丛书的学习而有所裨益,笔者将感到欣慰。

~\\

编著者

1999年春于北京 \footnote{2019年10月20日,舒高峰使用 \LaTeX{} 及 cchess 宏包整理而成。}


\clearpage
\tableofcontents
\thispagestyle{empty}


\clearpage
\chapter{胜局定式篇}
\setcounter{page}{1}
\section{兵类}
\subsection{高兵巧胜单士}
\begin{wrapfigure}{r}{5cm}
\centering
\vspace{-.8cm}
\smallboard
\begin{position}
\rj{f}{1} \rb{f}{7}

\bj{f}{10} \bs{d}{10}
\end{position}
\caption{} \label{高兵巧胜单士} 
\end{wrapfigure}
(图\ref{高兵巧胜单士}),单高兵对单士,正常形势下成为和局。当形势特殊时,单高兵攻单士属于巧胜局。此局巧胜要诀:(一)兵逼九宫胁将;(二)帅占中路,小兵灭士。

1. 兵四进一,士4进5

2. 兵四进一,将6平5

3. 帅四平五,将5平4

4. 兵四平五

红胜。

\subsection{高兵相巧胜单士}
\begin{wrapfigure}{r}{5cm}
\centering
\vspace{-.8cm}
\smallboard
\begin{position}
\rj{f}{1} \rx{e}{3} \rb{e}{7}

\bj{e}{10} \bs{d}{10}
\end{position}
\caption{} \label{高兵相巧胜单士} 
\end{wrapfigure}
(图\ref{高兵相巧胜单士}),此局巧胜要诀:(一)兵入中官控将,形成左(右)兵右(左)帅的势态;(二)等待黑方落土后,兵从无士一边进入象腰。

1.兵五进一,士4进5

2.帅四平五,士5退4

3.兵五平四,将5平6

4.帅五平四,士4进5

5.兵四进一,将6平5

6.帅四平五,士5退6

7.帅五平六

形成“左帅右兵”,已成胜式。

7. ..............,士6进5

8. 相五退三,士5退6

9. 帅六进一,..............

落相、进帅,运用等着入局。

9. ..............,士6进5

10.帅六平五

红胜。
\subsection{高兵仕巧胜单士}
\begin{wrapfigure}{r}{5cm}
\centering
\vspace{-.8cm}
\smallboard
\begin{position}
\rj{e}{1} \rs{e}{2} \rb{f}{7}

\bj{e}{10} \bs{d}{10}
\end{position}
\caption{}\label{高兵仕巧胜单士} 
\end{wrapfigure}

(图\ref{高兵仕巧胜单士}),此局巧胜要诀与上局类同,其中红方仕、兵亦有定的作用,帅借仕(兵)遮头,起控制或助攻之妙用。

1.兵四进一,将5进1

如改走士4进5(如将5平6,则帅五平四,士4进5,兵四进一,将6平5,帅四平五红胜),则兵四进一,士5进6,帅五平六,士6退5,仕五退四,士5进6,帅六进一,士6退5,帅六平五,红胜。

2.帅五平六!将5退1

3.兵四进一,士4进5

4.仕五退四,士5进4

5.帅六进一,将5平4

6.兵四平五

红胜。

\subsection{高兵巧胜双士}
\begin{wrapfigure}{r}{5cm}
\centering
\vspace{-.8cm}
\smallboard
\begin{position}
\rj{f}{1} \rb{e}{8}

\bj{f}{10} \bs{f}{8} \bs{d}{8}
\end{position}
\caption{}\label{高兵巧胜双士} 
\end{wrapfigure}
(图\ref{高兵巧胜双士}),此局巧胜要诀:(一)小兵破士(左、右士);(二)用帅走闲等着,得士或使黑方欠行。

【第一种胜法】

1.兵五平四,将6平5

2.兵四进一,将5平4

3.帅四平五,将4进1

4.帅五平六,将4退1

5.兵四平五

红胜。

【第二种胜法】

1.兵五平六,将6平5

2.兵六进一,士6退5

3.帅四平五,将5平6

4.兵六乎五

红胜。

\subsection{高低兵巧胜士象}
(图\ref{高低兵巧胜士象}),此局巧胜要诀:(一)首着红帅牵制将、土;(二)后兵插进,控将得士。

1.帅五平六!将4平5

如改走将4进1,则后兵进一,象9进7,帅六进一,象7退9,后兵平五,红胜。

2.后兵进一,象9进7 \qquad 3.帅六进一,士4退5

4.帅六平五,象7退5 \qquad 5.前兵平五,将5进1

6.兵四平五

红胜。

\begin{figure}[!htbp]
\centering
	\begin{minipage}{0.45\textwidth}
	\centering
		\smallboard
		\begin{position}
		\rj{e}{1} \rb{f}{7} \rb{f}{9}
		
		\bj{d}{10} \bs{d}{8} \bx{i}{8}
		\end{position}
	\caption{} \label{高低兵巧胜士象} 
	\end{minipage}
	\begin{minipage}{0.45\textwidth}
	\centering
		\smallboard
		\begin{position}
		\rj{f}{1} \rb{e}{7} \rb{c}{10}
		
		\bj{e}{10} \bs{d}{10} \bs{f}{10}
		\end{position}
	\caption{}\label{高底兵巧胜双士} 
	\end{minipage}
\end{figure}

\subsection{高底兵巧胜双士}
(图\ref{高底兵巧胜双士}),此局巧胜要诀:(一)冲中兵;(二)用帅控将。

1.兵五进一,士6进5 \qquad 2.帅四进一!士5退6

3.兵五平六,士4进5 \qquad 4.兵六进一

或走帅四平五!红胜。


\subsection{高低兵胜双士}
(图\ref{高低兵胜双士}),属例胜(稳胜)局。

1.兵七进一!将4退1

如改走将4进1,则帅四平五!以后再兵四平五,红胜。

2.帅四平五,士5进4 \qquad 3.兵四进一,士4退5

4.兵四平五,将4平5 \qquad 5.兵五进一!士6进5

6.兵七平六

红胜。

\begin{figure}[!htbp]
\centering
	\begin{minipage}{0.45\textwidth}
	\centering
		\smallboard
		\begin{position}
		\rj{f}{1} \rb{f}{7} \rb{c}{8}
		
		\bj{d}{9} \bs{e}{9} \bs{f}{10}
		\end{position}
	\caption{} \label{高低兵胜双士} 
	\end{minipage}
	\begin{minipage}{0.45\textwidth}
	\centering
		\smallboard
		\begin{position}
		\rj{f}{1} \rb{f}{9} \rb{c}{8}
		
		\bj{d}{9} \bx{a}{8} \bx{e}{8}
		\end{position}
	\caption{}\label{双低兵巧胜双象} 
	\end{minipage}
\end{figure}

\subsection{双低兵巧胜双象}
(图\ref{双低兵巧胜双象}),此局巧胜要诀:用帅控制黑方中象,分兵掠象破局。

1.帅四平五!象1进3 \qquad 2.兵七平八!象3退1

3.兵八平九

红胜。

\subsection{高低兵胜双象}
(图\ref{高低兵胜双象})高低兵攻双象,属例胜局。

1.兵二平三,将6平5 \qquad 2.兵四平五,象5退3

3.帅五平四!象3进5 \qquad 4.兵三平四,将5平4

5.帅四平五,象5进7 \qquad 6.兵五平六,象7退5

7.帅五平六

伏兵六进一冲兵杀,红胜。

\begin{figure}[!htbp]
\centering
	\begin{minipage}{0.45\textwidth}
	\centering
		\smallboard
		\begin{position}
		\rj{e}{1} \rb{f}{7} \rb{h}{9}
		
		\bj{f}{9} \bx{e}{8} \bx{c}{6}
		\end{position}
	\caption{} \label{高低兵胜双象} 
	\end{minipage}
	\begin{minipage}{0.45\textwidth}
	\centering
		\smallboard
		\begin{position}
		\rj{e}{1} \rb{e}{6} \rb{f}{9}
		
		\bj{e}{10} \bx{e}{8} \bs{e}{9} \bs{d}{10}
		\end{position}
	\caption{}\label{高低兵巧胜单缺象} 
	\end{minipage}
\end{figure}

\subsection{高低兵巧胜单缺象}
(图\ref{高低兵巧胜单缺象}),此局巧胜要诀:用高兵胁象,然后向低兵靠拢,控象、士禁毙入局。

1.兵五进一,象5退3 \qquad 2.兵五平四,象3进5

3.帅五进一!..............

等着!控制黑方士象。

3. ..............,象5退3 \qquad 4.后兵进一,象3进1

5.前兵平三!..............

分兵是获胜的关键着法。亦可走后兵平三!

5. ..............,象1进3

如改走将5平6,则兵四进一再兵三进一,红胜。

6.兵四进一,象3退5 \qquad 7.兵三进一!象5进3

8.兵三平四

红胜。
\subsection{高低兵巧胜单缺士}
\begin{wrapfigure}{r}{5cm}
\centering
\vspace{-.8cm}
\smallboard
\begin{position}
\rj{e}{1} \rb{f}{7} \rb{d}{9}

\bj{e}{10} \bs{e}{9} \bx{e}{8} \bx{g}{10}
\end{position}
\caption{}\label{高低兵巧胜单缺士} 
\end{wrapfigure}
(图\ref{高低兵巧胜单缺士}),此局巧胜要诀:分兵左右夹击,控制黑士,形成“兵前帅后”入局。

1.兵四平三!士5进4

如改走象5进3,则兵四进一,象3退5,兵四进一,士5进6,帅五平六,红胜。

2.兵三进一,将5平6

3.兵三平四,象5进7

4.帅五平四!象7退5

5.兵四进一

红胜。
\subsection{高低兵巧胜士象全(一)}
(图\ref{高低兵巧胜士象全(一)}),此局巧胜要诀:妙用高兵掠黑方双象。

1.兵九平八!象1进3 \qquad 2.兵八平七,象3进1

3.兵七平八,象1退3 \qquad 4.兵八进一,象3进1

5.帅五进一!..............

等着!逼黑方动象。

5. ..............,象1退3 \qquad 6.兵八平九,象3进1

7.兵九进一

红胜。
\begin{figure}[!htbp]
\centering
	\begin{minipage}{0.45\textwidth}
	\centering
		\smallboard
		\begin{position}
		\rj{e}{1} \rb{a}{6} \rb{d}{9}
		
		\bj{e}{10} \bs{e}{9} \bs{f}{10} \bx{c}{10} \bx{a}{8}
		\end{position}
	\caption{} \label{高低兵巧胜士象全(一)} 
	\end{minipage}
	\begin{minipage}{0.45\textwidth}
	\centering
		\smallboard
		\begin{position}
		\rj{e}{1} \rb{b}{7} \rb{f}{9}
		
		\bj{d}{9} \bs{e}{9} \bs{d}{8} \bx{e}{8} \bx{g}{6}
		\end{position}
	\caption{}\label{高低兵巧胜士象全(二)} 
	\end{minipage}
\end{figure}
\subsection{高低兵巧胜士象全(二)}
(图\ref{高低兵巧胜士象全(二)}),此局巧胜要诀:首着出帅控六路,冲兵破士。

1.帅五平六!..............

获胜要着!否则黑将4退1,立即成和。

1. ..............,土5进6 \qquad 2.兵八进一,将4退1

3.兵八平七,将4平5 \qquad 4.兵七平六,士6退5

5.兵六进一,十5进4 \qquad 6.帅六平五,象7退9

7. 帅五平四

红胜。

\subsection{双高兵胜单炮}








\end{document}
..............

\begin{wrapfigure}{r}{5cm}
\centering
\vspace{-.8cm}
\smallboard
\begin{position}
\rj{f}{1} \rb{e}{8}

\bj{f}{10} \bs{f}{8} \bs{d}{8}
\end{position}
\caption{}\label{高兵巧胜双士} 
\end{wrapfigure}

\begin{figure}[!htbp]
\centering
	\begin{minipage}{0.45\textwidth}
	\centering
		\smallboard
		\begin{position}
		\rj{e}{1} \rb{f}{7} \rb{h}{9}
		
		\bj{f}{9} \bx{e}{8} \bx{c}{6}
		\end{position}
	\caption{} \label{高低兵胜双象} 
	\end{minipage}
	\begin{minipage}{0.45\textwidth}
	\centering
		\smallboard
		\begin{position}
		\rj{e}{1} \rb{e}{6} \rb{f}{9}
		
		\bj{e}{10} \bx{e}{8} \bs{e}{9} \bs{d}{10}
		\end{position}
	\caption{}\label{高低兵巧胜单缺象} 
	\end{minipage}
\end{figure}

\documentclass[UTF8]{ctexbook}

%\usepackage{graphicx,booktabs}%插入图表功能
\usepackage[top=1in , bottom =1in , left=1.25in , right =1.25in]{geometry}
\usepackage{cchess}
\usepackage{subfig}
\usepackage{wrapfig}

\newcommand{\rc}[2]{\piece{#1}{#2}{r}} \newcommand{\bc}[2]{\piece{#1}{#2}{R}}
\renewcommand{\rm}[2]{\piece{#1}{#2}{n}} \newcommand{\bm}[2]{\piece{#1}{#2}{N}}
\newcommand{\rx}[2]{\piece{#1}{#2}{b}} \newcommand{\bx}[2]{\piece{#1}{#2}{B}}
\newcommand{\rs}[2]{\piece{#1}{#2}{g}} \newcommand{\bs}[2]{\piece{#1}{#2}{G}}
\newcommand{\rj}[2]{\piece{#1}{#2}{k}} \newcommand{\bj}[2]{\piece{#1}{#2}{K}}
\newcommand{\rp}[2]{\piece{#1}{#2}{c}} \newcommand{\bp}[2]{\piece{#1}{#2}{C}}
\newcommand{\rb}[2]{\piece{#1}{#2}{p}} \newcommand{\bb}[2]{\piece{#1}{#2}{P}}


\title{象棋胜和定式}
\author{金启昌\quad 杨\quad 典}
\date{1999 年 7 月  \\ \vspace{14cm} 北京体育大学出版社
}

\begin{document}

\maketitle

\thispagestyle{empty}
\begin{center}
{\zihao{2}\textbf{前言} }
\end{center}
\vspace{3cm}

象棋对弈的全过程,一般要经历开局(布局)、中局、残局三个阶段。本丛书专门论述残局实战技巧和胜和规律。

残局,作为全局中的结束阶段,一局棋的胜、负、和往往到此阶段决出结果。残晑中双方的对峙形势,往往是棋手在中局甚至布局时确定战略方针的依据。但残局又具有与其它阶段不同的特点和胜和规律。残局子力少,局势相对较为筒明,利于锻炼各种能力。所以,古往今来的众多棋手都把残局作为必修的基本功。

为帮助广大象棋爱好者丰富残局的知识和经验,掌握残局的攻守技巧和胜和规律,尽快提高驾驭残局的能力,特策划编写出版了这套《象棋残局战法丛书》。

《象棋残局战法丛书》共分5册:《象棋残局攻杀技巧》、《象棋残局妙手精华》、《象棋残局名局战术》、《象棋残局胜和定式》、《象棋残局生死棋型》.前3册战例均选自国内外重大象棋比赛的名手对局,实战性和实用性皆较强。《胜和定式》大部分来自实用残局;《生死棋型》是把常见的江湖残局归纳整理,进一步升华。

残局是以多种多样的战术手段为主体的战斗,“残局妙手”往往是在胜和要点的关键时刻起决定性作用,它是棋手们呕心呖血的艺术结局,是残局中各种各样的战术的精华,颇能引人入胜、引人入迷。本丛书以实战性和实用性为宗旨,以实战技巧和胜和规律为主线,选用了1000余例精妙残局,精心归纳整理,就象棋残局的不同攻杀特点和胜和规律,及与此有关的战略战术运用要点,分别详加阐述和印证,完整地展示了象棋残局研究的全貌,是一部难得的上佳之作。

由于写作时间仓促,书中不当之处,恳请读者、方家见谅并不吝指正。

如果你通过本丛书的学习而有所裨益,笔者将感到欣慰。

~\\

编著者

1999年春于北京 \footnote{2019年10月20日,Gaofeng Shu 使用 \LaTeX{} 及 cchess 宏包整理而成。}


\clearpage
\tableofcontents
\thispagestyle{empty}


\clearpage
\chapter{胜局定式篇}
\setcounter{page}{1}
\section{兵类}
\subsection{高兵巧胜单士}
\begin{wrapfigure}{r}{5cm}
\centering
\vspace{-1.5cm}
\smallboard
\begin{position}
\rj{f}{1} \rb{f}{7}

\bj{f}{10} \bs{d}{10}
\end{position}
\caption{} \label{高兵巧胜单士} 
\end{wrapfigure}
(图\ref{高兵巧胜单士}),单高兵对单士,正常形势下成为和局。当形势特殊时,单高兵攻单士属于巧胜局。此局巧胜要诀:(一)兵逼九宫胁将;(二)帅占中路,小兵灭士。

1. 兵四进一,士4进5

2. 兵四进一,将6平5

3. 帅四平五,将5平4

4. 兵四平五

红胜。

\subsection{高兵相巧胜单士}
\begin{wrapfigure}{r}{5cm}
\centering
\vspace{-1.5cm}
\smallboard
\begin{position}
\rj{f}{1} \rx{e}{3} \rb{e}{7}

\bj{e}{10} \bs{d}{10}
\end{position}
\caption{} \label{高兵相巧胜单士} 
\end{wrapfigure}
(图\ref{高兵相巧胜单士}),此局巧胜要诀:(一)兵入中官控将,形成左(右)兵右(左)帅的势态;(二)等待黑方落后,兵从无士一边进入象腰。

1.兵五进一,士4进5 \qquad 2.帅四平五,士5退4

3.兵五平四,将5平6 \qquad 4.帅五平四,士4进5

5.兵四进一,将6平5 \qquad 6.帅四平五,士5退6

7.帅五平六

形成“左帅右兵”,已成胜式。

7. ..............,士6进5

8. 相五退三,士5退6

9. 帅六进一,..............

落相、进帅,运用等着入局。

9. ..............,士6进5

10.帅六平五

红胜。
\subsection{高兵仕巧胜单士}
\begin{wrapfigure}{r}{5cm}
\centering
\vspace{-1.5cm}
\smallboard
\begin{position}
\rj{e}{1} \rs{e}{2} \rb{f}{7}

\bj{e}{10} \bs{d}{10}
\end{position}
\caption{}\label{高兵仕巧胜单士} 
\end{wrapfigure}

(图\ref{高兵仕巧胜单士}),此局巧胜要诀与上局类同,其中红方仕、兵亦有定的作用,帅借仕(兵)遮头,起控制或助攻之妙用。

1.兵四进一,将5进1

如改走士4进5(如将5平6,则帅五平四,士4进5,兵四进一,将6平5,帅四平五红胜),则兵四进一,士5进6,帅五平六,士6退5,仕五退四,士5进6,帅六进一,士6退5,帅六平五,红胜。

2.帅五平六!将5退1 \qquad 3.兵四进一,士4进5

4.仕五退四,士5进4 \qquad 5.帅六进一,将5平4

6.兵四平五

红胜。

\subsection{高兵巧胜双士}
\begin{wrapfigure}{r}{5cm}
\centering
\vspace{-1.5cm}
\smallboard
\begin{position}
\rj{f}{1} \rb{e}{8}

\bj{f}{10} \bs{f}{8} \bs{d}{8}
\end{position}
\caption{}\label{高兵巧胜双士} 
\end{wrapfigure}
(图\ref{高兵巧胜双士}),此局巧胜要诀:(一)小兵破士(左、右士);(二)用帅走闲等着,得士或使黑方欠行。

【第一种胜法】

1.兵五平四,将6平5 \qquad 2.兵四进一,将5平4

3.帅四平五,将4进1 \qquad 4.帅五平六,将4退1

5.兵四平五

红胜。

【第二种胜法】

1.兵五平六,将6平5 \qquad 2.兵六进一,士6退5

3.帅四平五,将5平6

4.兵六乎五

红胜。

\subsection{高低兵巧胜士象}
(图\ref{高低兵巧胜士象}),此局巧胜要诀:(一)首着红帅牵制将、;(二)后兵插进,控将得士。

1.帅五平六!将4平5

如改走将4进1,则后兵进一,象9进7,帅六进一,象7退9,后兵平五,红胜。

2.后兵进一,象9进7 \qquad 3.帅六进一,士4退5

4.帅六平五,象7退5 \qquad 5.前兵平五,将5进1

6.兵四平五

红胜。

\begin{figure}[!htbp]
\centering
	\begin{minipage}{0.45\textwidth}
	\centering
		\smallboard
		\begin{position}
		\rj{e}{1} \rb{f}{7} \rb{f}{9}
		
		\bj{d}{10} \bs{d}{8} \bx{i}{8}
		\end{position}
	\caption{} \label{高低兵巧胜士象} 
	\end{minipage}
	\begin{minipage}{0.45\textwidth}
	\centering
		\smallboard
		\begin{position}
		\rj{f}{1} \rb{e}{7} \rb{c}{10}
		
		\bj{e}{10} \bs{d}{10} \bs{f}{10}
		\end{position}
	\caption{}\label{高底兵巧胜双士} 
	\end{minipage}
\end{figure}

\subsection{高底兵巧胜双士}
(图\ref{高底兵巧胜双士}),此局巧胜要诀:(一)冲中兵;(二)用帅控将。

1.兵五进一,士6进5 \qquad 2.帅四进一!士5退6

3.兵五平六,士4进5 \qquad 4.兵六进一

或走帅四平五!红胜。


\subsection{高低兵胜双士}
(图\ref{高低兵胜双士}),属例胜(稳胜)局。

1.兵七进一!将4退1

如改走将4进1,则帅四平五!以后再兵四平五,红胜。

2.帅四平五,士5进4 \qquad 3.兵四进一,士4退5

4.兵四平五,将4平5 \qquad 5.兵五进一!士6进5

6.兵七平六

红胜。

\begin{figure}[!htbp]
\centering
	\begin{minipage}{0.45\textwidth}
	\centering
		\smallboard
		\begin{position}
		\rj{f}{1} \rb{f}{7} \rb{c}{8}
		
		\bj{d}{9} \bs{e}{9} \bs{f}{10}
		\end{position}
	\caption{} \label{高低兵胜双士} 
	\end{minipage}
	\begin{minipage}{0.45\textwidth}
	\centering
		\smallboard
		\begin{position}
		\rj{f}{1} \rb{f}{9} \rb{c}{8}
		
		\bj{d}{9} \bx{a}{8} \bx{e}{8}
		\end{position}
	\caption{}\label{双低兵巧胜双象} 
	\end{minipage}
\end{figure}

\subsection{双低兵巧胜双象}
(图\ref{双低兵巧胜双象}),此局巧胜要诀:用帅控制黑方中象,分兵掠象破局。

1.帅四平五!象1进3 \qquad 2.兵七平八!象3退1

3.兵八平九

红胜。

\subsection{高低兵胜双象}
(图\ref{高低兵胜双象})高低兵攻双象,属例胜局。

1.兵二平三,将6平5 \qquad 2.兵四平五,象5退3

3.帅五平四!象3进5 \qquad 4.兵三平四,将5平4

5.帅四平五,象5进7 \qquad 6.兵五平六,象7退5

7.帅五平六

伏兵六进一冲兵杀,红胜。

\begin{figure}[!htbp]
\centering
	\begin{minipage}{0.45\textwidth}
	\centering
		\smallboard
		\begin{position}
		\rj{e}{1} \rb{f}{7} \rb{h}{9}
		
		\bj{f}{9} \bx{e}{8} \bx{c}{6}
		\end{position}
	\caption{} \label{高低兵胜双象} 
	\end{minipage}
	\begin{minipage}{0.45\textwidth}
	\centering
		\smallboard
		\begin{position}
		\rj{e}{1} \rb{e}{6} \rb{f}{9}
		
		\bj{e}{10} \bx{e}{8} \bs{e}{9} \bs{d}{10}
		\end{position}
	\caption{}\label{高低兵巧胜单缺象} 
	\end{minipage}
\end{figure}

\subsection{高低兵巧胜单缺象}
(图\ref{高低兵巧胜单缺象}),此局巧胜要诀:用高兵胁象,然后向低兵靠拢,控象、士禁毙入局。

1.兵五进一,象5退3 \qquad 2.兵五平四,象3进5

3.帅五进一!..............

等着!控制黑方士象。

3. ..............,象5退3 \qquad 4.后兵进一,象3进1

5.前兵平三!..............

分兵是获胜的关键着法。亦可走后兵平三!

5. ..............,象1进3

如改走将5平6,则兵四进一再兵三进一,红胜。

6.兵四进一,象3退5 \qquad 7.兵三进一!象5进3

8.兵三平四

红胜。
\subsection{高低兵巧胜单缺士}
\begin{wrapfigure}{r}{5cm}
\centering
\vspace{-1.5cm}
\smallboard
\begin{position}
\rj{e}{1} \rb{f}{7} \rb{d}{9}

\bj{e}{10} \bs{e}{9} \bx{e}{8} \bx{g}{10}
\end{position}
\caption{}\label{高低兵巧胜单缺士} 
\end{wrapfigure}
(图\ref{高低兵巧胜单缺士}),此局巧胜要诀:分兵左右夹击,控制黑士,形成“兵前帅后”入局。

1.兵四平三!士5进4

如改走象5进3,则兵四进一,象3退5,兵四进一,士5进6,帅五平六,红胜。

2.兵三进一,将5平6

3.兵三平四,象5进7

4.帅五平四!象7退5

5.兵四进一

红胜。
\subsection{高低兵巧胜士象全(一)}
(图\ref{高低兵巧胜士象全(一)}),此局巧胜要诀:妙用高兵掠黑方双象。

1.兵九平八!象1进3 \qquad 2.兵八平七,象3进1

3.兵七平八,象1退3 \qquad 4.兵八进一,象3进1

5.帅五进一!..............

等着!逼黑方动象。

5. ..............,象1退3 \qquad 6.兵八平九,象3进1

7.兵九进一

红胜。
\begin{figure}[!htbp]
\centering
	\begin{minipage}{0.45\textwidth}
	\centering
		\smallboard
		\begin{position}
		\rj{e}{1} \rb{a}{6} \rb{d}{9}
		
		\bj{e}{10} \bs{e}{9} \bs{f}{10} \bx{c}{10} \bx{a}{8}
		\end{position}
	\caption{} \label{高低兵巧胜士象全(一)} 
	\end{minipage}
	\begin{minipage}{0.45\textwidth}
	\centering
		\smallboard
		\begin{position}
		\rj{e}{1} \rb{b}{7} \rb{f}{9}
		
		\bj{d}{9} \bs{e}{9} \bs{d}{8} \bx{e}{8} \bx{g}{6}
		\end{position}
	\caption{}\label{高低兵巧胜士象全(二)} 
	\end{minipage}
\end{figure}
\subsection{高低兵巧胜士象全(二)}
(图\ref{高低兵巧胜士象全(二)}),此局巧胜要诀:首着出帅控六路,冲兵破士。

1.帅五平六!..............

获胜要着!否则黑将4退1,立即成和。

1. ..............,5进6 \qquad 2.兵八进一,将4退1

3.兵八平七,将4平5 \qquad 4.兵七平六,士6退5

5.兵六进一,十5进4 \qquad 6.帅六平五,象7退9

7. 帅五平四

红胜。

\subsection{双高兵胜单炮}
\begin{wrapfigure}{r}{5cm}
\centering
\vspace{-1.5cm}
\smallboard
\begin{position}
\rj{e}{1} \rb{e}{7} \rb{f}{7}

\bj{e}{9} \bp{e}{10}
\end{position}
\caption{}\label{双高兵胜单炮} 
\end{wrapfigure}
(图 \ref{双高兵胜单炮}),属例胜局。

1.帅五进一!..............

进帅等着!迫使黑方动将。

1. ..............,将5平4

如改走炮5平9,则兵五进一,将5平4,兵四进一,炮9进1,兵四进一,红胜。

2.兵五平六,炮5平4 \qquad 3.兵六平七!炮4平5

4.兵四进一,炮5进1 \qquad 5.兵七进一,炮5退1

6.帅五平四,炮5平9 \qquad 7.兵四平五,炮9进1

8.兵七平六,将4退1 \qquad 9.兵五进一

再帅四平五,红胜。
\subsection{双高兵巧胜炮象}
\begin{wrapfigure}{r}{5cm}
\centering
\vspace{-1.5cm}
\smallboard
\begin{position}
\rj{e}{1} \rb{f}{7} \rb{d}{7}

\bj{f}{10} \bx{i}{8} \bp{e}{6}
\end{position}
\caption{} \label{双高兵巧胜炮象} 
\end{wrapfigure}
(图15),此局巧胜要诀:双兵逼宫,用帅牵黑炮、将。

1.兵六进一,象9进7

另有两种着法:①将6进1,兵六平五,象9进7,兵四进一,将6退1,兵五进一,红胜;②炮5退4,兵四进一,象9进7,帅五平四,炮5平1,兵四进一,将6平5,兵六进一,红胜。

2.兵四进一,将6平5 \qquad 3.兵六进一,炮5进2

4.兵四进一,象7退5 \qquad 5.帅五平六,炮5平4

6.兵六平五,将5平4 \qquad 7.兵四进一

红胜。


\subsection{双高兵巧胜炮士}
(图\ref{双高兵巧胜炮士}),此局巧胜要诀:冲中兵巧吃黑士。

1.兵五进一!士5进6 \qquad 2.帅四平五,炮6平5

如改走炮6平8,则兵五平六!将4退1,兵七进一,士6退5,兵七平六,将4平5,前兵平五,将5平6,兵六平五,红胜。

3.兵五平四,炮5进1 \qquad 4.帅五平四,炮5进2

5.兵四平五,炮5平8 \qquad 6.兵七进一,将4退1

7.兵五进一

红胜。
\begin{figure}[!htbp]
\centering
	\begin{minipage}{0.45\textwidth}
	\centering
		\smallboard
		\begin{position}
		\rj{f}{1} \rb{e}{7} \rb{c}{8}
		
		\bj{d}{9} \bs{e}{9} \bp{f}{10}
		\end{position}
	\caption{}\label{双高兵巧胜炮士} 
	\end{minipage}
	\begin{minipage}{0.45\textwidth}
	\centering
		\smallboard
		\begin{position}
		\rj{d}{2} \rb{d}{7} \rb{g}{9}
		
		\bj{e}{9} \bm{e}{6}
		\end{position}
	\caption{} \label{高低兵巧胜单马} 
	\end{minipage}
\end{figure}

\subsection{高低兵巧胜单马}
(图\ref{高低兵巧胜单马}),双高兵攻单马是例胜,高低兵属巧胜。此局巧胜要诀:双兵逼将,用帅控马助攻。

1.帅六平五!将5平4

如改走将5进1(如将5退1,则兵三平四!),则帅五退一,将5平6,兵六平五,红胜。

2.兵三平四,马5进4 \qquad 3.帅五平六,马4退5

如改走马4进6,则帅六退一,马6退4,兵六平五,将4退1,兵五进一,将4平5,兵五进一再兵四进一,红胜。

4.兵六进!..............

弃兵困马,妙手入局。

4. ..............,马5退4 \qquad 5.帅六退一,将4退1

6.兵四平五

黑方欠行,红胜。
\subsection{双低兵巧胜马士}
(图\ref{双低兵巧胜马士}),此局巧胜要诀:用帅助攻,双兵胁士。

1.帅四平五!马7进5

如改走马7退5,则兵三平四,士5进6,帅五进一(或帅五平六),士6退5,兵六平五,将5平4,兵四进一,马5退7,帅五退一,红胜。

2.兵三平四,马5进4 \qquad 3.帅五退一,马6退5

4.兵六平五,将5平4 \qquad 5.兵四进一,马5退6

6.帅五进一,马6进5 \qquad 7.兵四平五

红胜。

\begin{figure}[!htbp]
\centering
	\begin{minipage}{0.45\textwidth}
	\centering
		\smallboard
		\begin{position}
		\rj{f}{2} \rb{d}{9} \rb{g}{9}
		
		\bj{e}{10} \bs{e}{9} \bm{g}{7}
		\end{position}
	\caption{} \label{双低兵巧胜马士} 
	\end{minipage}
	\begin{minipage}{0.45\textwidth}
	\centering
		\smallboard
		\begin{position}
		\rj{f}{1} \rb{d}{9} \rb{g}{9}
		
		\bj{e}{10} \bx{c}{6} \bx{g}{6} \bm{e}{7}
		\end{position}
	\caption{}\label{双低兵巧胜马双象} 
	\end{minipage}
\end{figure}
\subsection{双低兵巧胜马双象}
(图\ref{双低兵巧胜马双象}),此局巧胜要诀:用帅拴马,弃兵闷杀。

1.帅四平五!将5平6

如改走象7退5,则兵三平四,马5退7,帅五平六,再兵六进一,杀局。

2.兵六平五,象7退9 \qquad 3.兵三平四!马5退6

4.兵五进一

红胜。
\subsection{三低兵巧胜单缺象}
\begin{wrapfigure}{r}{5cm}
\centering
\vspace{-1.5cm}
\smallboard
\begin{position}
\rj{e}{1} \rb{a}{9} \rb{d}{9} \rb{g}{9}

\bj{f}{10} \bs{e}{9} \bs{d}{10} \bx{a}{8}
\end{position}
\caption{}\label{三低兵巧胜单缺象} 
\end{wrapfigure}
(图\ref{三低兵巧胜单缺象}),此局巧胜要诀:冲底兵破双士,利用等着入局。

1.兵九平八,象1进3 \qquad 2.兵八平七,象3退1

3.帅五进一!象1进3 \qquad 4.兵七进一!象3退5

5.兵七平六!士5退4 \qquad 6.兵六进一,象5进3

7.兵六平五

红胜。

\subsection{三兵胜卒单缺士(一)}
(图\ref{三兵胜卒单缺士(一)}),红有高兵属例胜局。

1.兵七平六,卒5进1 \qquad 2.兵三进一,象3退1

3.兵三平四,士5退4 \qquad 4.兵六进一!..............

弃兵破士,引离黑将,妙手入局。

4. ..............,将5平4 \qquad 5.前兵平五

红胜。

\begin{figure}[!htbp]
\centering
	\begin{minipage}{0.45\textwidth}
	\centering
		\smallboard
		\begin{position}
		\rj{e}{1} \rb{f}{7} \rb{g}{8}  \rb{c}{9}
		
		\bj{e}{10} \bs{e}{9} \bx{e}{8} \bx{c}{6} \bb{e}{4}
		\end{position}
	\caption{} \label{三兵胜卒单缺士(一)} 
	\end{minipage}
	\begin{minipage}{0.45\textwidth}
	\centering
		\smallboard
		\begin{position}
		\rj{e}{2} \rb{d}{7} \rb{f}{9}  \rb{c}{9}
		
		\bj{d}{10} \bs{f}{8} \bx{a}{8} \bx{g}{6} \bb{e}{4}
		\end{position}
	\caption{}\label{三兵胜卒单缺士(二)} 
	\end{minipage}
\end{figure}

\subsection{三兵胜卒单缺士(二)}
(图\ref{三兵胜卒单缺士(二)}),此局属例胜局,首着平中兵是获胜的关键。

1.兵六平五!..............

如误走兵六进一,则将4平5,兵七平六,象1进3,帅五平四,卒5平6,前兵平七,象3退1,兵六进一,象7退5,和局。

1. ..............,将4平5 \qquad 2.兵七平六,卒5平4

3.帅五平四,卒4平5 \qquad 4.兵五平四

捉死黑士,红胜。
\subsection{三兵胜士象全}
\begin{wrapfigure}{r}{5cm}
\centering
\vspace{-1.5cm}
\smallboard
\begin{position}
\rj{e}{1} \rb{c}{7} \rb{d}{7} \rb{f}{7}

\bj{d}{9} \bs{e}{9} \bs{f}{10} \bx{c}{6} \bx{g}{10}
\end{position}
\caption{}\label{三兵胜士象全} 
\end{wrapfigure}
(图\ref{三兵胜士象全}),属例胜局。

1.兵四平三,象7进5 \qquad 2.兵三进一,象5进7

3.兵三进一,象3退1 \qquad 4.兵三平四!..............

冲兵卡肋,是获胜的第一步骤。

4. ..............,象1进3 \qquad 5.兵七进一,象7退5

6.帅五平六!象5进7 \qquad 7.兵六进一!..............

冲高兵叫将,是获胜的第二步骤。

7. ..............,将4退1 \qquad 8.兵六进一,将4平5

附图所示,红方有两种胜法,为获胜的第三步骤。

【第一种胜法】

9.兵六平五!..............

破士,获胜精妙着法。

9. ..............,士6进5 \qquad 10.帅六平五,象3退5

11.兵七进一,象7退9 \qquad 12.兵七平六,士5进4

\begin{wrapfigure}{r}{5cm}
\centering
%\vspace{-1.5cm}
\smallboard
\begin{position}
\rj{d}{1} \rb{c}{8} \rb{d}{9} \rb{f}{9}

\bj{e}{10} \bs{e}{9} \bs{f}{10} \bx{c}{6} \bx{g}{6}
\end{position}
\caption*{附图}\label{三兵胜士象全:附图} 
\end{wrapfigure}
13.帅五平四

红胜。

【第二种胜法】

9.帅六平五,..............

如误走兵七进一,则士5进4,成为和局。但如红方有相则必胜,读者不妨自演。

9. ..............,象7退5 \qquad 10.兵七进一,士5进4

11.帅五平四,6进5 \qquad 12.兵六平五!士4退5

13.兵七平六,5进6 \qquad 14.帅四平五

再帅五平六,红胜。

\subsection{三兵胜炮双象}
\begin{wrapfigure}{r}{5cm}
\centering
\vspace{-1.5cm}
\smallboard
\begin{position}
\rj{e}{1} \rb{d}{9} \rb{g}{7} \rb{h}{7}

\bj{f}{9} \bx{e}{8} \bx{c}{6} \bp{b}{8}
\end{position}
\caption{}\label{三兵胜炮双象} 
\end{wrapfigure}
(图\ref{三兵胜炮双象}),双高兵一低兵攻炮双象,属例胜局。

1.兵三平四,炮2进1 \qquad 2.兵二平三,炮2平1

3.帅五平四,炮1退1 \qquad 4.兵三进一!..............

弃兵引离黑炮,精巧入局。

4. ..............,炮1平7

如改走将6退1,则兵六平五,红胜。

5.兵四进一,将6退1 \qquad 6.兵四进一,将6平5

7.兵四进一

红胜。

\subsection{三兵巧胜炮双士}
\begin{wrapfigure}{r}{5cm}
\centering
\vspace{-1.5cm}
\smallboard
\begin{position}
\rj{e}{1} \rb{d}{6} \rb{e}{7} \rb{c}{9}

\bj{d}{10} \bs{e}{9} \bs{f}{8} \bp{d}{8}
\end{position}
\caption{}\label{三兵巧胜炮双士} 
\end{wrapfigure}
(图\ref{三兵巧胜炮双士}),此局巧胜要诀:双高兵推进腹地,巧妙破士。

1.兵五进一,炮4平1 \qquad 2.兵六进一,炮1进2

3.兵五平四!士5进6

形成高低兵巧胜炮士之局。

4.兵六进一,炮1平5

如改走士6退5,则兵七平六,将4平5,兵六平五,将5平6,兵六平五,红胜。

5.兵七平六,将4平5 \qquad 6.后兵平五,炮5平8

7.帅五半四,炮8退4 \qquad 8.兵五平四,炮8平6

9.兵四平五,炮6进4 \qquad 10.兵五进一,将5平4

11.兵六进一

红胜。

\subsection{三兵巧胜炮士象}
(图\ref{三兵巧胜炮士象}),此局巧胜要诀:高兵从左翼进击。

1.兵六进一!炮5平4

\begin{wrapfigure}{r}{5cm}
\centering
\vspace{-1.5cm}
\smallboard
\begin{position}
\rj{e}{1} \rb{d}{6} \rb{b}{8} \rb{f}{9}

\bj{d}{9} \bs{e}{9} \bx{e}{8} \bp{e}{10}
\end{position}
\caption{}\label{三兵巧胜炮士象} 
\end{wrapfigure}

如改走士5进6,则帅五平四,炮5平4,兵六平五,象5进3,兵八平七,炮4平1,兵五平六,炮1进1,兵六进一,将4退l,帅四平五,伏兵七进一,红胜。

2.兵六平七,炮4平5 \qquad 3.兵八平七,士5进6

4.帅五平四,炮5平1

如改走炮5平9,则后兵平六,象5进7,兵六进一,将4退1,帅四平五,炮9进2,帅五平六,红胜。

5.后兵平六,象5进7 \qquad 6.帅四平五,炮1进1

7.兵六进一,将4退1 \qquad 8.兵七进一!象7退5

9.兵四进一

再兵七平六杀,红胜。

\subsection{三兵胜马双象}
\begin{wrapfigure}{r}{5cm}
\centering
\vspace{-1.5cm}
\smallboard
\begin{position}
\rj{e}{1} \rb{e}{6} \rb{c}{9} \rb{f}{9}

\bj{d}{10} \bx{e}{8} \bx{a}{8} \bm{g}{8}
\end{position}
\caption{}\label{三兵胜马双象} 
\end{wrapfigure}
(图\ref{三兵胜马双象}),一高兵两低兵,属例胜局。

1.兵五进一!..............

弃中兵,妙!获胜关键着法。

1. ..............,马7进5

如改走象5进7,则兵五平六再进一,红胜。

2.兵四平五,象5进7 \qquad 3.兵七平六!马5退4

4.兵五进一

弃兵闷杀,红胜。

\subsection{三兵巧胜马双士}
\begin{wrapfigure}{r}{5cm}
\centering
\vspace{-1.5cm}
\smallboard
\begin{position}
\rj{e}{1} \rb{e}{7} \rb{c}{7} \rb{c}{9}

\bj{d}{10} \bs{e}{9} \bs{f}{10} \bm{f}{8}
\end{position}
\caption{}\label{三兵巧胜马双士} 
\end{wrapfigure}
(图\ref{三兵巧胜马双士}),此局巧胜要诀:三兵有两种胜法:(一)弃兵困毙,黑方欠行;(二)弃兵破士入局。

【第一种胜法】

1.前兵进一,将4平5

如改走将4进1,则兵七进一,红胜。

2.兵五平四,马6进8 \qquad 3.后兵进一,马8进7

4.后兵平六!..............

靠兵弃兵,妙!算准入局。

4. ..............,马7退6 \qquad 5.兵六进一,马6进5

6.帅五进一,马5退4 \qquad 7.兵七平六

红胜。

【第二种胜法】

1.兵五进一,马6进5 \qquad 2.前兵进一!将4平5

3.兵七进一,士5退4 \qquad 4.兵五平四,士6进5

5.兵四进一,马5退7 \qquad 6.前兵平六!将5平4

7.兵四平五

红胜。

\subsection{三兵巧胜马士象}

(图\ref{三兵巧胜马士象}),此局巧胜要诀:冲兵逼马,以兵换象,弃兵困毙。

1.兵六进一,马3进1 \qquad 2.兵七平六,士5进6
3.后兵进一,马1退3 \qquad 4.后兵平五!马3退5

如改走马3退4,则兵五进一,士6退5,兵四平五,将5平6,兵五平六,红胜。

5.帅五进一!6退5 \qquad 6.兵四平五,将5平6

7.兵六进一,马5退3 \qquad 8.帅五退一,马3进4

9.兵六平五

红胜。

\begin{figure}[!htbp]
\centering
	\begin{minipage}{0.45\textwidth}
	\centering
		\smallboard
		\begin{position}
		\rj{e}{1} \rb{d}{6} \rb{c}{9} \rb{f}{9}
		
		\bj{e}{10} \bs{e}{9} \bx{e}{8} \bm{c}{7}
		\end{position}
		\caption{}\label{三兵巧胜马士象} 
	\end{minipage}
	\begin{minipage}{0.45\textwidth}
	\centering
		\smallboard
		\begin{position}
		\rj{d}{1} \rb{d}{9} \rb{e}{9} \rb{g}{7}
		
		\bj{f}{10} \bc{b}{8}
		\end{position}
		\caption{}\label{三兵巧胜单车}
	\end{minipage}
\end{figure}
\subsection{三兵巧胜单车}

(图\ref{三兵巧胜单车}),此局巧胜要诀:首着高兵靠肋,运用帅等着过渡移至四路,使黑单车难以防范,形成“三英战吕布”。

1.兵三平四!..............

如误走帅六平五,则车2平5,帅五平六,车5平6,帅六进一,车6平4,帅六平五,车6平5,帅五平六,车5平6,和局。

1. ..............,车2平5 \qquad 2.帅六进一,车5平9

3.帅六平五,车9平5 \qquad 4.帅五平四,车5进1

如改走车5平4,则兵六进一!车4退2,兵四进一,红胜。

5.兵四进一!车5平6 \qquad 6.帅四平五车6退1

7.兵五进一,将6进1 \qquad 8.兵六平五

红胜。

%==================================================================
\section{炮类}
\subsection{炮仕胜双士}
(图\ref{炮仕胜双士}),属例胜局。此局取胜要诀:用炮等着,控将得士。

1.仕五进六,将5平6 \qquad 2.炮六平四,将6平5

3.炮四平三!..............

等着!准备控制黑将。

3. ..............,将5平6 \qquad 4.炮三平六,将6平5

5.炮六平四!将5平4 \qquad 6.炮四平五,..............

以上红方连续四步平炮,耐人寻味,是获胜的要着。

6. ..............,士5进4 \qquad 7.炮五平六,士6退5

8.炮六退一!将4平5 \qquad 9.炮六进七

得,红胜。


\begin{figure}[!htbp]
\centering
	\begin{minipage}{0.45\textwidth}
	\centering
		\smallboard
		\begin{position}
		\rj{e}{1} \rs{e}{2} \rp{d}{2}
		
		\bj{e}{10} \bs{e}{9} \bs{f}{8}
		\end{position}
	\caption{}\label{炮仕胜双士} 
	\end{minipage}
	\begin{minipage}{0.45\textwidth}
	\centering
		\smallboard
		\begin{position}
		\rj{f}{2} \rs{e}{2} \rp{d}{1}
		
		\bj{e}{10} \bb{f}{4}
		\end{position}
	\caption{}\label{炮仕巧胜高卒} 
	\end{minipage}
\end{figure}
\subsection{炮仕巧胜高卒}
(图\ref{炮仕巧胜高卒}),此局巧胜要诀:首着平炮打卒,抢中制胜。

1.炮六平四!..............

如误走炮六平五,则将5平4,炮五平四,卒6平5,帅四进一,将4进1,炮四平五,卒5平4,帅四平五,卒4平5,帅五平四,卒5平4,和局。

1. ..............,卒6平7 \qquad 2.帅四进一,将5平4

如改走卒7进1,则帅四退一,将5进1,炮四平七,将5退1,炮七进二,将5进1,仕五进六,卒7进1,帅四进一,将5退1,仕六退五,将5退1,炮七退二,将5进1,炮七平五,将5平4,帅四平五,红胜。

3.帅四平五,卒7平6 \qquad 4.仕五进六,卒6平5

5.帅五退一,将4进1 \qquad 6.炮四平六,将4平5

7.炮六平五

红胜。

\subsection{炮仕巧胜双低卒}
\begin{wrapfigure}{r}{5cm}
\centering
\vspace{-1.5cm}
\smallboard
\begin{position}
\rj{d}{3} \rs{f}{3} \rp{h}{5}

\bj{f}{10} \bb{c}{2} \bb{g}{2}
\end{position}
\caption{}\label{炮仕巧胜双低卒} 
\end{wrapfigure}
(图\ref{炮仕巧胜双低卒}),此局巧胜要诀:运炮伏击卒,使黑卒远离中,抢中夺胜。

1.炮二平一,卒7平8 \qquad 2.炮一平九,卒3平2

3.帅六平五,卒8平7 \qquad 4.炮九平一,卒7平8

5.炮一退四,卒8平7 \qquad 6.帅五退一,卒7进1

7.炮一进一

红胜。

\subsection{炮双仕巧胜双底卒}
(图\ref{炮双仕巧胜双底卒}),此局巧胜要诀:采用进炮阻将法,逼黑将进中,调炮胜。

1.帅五平四!将6进1 \qquad 2.炮四进二!将6退1

3.炮四进一!将6平5 \qquad 4.炮四平七,将5进1

5.炮七退八,将5退1 \qquad 6.炮七平五,将5平4

7.帅六平五

红胜。

\begin{figure}[!htbp]
\centering
	\begin{minipage}{0.45\textwidth}
	\centering
		\smallboard
		\begin{position}
		\rj{e}{3} \rs{e}{2} \rs{d}{3} \rp{f}{6}
		
		\bj{f}{10} \bb{f}{2} \bb{g}{2}
		\end{position}
	\caption{}\label{炮双仕巧胜双底卒} 
	\end{minipage}
	\begin{minipage}{0.45\textwidth}
	\centering
		\smallboard
		\begin{position}
		\rj{f}{3} \rs{e}{2} \rs{d}{3} \rp{d}{1}
		
		\bj{e}{9} \bb{e}{4} \bx{e}{8}
		\end{position}
	\caption{}\label{炮双仕巧胜高卒象} 
	\end{minipage}
\end{figure}
\subsection{炮双仕巧胜高卒象}
(图\ref{炮双仕巧胜高卒象}),属例胜局。

1.炮六平五!卒5平4

如改走卒5平6,帅四平五,将5平4,仕五进四,将4平5,炮五进一,红胜。(同正变)

2.帅四退一,卒4平3 \qquad 3.炮五平六,将5退1

4.帅四进一,将5进1 \qquad 5.帅四平五,将5平6

6.仕五进四,将6平5 \qquad 7.炮六平五,卒3平4

8.炮五进一!

进炮等着!红必得象,胜定。
\subsection{炮双仕巧胜士象}
\begin{wrapfigure}{r}{5cm}
\centering
\vspace{-1.5cm}
\smallboard
\begin{position}
\rj{d}{2} \rs{e}{2} \rs{f}{3} \rp{g}{1}

\bj{f}{10} \bs{d}{8} \bx{i}{8}
\end{position}
\caption{}\label{炮双仕巧胜士象} 
\end{wrapfigure}

(图\ref{炮双仕巧胜士象}),此局巧胜要诀:首着炮镇中路,用帅拴黑、将,形成胜局。否则是和局。

1.炮三平五!象9进7 \qquad 2.帅六进一象7退9

如改走将6进1,则炮五平四,将6平5,炮四平六,得士,红胜。

3.帅六平五,士4退5 \qquad 4.炮五平四,将6平5

5.仕五进六,将5平4 \qquad 6.炮四平六,将4平5

7.炮六平五

红胜。
\subsection{炮双仕巧胜单马}
(图\ref{炮双仕巧胜单马}),此局巧胜要诀:帅控中路,用炮制马。

1.帅六平五!马3进5

如改走马3进4踩仕,则炮九退六,马4进3,帅五进一,将6进1,炮九平八!再进一,红胜。

2.炮九退六,马5进7 \qquad 3.炮九平三,将6进1

4.炮三进一

红胜。

\begin{figure}[!htbp]
\centering
	\begin{minipage}{0.45\textwidth}
	\centering
	\smallboard
	\begin{position}
	\rj{d}{2} \rs{d}{3} \rs{f}{3} \rp{a}{7}
	
	\bj{f}{10} \bm{c}{5}
	\end{position}
	\caption{}\label{炮双仕巧胜单马} 
	\end{minipage}
	\begin{minipage}{0.45\textwidth}
	\centering
	\smallboard
	\begin{position}
	\rj{e}{3} \rs{d}{3} \rs{e}{2} \rp{g}{5}
	
	\bj{d}{8} \bp{e}{9}
	\end{position}
	\caption{}\label{炮双仕巧胜单炮} 
	\end{minipage}
\end{figure}
\subsection{炮双仕巧胜单炮}
(图\ref{炮双仕巧胜单炮}),此局由于黑将位不佳,红方退炮控中路取胜。

1.炮三退四,炮5进52.炮三平六,将4平5

3.仕五进四,将5平64.炮六平四,将6平5

5.炮四平五

红胜。

\subsection{双炮胜双士}
\begin{wrapfigure}{r}{5cm}
\centering
\vspace{-1.5cm}
\smallboard
\begin{position}
\rj{f}{3} \rp{f}{1} \rp{g}{2}

\bj{d}{9} \bs{e}{9} \bs{f}{8}
\end{position}
\caption{}\label{双炮胜双士} 
\end{wrapfigure}
(图\ref{双炮胜双士}),属例胜局。

1.炮四平六,将4退1 \qquad 2.炮三平六,将4平5

3.帅四平五!..............

进帅控制黑士活动,必要的等着。

3. ..............,将5平6 \qquad 4.前炮平四,将6平5

5.炮四进二!..............

高炮逼将、控中,获胜关键着法。

5. ..............,将5平4 \qquad 6.炮四平五!将4进1

7.炮五平六

重炮杀,红胜。


\subsection{双炮双象胜双象}
(图\ref{双炮胜双士}),属例胜局。

1.炮七平五!将5平6 \qquad 2.帅六平五!..............

红炮打将进帅控制双象,紧要之着。

2. ..............,将6退1 \qquad 3.炮三平四,将6进1

4.炮四进五!..............

塞象腰得象,获胜关键之着。

4. ..............,象5退3 \qquad 5.炮五平四

红胜。

\begin{figure}[!htbp]
\centering
	\begin{minipage}{0.45\textwidth}
	\centering
	\smallboard
	\begin{position}
	\rj{d}{3} \rp{c}{1} \rp{g}{2} \rx{c}{5} \rx{g}{5}
	
	\bj{e}{9} \bx{e}{8} \bx{g}{6}
	\end{position}
	\caption{}\label{双炮双象胜双象} 
	\end{minipage}
	\begin{minipage}{0.45\textwidth}
	\centering
		\smallboard
		\begin{position}
		\rj{e}{3} \rp{c}{2} \rp{h}{3} \rx{c}{5} \rx{g}{5}
		
		\bj{e}{10} \bs{e}{9} \bs{f}{10} \bx{e}{8} \bx{g}{10}
		\end{position}
	\caption{}\label{双炮双象胜士象全} 
	\end{minipage}
\end{figure}
\subsection{双炮双象胜士象全}
(图\ref{双炮双象胜士象全}),属例胜局。

1.帅五平六!士5进6

出帅控将,紧着。黑如改走象5进7,则炮七平五,士5进4,炮二平五,将5平4,后炮平六,6进5,炮五退二,伏帅六平五,红得士胜定。

2.炮二进七,象5进7 \qquad 3.炮七平四,士6退5

4.炮四平三,象7退9

如改走象7退5,则炮三平五,黑方欠行。

5.炮二退一!象7进5 \qquad 6.炮三平五

红胜。
\subsection{双炮仕胜士象全}
\begin{wrapfigure}{r}{5cm}
\centering
\vspace{-1.5cm}
\smallboard
\begin{position}
\rj{e}{3} \rs{e}{2} \rp{d}{1} \rp{d}{2}

\bj{e}{10} \bs{e}{9} \bs{f}{8} \bx{e}{8} \bx{g}{6}
\end{position}
\caption{}\label{双炮仕胜士象全} 
\end{wrapfigure}
(图\ref{双炮胜双士}),属例胜局。

1.仕五进四,象7退9

如改走士5退4,则后炮平五,士6退5,炮六平九,士5进4,炮五进七!象7退5,炮九平五,得象,形成炮仕必胜双的局面。

2.后炮平五,象9进7 \qquad 3.帅五平六!..............

出帅控将,紧要之着。

3. ..............,将5平6 \qquad 4.炮六平四!象5退7

5.炮四退一!象7进9 \qquad 6.帅六平五,象9退7

7.炮五进八

得士,红方胜定。

7. ..............,将6进1 \qquad 8.炮五平八,象7退5

9.炮八退七,象7进9 \qquad 10.炮四进七!象9进7

不能走将6进1,则炮八平四杀。

11.炮八平五,象5进3 \qquad 12.炮四退一,将6退1

13.炮五平四,象3退5 \qquad 14.前炮平五

红胜。

\subsection{双炮巧胜单炮}
(图\ref{双炮胜双士}),此局巧胜要诀:利用空头炮制胜。

1.炮八平四!炮1平2

如不动炮,则炮九退四,红胜。

2.炮九平八,炮2平3 \qquad 3.炮八平七,炮3平4

4.炮七平六,炮4平5 \qquad 5.炮六退四,将6进1

6.炮六平四,将6平5 \qquad 7.炮四平五

红胜。

\begin{figure}[!htbp]
\centering
	\begin{minipage}{0.45\textwidth}
	\centering
	\smallboard
	\begin{position}
	\rj{e}{2} \rp{a}{5} \rp{b}{5}
	
	\bj{f}{10} \bp{a}{8}
	\end{position}
	\caption{}\label{双炮巧胜单炮} 
	\end{minipage}
	\begin{minipage}{0.45\textwidth}
	\centering
		\smallboard
		\begin{position}
		\rj{f}{3} \rp{e}{1} \rp{g}{2}
		
		\bj{d}{9} \bm{f}{5}
		\end{position}
	\caption{}\label{双炮胜单马} 
	\end{minipage}
\end{figure}

\subsection{双炮胜单马}
(图\ref{双炮胜单马}),此局属例胜局,但要注意马换炮则成和局。

1.炮三平四!马6退5

另有两种应法:①马6退8,帅六平五,马8进7,炮四进二,马7退6,炮四平六,红胜。②马6退4,帅四平五,将4退1(如马4进3,则帅五退一,将4进1,炮四进二再平六,亦红胜),炮四平六,马4进3,炮六进二,马3进2,炮五平七,马2进4,炮六退二,马4退6,帅五退一,红胜。

2.帅四平五,马5进4 \qquad 3.帅五退一,将4退1

4.炮四进一,将4进1 \qquad 5.炮五平六,马4进6

6.帅五进一,将4退1 \qquad 7.炮四退一,马6进7

8.炮六进一,马7退8 \qquad 9.炮四退一,马8退6

10.帅五退一,马6进7 \qquad 11.炮四进二,马7进6

12.炮六退一,马6退7 \qquad 13.帅五退一

再炮四平六重炮杀,红胜。

\subsection{双炮巧胜马双士}
\begin{wrapfigure}{r}{5cm}
\centering
\vspace{-1.5cm}
\smallboard
\begin{position}
\rj{e}{2} \rp{g}{7} \rp{h}{6}

\bj{e}{10} \bs{e}{9} \bs{d}{10} \bm{f}{8}
\end{position}
\caption{}\label{双炮巧胜马双士} 
\end{wrapfigure}
(图\ref{双炮巧胜马双士}),此局巧胜要诀:利用炮帅拴马法取胜。

1.炮三平七!将5平6 \qquad 2.帅五平四,士5进4

3.炮七进一!..............

打马紧着!控制黑马。

3. ..............,士4进5 \qquad 4.炮二平五!将6进1

5.帅四进一!将6退1 \qquad 6.炮七退七,将6进1

7.炮七平四,将6退1 \qquad 8.炮五退四,将6进1

9.帅四平五,马6进7 \qquad 10.炮五平四

重炮杀,红胜。

\subsection{双炮巧胜炮双士}
\begin{wrapfigure}{r}{5cm}
\centering
\vspace{-1.5cm}
\smallboard
\begin{position}
\rj{d}{2} \rp{e}{2} \rp{g}{6}

\bj{d}{10} \bs{f}{8} \bs{f}{10} \bp{d}{9}
\end{position}
\caption{}\label{双炮巧胜炮双士} 
\end{wrapfigure}
(图\ref{双炮巧胜炮双士}),此局巧胜要诀:类同上局。

1.炮三进二!..............

进炮下二路,控制黑方上士。

1. ..............,将4平5\qquad 2.帅六进一,炮4进1

3.炮三退八,将5进1 \qquad 4.炮三平五,将5平4

如改走将5平6,则帅六平五,再炮五平四,红胜。

5.前炮平六,士6进5 \qquad 6.帅六平五炮4平3

7.炮五平六

重炮杀,红胜。

\subsection{双炮仕胜炮双士}
(图\ref{双炮仕胜炮双士}),属例胜局。

l.炮七进二!炮2进7 \qquad 2.炮七平五!..............

海底叫将,扰乱黑,好棋!

2. ..............,士5进6

如改走士5进4,则仕六退五,将5退1,炮五平九,炮2平5,炮六平四,得士红方胜定。

3.炮六平四,炮2退5 \qquad 4.帅四进一,将5退1

5.炮五平九,将5进1 \qquad 6.仕六退五,炮2平1

7.炮四平五,将5平4 \qquad 8.帅四平五,士6进5

9.炮五平六,将4退1 \qquad 10.仕五进六,士5进4

11.帅五平四,炮1进5 \qquad 12.帅四退一,炮1退6

13.帅四退一,炮1进8 \qquad 14.炮六进一,炮1退8

15.帅四平五,炮1进1 \qquad 16.炮九平五!炮1进5

17.炮五退五,炮1平2 \qquad 18.炮五平六,士4退5

19.仕六退五

重炮杀,红胜。


\begin{figure}[!htbp]
\centering
	\begin{minipage}{0.45\textwidth}
	\centering
		\smallboard
		\begin{position}
		\rj{f}{2} \rs{d}{3} \rp{d}{1} \rp{c}{8}
		
		\bj{e}{8} \bs{e}{9} \bs{f}{10} \bp{b}{10}
		\end{position}
		\caption{}\label{双炮仕胜炮双士} 
	\end{minipage}
	\begin{minipage}{0.45\textwidth}
	\centering
		\smallboard
		\begin{position}
		\rj{d}{3} \rs{f}{3} \rp{d}{1} \rp{e}{4}
		
		\bj{f}{10} \bs{e}{9} \bs{d}{8} \bm{e}{6}
		\end{position}
		\caption{}\label{双炮仕胜马双士}  
	\end{minipage}
\end{figure}
\subsection{双炮仕胜马双士}
(图\ref{双炮仕胜马双士}),此局属例胜局,红方用炮兑马或破士,均可取胜。

1.炮六平四!士5进6

如改走马5退6,则炮四进七兑马,红方胜定。

2.炮五退二,马5进3 \qquad 3.帅六平五,马3退4

4.炮五退一,士4退5

如改走将6进1,则炮五平六!马4进5,炮四进七!将6进1,炮六平四,将6平5,帅四平五,红胜。

5.帅五平六,马4进3 \qquad 6.帅六退一,马3进2

7.帅六进一,马2退4 \qquad 8.炮五进三,马4退5

9.炮五进五

破士,红胜。

\subsection{双炮仕相全胜马双象}
(图\ref{双炮仕相全胜马双象}),属例胜局。

1.炮五平三!马7进6 \qquad 2.帅五退一,象7进9

如改走马6进7,则炮三进一!象7进9,炮九平五,象9进7,炮五平三,马7退6,前炮进四打象,胜法同正变。

3,炮九平五,象9进7 \qquad 4.炮三进五!象5进7

5,炮五平六

红胜。

\begin{figure}[!htbp]
\centering
	\begin{minipage}{0.45\textwidth}
	\centering
		\smallboard
		\begin{position}
		\rj{e}{3} \rs{d}{3} \rs{f}{3} \rp{e}{1} \rp{a}{1} \rx{c}{5} \rx{g}{5}
		
		\bj{d}{10} \bx{e}{8} \bx{g}{10} \bm{g}{7}
		\end{position}
		\caption{}\label{双炮仕相全胜马双象} 
	\end{minipage}
	\begin{minipage}{0.45\textwidth}
	\centering
		\smallboard
		\begin{position}
		\rj{e}{1} \rs{d}{3} \rs{f}{3} \rp{a}{2} \rp{h}{2} \rx{e}{3} \rx{g}{5}
		
		\bj{e}{10} \bx{e}{8} \bx{g}{6} \bp{b}{2}
		\end{position}
		\caption{}\label{双炮仕相全巧胜炮双象}  
	\end{minipage}
\end{figure}
\subsection{双炮仕相全巧胜炮双象}
(图\ref{双炮仕相全巧胜炮双象}),此局巧胜要诀:困炮攻象取胜,着法曲折。

1.炮二平八!象5进3 \qquad 2.炮九退一!将5进1

3.相三退一,将S退1 \qquad 4.相一退三,将5进1

5.仕六退五,将5平4 \qquad 6.帅五平四,将4平5

7.帅四进一,将5退1 \qquad 8.仕五退六,将5平6

9.相五退七,将6平5 \qquad 10.仕四退五,将5进1

11.仕五进六,将5退1 \qquad 12.帅四进一,将5进l

13.相三进五,将5退1 \qquad 14.相五进七,将5进1

15.相七进五,将5退1 \qquad 16.仕六退五,将5进1

l7.仕五退四,将5退1 \qquad 18.仕六进五,将5进1

19.仕五进六,将5退1 \qquad 20.炮九进一,象7退5

21.炮九进二!..............

红方先用双炮困住黑炮,一系列调整帅、仕、相,着法精妙。现在高炮准备展开中路攻势。

21. ..............,炮8平9 \qquad 22.炮九平五,象5退7

23.相七退九!炮1退1 \qquad 24.相五进三,将5进1

25.炮五退三,将5退1 \qquad 26.炮八平三!象7进9

27.相三退五

黑必丢象,红胜。

\section{炮兵类}
\subsection{炮底兵巧胜双士}
(图\ref{炮底兵巧胜双士}),此局巧胜要诀:炮兵禁将。

1.炮八进六!..............

要着!防止黑动士。

1. ..............,将6退1 \qquad 2.兵二平三,将6进1

3.炮八平七

困毙,红胜。

\begin{figure}[!htbp]
\centering
	\begin{minipage}{0.45\textwidth}
	\centering
		\smallboard
		\begin{position}
		\rj{e}{2} \rp{b}{3} \rb{h}{10}
		
		\bj{f}{9} \bs{d}{8} \bs{f}{8}
		\end{position}
		\caption{}\label{炮底兵巧胜双士} 
	\end{minipage}
	\begin{minipage}{0.45\textwidth}
	\centering
		\smallboard
		\begin{position}
		\rj{e}{2} \rs{d}{3} \rx{e}{3} \rp{e}{1} \rb{e}{10}
		
		\bj{e}{9} \bs{d}{8} \bx{i}{8}
		\end{position}
		\caption{}\label{炮底兵仕相胜士象}  
	\end{minipage}
\end{figure}
\subsection{炮底兵仕相胜士象}
(图\ref{炮底兵仕相胜士象}),属例胜局。

1.炮五平六!将5平4

如改走将5进1,则帅五平四,士4退5,炮六平五,将5平4,炮五进八得士,红胜。

2.相五进三,象9进7 \qquad 3.炮六平五,象7退9

4.炮五平三!士4退5 \qquad 5.帅五退一,士5进4

6.炮三进三,士4退5 \qquad 7.炮三退二,士5进4

8.兵五平四,..............

以上红运炮、分兵,意在控制黑象。

8. ..............,士4退5 \qquad 9.炮三平六,士5进4

10.帅五平六!象9进7 \qquad 11.炮六平三,象7退9

12.仕六退五!将4平5 \qquad 13.仕五进四,将5进1

14.相三退五,象9进7 \qquad 15.帅六进一,象7退9

16.帅六进一,将5平6 \qquad 17.炮三平六,士4退5

18.炮六平四,将6平5 \qquad 19.炮四平五,将5平6

20.炮五进八

得士,红胜。

\subsection{炮低兵巧胜单象}
\begin{wrapfigure}{r}{5cm}
\centering
\vspace{-1.5cm}
\smallboard
\begin{position}
\rj{e}{2} \rp{e}{3} \rb{b}{8}

\bj{f}{8} \bx{g}{6}
\end{position}
\caption{}\label{炮低兵巧胜单象} 
\end{wrapfigure}
(图\ref{炮低兵巧胜单象}),此局巧胜要诀:炮兵控象。

1.炮五平九!将6退1 \qquad 2.炮九进五,将6退1

3.兵八平七,将6进1 \qquad 4.兵七平六,将6退1

5.炮九退一,将6进1 \qquad 6.炮九平四!将6退1

7.兵六平五,将6平5 \qquad 8.帅五平六,象7退9

9.兵五平四,将5进1

如改走象9退7,则兵四进一,象7进9,炮四平二!黑欠行,红胜。

10.炮四平二!象9退7 \qquad 炮二进二,象7进5

12.帅六平五

得象,红胜

\subsection{炮低兵巧胜士象}
(图\ref{炮低兵巧胜士象}),此局巧胜要诀:运炮控将、象取胜。

1.炮二进七!象7进9 \qquad 2.炮二平八,象9进7

3.炮八退二!将4退1 \qquad 4.兵四平五!

黑必失象,红胜。

\begin{figure}[!htbp]
\centering
	\begin{minipage}{0.45\textwidth}
	\centering
		\smallboard
		\begin{position}
		\rj{d}{1} \rp{h}{3} \rb{f}{9}
		
		\bj{d}{9} \bs{d}{8} \bx{g}{10}
		\end{position}
		\caption{}\label{炮低兵巧胜士象}  
	\end{minipage}
	\begin{minipage}{0.45\textwidth}
	\centering
		\smallboard
		\begin{position}
		\rj{d}{3} \rp{i}{2} \rb{d}{9}
		
		\bj{e}{8} \bs{e}{9} \bs{d}{10}
		\end{position}
		\caption{}\label{炮低兵巧胜双士}  
	\end{minipage}
\end{figure}
\subsection{炮低兵巧胜双士}
(图\ref{炮低兵巧胜双士}),此局巧胜要诀:运炮海底控中,搏士取胜。

1.炮一进八!土5进4 \qquad 2.炮一平五!..............

平中炮控士,抢夺中帅,乃获胜关键之着。

2. ..............,将5平6 \qquad 3.帅六平五,士4退5

4.炮五平一,土5进4 \qquad 5.炮一退三,将6退1

6.炮一平九,士4退5

如改走将6退1,则炮九进二,士4进5,炮九平五,红胜。

7.炮九进二!将6进1 \qquad 8.炮九平五!将6退1

9.炮五退二,士4进5 \qquad 10.炮五平九,士5进4

11.炮九进二,将6退1 \qquad 12.炮九乎八

红胜。

\subsection{炮低兵相巧胜单象}
(图\ref{炮低兵相巧胜单象}),此局巧胜要诀:运炮捉象而胜。

1.帅五进一,象5进7 \qquad 2.炮六平五!象7退9

3.相七进五,象9进7 \qquad 4.相五进三!象7退9

5.炮五平三!

黑必失象,红胜。

\begin{figure}[!htbp]
\centering
	\begin{minipage}{0.45\textwidth}
	\centering
		\smallboard
		\begin{position}
		\rj{e}{1} \rp{d}{1} \rx{c}{1} \rb{e}{9}
		
		\bj{f}{10} \bx{e}{8}
		\end{position}
		\caption{}\label{炮低兵相巧胜单象}  
	\end{minipage}
	\begin{minipage}{0.45\textwidth}
	\centering
		\smallboard
		\begin{position}
		\rj{f}{3} \rp{d}{1} \rx{c}{5} \rb{d}{8}
		
		\bj{e}{10} \bx{e}{8} \bx{g}{6}
		\end{position}
		\caption{}\label{炮低兵相巧胜双象}  
	\end{minipage}
\end{figure}
\subsection{炮低兵相巧胜双象}
(图\ref{炮低兵相巧胜双象}),此局巧胜要诀:运炮控中掠象取胜。

1.兵六进一!象5进3 \qquad 2.炮六平五,象3退1

3.相七退五,象7退5 \qquad 4.炮五进七,象1进3

5.炮五退一,象3退1 \qquad 6.炮五平八!象1退3

7.炮八进二!

困毙,红胜。
\subsection{炮低兵仕胜单缺士}
\begin{wrapfigure}{r}{5cm}
\centering
\vspace{-1.5cm}
\smallboard
\begin{position}
\rj{d}{1} \rs{f}{3} \rp{e}{2} \rb{f}{9}

\bj{e}{10} \bs{d}{8} \bs{f}{8} \bx{c}{6}
\end{position}
\caption{}\label{炮低兵仕胜单缺士} 
\end{wrapfigure}
(图\ref{炮低兵仕胜单缺士}),属例胜局。

1.炮五平六!士4退5 \qquad 2.仕四退五!士5进4

如改走将5平4,则仕五进六,将4平5,帅六平五,象3退5,炮六平五,以兵破双士,红胜。

3.帅六平五,象3退1 \qquad 4.帅五平四,象1进3

5.帅四进一,象3退1 \qquad 6.帅四进一,象l进3

7.仕五进六,士4退5 \qquad 8.帅四平五,象3退5

9.兵四平五!..............

兵破士,是获胜要着。

9. ..............,士6退5 \qquad 10.炮六平五

黑士、象必失其一,红胜。

\subsection{炮低兵巧胜单缺象}
\begin{wrapfigure}{r}{5cm}
\centering
\vspace{-1.5cm}
\smallboard
\begin{position}
\rj{d}{1} \rp{i}{3} \rb{g}{9}

\bj{e}{10} \bs{e}{9} \bs{f}{8} \bx{e}{8}
\end{position}
\caption{}\label{炮低兵巧胜单缺象} 
\end{wrapfigure}
(图\ref{炮低兵巧胜单缺象}),此局巧胜要诀:中炮法而胜。

1.炮一平五!将5平6 \qquad 2.兵三平四,将6平5

如改走将6进1,则炮五平四闷杀,红胜。

3.帅六进一

困毙,红胜。

\subsection{炮低兵相巧胜单缺象}
\begin{wrapfigure}{r}{5cm}
\centering
\vspace{-1.5cm}
\smallboard
\begin{position}
\rj{e}{1} \rp{g}{1} \rx{c}{5} \rb{f}{9}

\bj{d}{9} \bs{d}{8} \bs{f}{8} \bx{c}{10}
\end{position}
\caption{}\label{炮低兵相巧胜单缺象} 
\end{wrapfigure}
(图\ref{炮低兵相巧胜单缺象}),此局巧胜要诀:运炮谋士。

1.帅五平六!象3进1 \qquad 2.炮三进九!..............

沉底炮是获胜要着。如错走炮三进七,则象1进3,炮三平六,士6退5,炮六退二,士5退4,和局。

以下黑有两种着法:

【第一种着法】

2. ..............,象1进3 \qquad 3.炮三平八,象3退1

如改走士6退5,则炮八退三,士5进6(如土5退6,则炮八平四打士,红胜),帅六进一,将4退1,炮八进一,红胜。

4.炮八退三,象1退3 \qquad 5.炮八进二,将4退1

6.炮八退一

得士,红胜。

【第二种着法】

2. ..............,象1退3 \qquad 3.帅六进一!象3进1

如改走士6退5,则炮三退一伏平五打士,红胜。

4.炮三平八,士6退5

如改走象1进3,则炮八退三!象3退5,炮八进一!打象必得土,红胜。

5.炮八退三,士5进6 \qquad 6.相七退五,象1退3

7.炮八进二,将4退1 \qquad 8.炮八退一

得士,红胜。
\subsection{炮低兵单缺相胜单炮}
\begin{wrapfigure}{r}{5cm}
\centering
\vspace{-1.5cm}
\smallboard
\begin{position}
\rj{e}{3} \rs{e}{2} \rs{f}{3} \rx{c}{6} \rp{b}{1} \rb{f}{9}

\bj{d}{9} \bp{c}{3}
\end{position}
\caption{}\label{炮低兵单缺相胜单炮} 
\end{wrapfigure}
(图\ref{炮低兵单缺相胜单炮}),属例胜局。

1.兵四平五,将4进1 \qquad 2.炮八进二,..............

贴炮逼走黑炮,活动仕助攻,要着!如改走炮八平六,则炮3平2,黑可应付。

2. ..............,炮3退1 \qquad 3.仕五进六,炮3进1

4.帅五退一,炮3退1 \qquad 5.炮八退一,炮3平5

如改走炮3进2,则相七退九,红胜。

6.炮八平六,将4平5 \qquad 7.炮六退一!将5平6

8.炮六平四,将6平5 \qquad 9.炮四平五

红胜。


\subsection{炮低兵巧胜马卒双士}
(图\ref{炮低兵巧胜马卒双士}),此局巧胜要诀:底兵建功,运炮声东击西。

1.兵三进一!将6平5 \qquad 2.炮九进一,马4退3

3.炮九退五,马3进4 \qquad 4.炮九平一!马4进5

5.炮一平五,卒8平7 \qquad 6.帅五平四,卒7平6

7.帅四进一,卒6平5 \qquad 8.兵三平四

红胜。

\begin{figure}[!htbp]
\centering
	\begin{minipage}{0.45\textwidth}
	\centering
		\smallboard
		\begin{position}
		\rj{e}{1} \rp{a}{9} \rb{g}{9}
		
		\bj{f}{10} \bs{d}{10} \bs{e}{9} \bm{d}{8} \bb{h}{4}
		\end{position}
		\caption{}\label{炮低兵巧胜马卒双士} 
	\end{minipage}
	\begin{minipage}{0.45\textwidth}
	\centering
		\smallboard
		\begin{position}
		\rj{e}{1} \rp{a}{9} \rb{g}{9}
		
		\bj{f}{10} \bs{d}{10} \bs{e}{9} \bm{d}{8} \bb{h}{4}
		\end{position}
		\caption{}\label{炮低兵仕相巧胜士象全} 
	\end{minipage}
\end{figure}
\subsection{炮低兵仕相巧胜士象全}
(图\ref{炮低兵仕相巧胜士象全}),此局巧胜要诀:抢中占势而胜。

l.炮六平三!象7退9

如改走将5平4,则炮三进四,将4退1,相三退五,象3退5,炮三退四,将4退1,炮三平六,将4平5,炮六平五,象5进3,兵四进一,将5平6,炮五进七,红胜。

2.仕六退五,将5平6 \qquad 3.兵四平三,将6平5

4.仕五进四,士5进6 \qquad 5.炮三退一,将5退1

6.炮三平四,将5进1 \qquad 7.帅六进一,象9进7

8.帅六进一,士6退5 \qquad 9.炮四平五,象3退1

10.仕四退五,将5平6 \qquad 11.帅六平五

再仕五进四、炮五平四,红胜。

\subsection{炮高兵胜士象}
\begin{wrapfigure}{r}{5cm}
\centering
\vspace{-1.5cm}
\smallboard
\begin{position}
\rj{f}{1} \rp{b}{5} \rb{d}{7}

\bj{e}{10} \bs{f}{10} \bx{c}{10}
\end{position}
\caption{}\label{炮高兵胜士象} 
\end{wrapfigure}
(图\ref{炮高兵胜士象}),属例胜局。

1.炮八平五!象3进1 \qquad 2.炮五退四,象1进3

3.帅四进一,象3退1 \qquad 4.兵六平七,象1退3

5.兵七平八,象3进1 \qquad 6.兵八平九,象1退3

红方三步分兵控象,紧着!

7.帅四进一,将5平4

如改走将5进1,则炮六平四打士,红胜。

8.帅四平五,将4进1 \qquad 9.兵九平八,将4退1

10.兵八平七,象3进1 \qquad 11.兵七平六,士6进5

12.炮五进八

得士,红胜。

\subsection{炮高兵相胜双象}
\begin{wrapfigure}{r}{5cm}
\centering
%\vspace{-1.5cm}
\smallboard
\begin{position}
\rj{f}{1} \rx{e}{3} \rp{c}{1} \rb{e}{7}

\bj{e}{10} \bx{c}{10} \bx{g}{10}
\end{position}
\caption{}\label{炮高兵相胜双象} 
\end{wrapfigure}
(图\ref{炮高兵相胜双象}),属例胜局。

1.兵五平六!..............

形成“左兵右将”攻形。

1. ..............,将5平4 \qquad 2.炮七平六,将4平5

3.兵六进一,将5进1 \qquad 4.炮六平五,象3进5

5.帅四进一,象7进9 \qquad 6.炮五进七

得象,红胜。


\subsection{炮高兵仕胜单缺士}
(图\ref{炮高兵仕胜单缺士}),属例胜局。

1.兵四进一!象3退1

如改走士4进5,则炮五进七破士,亦红胜。

2.兵四平五,将6进1 \qquad 3.炮五平四,象1进3

4.炮六进八

红胜。(炮仕例胜单象)

\begin{figure}[!htbp]
\centering
	\begin{minipage}{0.45\textwidth}
	\centering
		\smallboard
		\begin{position}
		\rj{e}{1} \rs{d}{3} \rp{e}{2} \rb{f}{7}
		
		\bj{f}{10} \bs{d}{10} \bx{e}{8} \bx{c}{6}
		\end{position}
		\caption{}\label{炮高兵仕胜单缺士} 
	\end{minipage}
	\begin{minipage}{0.45\textwidth}
	\centering
		\smallboard
		\begin{position}
		\rj{e}{3} \rs{f}{3} \rx{g}{5} \rp{c}{1} \rb{e}{7}
		
		\bj{f}{10} \bs{e}{9} \bs{d}{8} \bx{g}{6}
		\end{position}
		\caption{}\label{炮高兵仕相胜单缺象} 
	\end{minipage}
\end{figure}
\subsection{炮高兵仕相胜单缺象}
(图\ref{炮高兵仕相胜单缺象}),属例胜局。

1.炮七平三!象7退9 \qquad 2.兵五平六!将6进1

3.兵六进一,将6退1

如改走士5进4,则炮三平四杀,红胜。

4.炮三平四,将6平5 \qquad 5.兵六进一

红胜。

\subsection{炮高兵双仕胜炮象}
(图\ref{炮高兵双仕胜炮象}),属例胜局。

1.炮七退六!炮4平6 \qquad 2.帅四平五,炮6平5

如改走将4退1,则炮七平六,将4平5,兵四进一,红胜。

3.帅五平四,象5退7 \qquad 4.炮七平六,将4平5

5.炮六平五,象7进5 \qquad 6.兵四进一

红胜。

\begin{figure}[!htbp]
\centering
	\begin{minipage}{0.45\textwidth}
	\centering
		\smallboard
		\begin{position}
		\rj{f}{3} \rs{e}{2} \rs{d}{3} \rp{c}{7} \rb{f}{7}
		
		\bj{d}{9} \bx{e}{8} \bp{d}{10}
		\end{position}
		\caption{}\label{炮高兵双仕胜炮象} 
	\end{minipage}
	\begin{minipage}{0.45\textwidth}
	\centering
		\smallboard
		\begin{position}
		\rj{e}{2} \rs{d}{3} \rs{f}{3} \rp{g}{1} \rb{d}{6}
		
		\bj{d}{10} \bs{e}{9} \bm{f}{8}
		\end{position}
		\caption{}\label{炮高兵双仕胜马士} 
	\end{minipage}
\end{figure}
\subsection{炮高兵双仕胜马士}
(图\ref{炮高兵双仕胜马士}),属例胜局。

1.炮三平五,士5退6

如改走士5进4,则炮五平六,伏兵六进一,亦红胜。

2.兵六进一!士6进5 \qquad 3.兵六平五,马6进7

4.兵五进一,士5退6 \qquad 5.兵五平六,马7进8

6.兵六进一!

弃兵伏炮五平六杀,红胜。

\subsection{炮高兵单缺仕胜士象全(一)}
\begin{wrapfigure}{r}{5cm}
\centering
\vspace{-1.5cm}
\smallboard
\begin{position}
\rj{e}{3} \rs{d}{3} \rx{c}{5} \rx{g}{5} \rp{e}{1} \rb{e}{7}

\bj{e}{10} \bs{e}{9} \bs{d}{10} \bx{c}{10} \bx{c}{6}
\end{position}
\caption{}\label{炮高兵单缺仕胜士象全(一)} 
\end{wrapfigure}
(图\ref{炮高兵单缺仕胜士象全(一)}),属例胜局。

1.帅五平四!..............

控将,紧要之着。

1. ..............,象3退1 \qquad 2.兵五平六,士5退6

如改走士5进6,则兵六进一,象1进3,兵六进一,象3进1,兵六平七!将5平6(如象1退3,则兵七进一,象3退1,兵七平六,将5平4,帅四平五,红胜),仕六退五,将6进1,炮五平四,破士红胜。

3.兵六进一,象1进3 \qquad 4.兵六进一,象3进1

5.炮五进三!象1退3 \qquad 6.帅四平五,象3进1

7.仕六退五,象1退3 \qquad 8.仕五进四,象3进1

9.帅五平六!象1退3 \qquad 10.炮五退三,象3进1

11.相三退一,象1退3 \qquad 12.兵六进一!将5进1

13.兵六平七

红胜。

\subsection{炮高兵单缺仕胜士象全(二)}
\begin{wrapfigure}{r}{5cm}
\centering
\vspace{-1.5cm}
\smallboard
\begin{position}
\rj{e}{1} \rs{f}{3} \rx{c}{5} \rx{g}{5} \rp{e}{4} \rb{e}{7}

\bj{d}{10} \bs{e}{9} \bs{f}{8} \bx{e}{8} \bx{g}{10}
\end{position}
\caption{}\label{炮高兵单缺仕胜士象全(二)} 
\end{wrapfigure}
(图\ref{炮高兵单缺仕胜士象全(二)}),此局也是例胜局,先用兵管士,再用破羊角士法取胜。

1.兵五平四!象7进9 \qquad 2.仕四退五,象9进7

3.仕五进六,象7退9 \qquad 4.帅五平六,象9退7

5.炮五退二,象7进9 \qquad 6.相三退一!象9退7

7.炮五平四,象7进9 \qquad 8.炮四平六,将4平5

9.炮六平一!..............

先照将、后打象,运用顿挫,伏落仕控将,是炮兵攻士象全的重要步骤。

9. ..............,象9退7 \qquad 10.仕六退五,象5退3

11.仕五进四,象7进5 \qquad 12.炮一平四!象3进1

13.兵四平三,象1退3 \qquad 14.兵三进一,象3进1

15.炮四平五,将5平6 \qquad 16.炮五进二,象5退3

17.帅六平五,象3进5 \qquad 18.仕四退五,象1进3

19.帅五平四!象5退3 \qquad 20.仕五进六,象3进1

21.炮五退二,象1退3 \qquad 22.相一退三,象3进1

23.相三进五

黑必丢士,红胜。

\subsection{炮高兵单缺仕胜士象全(三)}
\begin{wrapfigure}{r}{5cm}
\centering
\vspace{-1.5cm}
\smallboard
\begin{position}
\rj{e}{3} \rs{d}{1} \rx{c}{5} \rx{g}{5} \rp{e}{4} \rb{e}{7}

\bj{f}{10} \bs{e}{9} \bs{d}{10} \bx{g}{10} \bx{i}{8}
\end{position}
\caption{}\label{炮高兵单缺仕胜士象全(三)} 
\end{wrapfigure}
(图\ref{炮高兵单缺仕胜士象全(三)}),此局例胜,用沉底炮攻法获胜。

1.炮五平四,将6平5

如改走象9进7,则兵五平四,将6平5,炮四平三,象7进9,兵四进一,红胜。

2.炮四平八,士5退6

如改走①士5进4,则帅五平四,士4进5,兵五平六!形成胜羊角士局法,与上局胜法相同。②士5进6,则炮八平五,将5平6,兵五平四,士6退5,炮五平四,将6平5,兵四进一,红胜。

3.炮八平二,士4进5 \qquad 4.帅五平六,士5进6

5.炮二进六!..............

沉底炮,是炮兵攻士象全的一种常用战术攻法。

5. ..............,象9进7 \qquad 6.兵五平四,士6退5

7.仕六进五,士5进4 \qquad 8.兵四进一,士4退5

9.兵四进一,象7退5 \qquad 10.帅六平五,士5进6

11.兵四进一,将5进1 \qquad 12.兵四平三,将5平4

13.炮二退一!象5退7 \qquad 14.仕五进六,将4退1

15.炮二退八,士6退5 \qquad 16.炮二平六,将4平5

17.炮六平五

红胜。

\subsection{炮低兵相巧胜士象全}
\begin{wrapfigure}{r}{5cm}
\centering
\vspace{-1.5cm}
\smallboard
\begin{position}
\rj{d}{2} \rx{c}{1} \rp{e}{3} \rb{f}{9}

\bj{d}{10} \bs{e}{9} \bs{f}{10} \bx{e}{8} \bx{g}{6} \bm{d}{9}
\end{position}
\caption{}\label{炮低兵相巧胜士象全} 
\end{wrapfigure}
(图\ref{炮低兵相巧胜士象全}),此局巧胜要诀:分炮叫杀沉底,破士掠象而胜。

1.炮五平二!象7退9 \qquad 2.炮二进七,象9退7

3.兵四平五,象5进3 \qquad 4.炮二平四,象7进5

5.炮四退九,象5进7 \qquad 6.炮四平五!象7退9

7.帅六进一,象9进7 \qquad 8.相七进五,象7退5

9.帅六退一,象3退1

如改走象5进7,则兵五进一杀,红胜。

10.炮五进七,象1进3 \qquad 11.炮五平四,象3退5

12.炮四退七,象5进3 \qquad 13.炮四平五

红胜。

\subsection{双炮兵仕相全胜马士象全}
(图\ref{双炮兵仕相全胜马士象全}),属例胜局。

1.兵六平五!象5进3 \qquad 2.炮九进三,象3退1

3.兵五进一,马6进5 \qquad 4.炮八退九,象1进3

如改走土5退6,则兵五平四,士6进5,兵四进一,士5进6,炮八平五,红胜。

\begin{figure}[!htbp]
\centering
	\begin{minipage}{0.45\textwidth}
	\centering
		\smallboard
		\begin{position}
		\rj{e}{2} \rs{d}{3} \rs{f}{3} \rx{c}{5} \rx{g}{5}
		\rp{a}{7} \rp{b}{10} \rb{d}{7}
		
		\bj{e}{10} \bs{e}{9} \bs{d}{10} \bx{e}{8} \bx{c}{10} \bm{f}{8}
		\end{position}
		\caption{}\label{双炮兵仕相全胜马士象全} 
	\end{minipage}
	\begin{minipage}{0.45\textwidth}
	\centering
		\smallboard
		\begin{position}
		\rj{e}{2} \rs{d}{3} \rs{f}{3} \rx{c}{5} \rx{g}{5}
		\rp{a}{7} \rp{b}{10} \rb{d}{7}
		
		\bj{e}{10} \bs{e}{9} \bs{d}{10} \bx{e}{8} \bx{c}{10} \bp{f}{9}
		\end{position}
		\caption{}\label{双炮兵仕相全胜炮士象全} 
	\end{minipage}
\end{figure}
5.兵五进一!将5进1 \qquad 6.炮八平五

打死马,红胜。

\subsection{双炮兵仕相全胜炮士象全}
(图\ref{双炮兵仕相全胜炮士象全}),属例胜局。

1.兵六平五!象5进3 \qquad 2.炮九进三,象3退1

3.兵五进一,土5退6 \qquad 4.炮八平六,象1进3

5.炮六退一!象3进1 \qquad 6.兵五进一!士6进5

7.炮六平四

得炮,红胜。

\section{马类}
\subsection{马胜单士}
\begin{wrapfigure}{r}{5cm}
\centering
\vspace{-1.5cm}
\smallboard
\begin{position}
\rj{e}{1} \rm{e}{7}

\bj{d}{10} \bs{f}{8}
\end{position}
\caption{}\label{马胜单士} 
\end{wrapfigure}
(图\ref{马胜单士}),属例胜局。

1.马五进七,将4进1 \qquad 2.帅五进一!..............

等着!逼黑上将。

2. ..............,将4进1 \qquad 3.马七退八,士6退5

4.马八进六,士5退4 \qquad 5.马六进八,士4进5

6.马八进七

红胜。

\subsection{马巧胜单象}
\begin{wrapfigure}{r}{5cm}
\centering
\vspace{-1.5cm}
\smallboard
\begin{position}
\rj{d}{1} \rm{b}{5}

\bj{f}{10} \bx{i}{8}
\end{position}
\caption{}\label{马巧胜单象} 
\end{wrapfigure}
(图\ref{马巧胜单象}),黑将、象在一侧,红单马有巧胜之机。

1.帅六平五!..............

以下黑有 3 种应法:

【第一种着法】

1. ..............,象9进7 \qquad 2.马八进六,将6进1

3.马六进四!将6退1 \qquad 4.帅五进一!象7退9

5.马四进二

红胜。

【第二种着法】

1. ..............,将6进1 \qquad 2.马八进六,将6退1

3.帅五进一!将6进1 \qquad 4.马六进四!将6退1

5.马四进二

红胜。

【第三种着法】

1. ..............,象9退7 \qquad 2.马八进六,将6进1

3.马六退四!将6退1 \qquad 4.马四进三

红胜。

\subsection{马巧胜双士}
\begin{wrapfigure}{r}{5cm}
\centering
\vspace{-1.5cm}
\smallboard
\begin{position}
\rj{f}{1} \rm{d}{5}

\bj{e}{8} \bs{d}{8} \bs{f}{10}
\end{position}
\caption{}\label{马巧胜双士} 
\end{wrapfigure}
(图\ref{马巧胜双士}),由于黑方将位不佳,位至“三楼”多危,故红单马有取胜机会。

1.马六进七,将5退1 \qquad 2.帅四进一!..............

等着!进马控士。如急于走马七进八,则将5平4,成为和局。

2. ..............,将5退1 \qquad 3.马七进八,士6进5

如改走士4退5,则帅四平五,红胜。

4.帅四退一

黑必丢士,红胜。

\subsection{马巧胜单卒}
\begin{wrapfigure}{r}{5cm}
\centering
\vspace{-1.5cm}
\smallboard
\begin{position}
\rj{e}{1} \rm{i}{7}

\bj{f}{10} \bb{c}{6}
\end{position}
\caption{}\label{马巧胜单卒} 
\end{wrapfigure}
(图\ref{马巧胜单卒},此局巧胜要诀:运马叫将、控卒过河;帅抢中路,马将抽卒而胜。

1.马一进二,将6进1 \qquad 2.马二退三,将6退1

如改走将6进1,则马三退五,将6平5,帅五平四,以下胜法与正变相同。

3.马三退五!将6平5 \qquad 4,帅五平四,将5进1

5.帅四进一!将5平4

如改走将5进1,则号五进七,将5退1(如将5平4,则帅四退一,卒3进1,马七退五抽卒,红胜),马七退九,将5退1,帅四退一,将5平4,帅四平五,红胜。

6.马五进七,将4退1 \qquad 7.马七退九!将4平5

8.帅四退一,将5平4

如改走将5进1,则马九进八,卒3进1,马八退六,红胜。

9.帅四平五,将4进1 \qquad 10.马九进八

抽卒,红胜。

\subsection{双马胜炮士象}
(图\ref{双马胜炮士象}),属例胜局。

1.马一退三,将6退1 \qquad 2.马三进二,将6进1

3.马九退七,炮5平3 \qquad 4.马七退五,炮3平5

5.马二退三,将6进1 \qquad 6.帅五进一,士5进4

7.马五退四,炮5进3 \qquad 8.马四进六,士4退5

9.马六进五

红胜。

\begin{figure}[!htbp]
\centering
	\begin{minipage}{0.45\textwidth}
	\centering
		\smallboard
		\begin{position}
		\rj{e}{1} \rm{a}{10} \rs{i}{8}
		
		\bj{f}{9} \bs{e}{9} \bx{e}{8} \bp{e}{10}
		\end{position}
		\caption{}\label{双马胜炮士象} 
	\end{minipage}
	\begin{minipage}{0.45\textwidth}
	\centering
		\smallboard
		\begin{position}
		\rj{d}{1} \rm{d}{7} \rm{g}{7}
		
		\bj{e}{10} \bs{e}{9} \bx{g}{10} \bm{d}{6}
		\end{position}
		\caption{}\label{双马胜马士象} 
	\end{minipage}
\end{figure}
\subsection{双马胜马士象}
(图\ref{双马胜马士象}),属例胜局。

1.马六进七,将5平6

如改走将5平4,则马三退四捉死马,红胜。

2.马三进二,将6进1 \qquad 3.马七退六,马4退6

4.帅六平五,马6退4 \qquad 5.马六退五,马4进5

6.马五进三,士5退4 \qquad 7.马三进二,将6平5

8.前马退四

捉死马,红胜。

\subsection{双马胜炮双士}
\begin{wrapfigure}{r}{5cm}
\centering
\vspace{-1.5cm}
\smallboard
\begin{position}
\rj{e}{1} \rm{e}{6} \rm{b}{7}

\bj{e}{10} \bs{e}{9} \bs{f}{10} \bp{d}{9}
\end{position}
\caption{}\label{双马胜炮双士} 
\end{wrapfigure}
(图\ref{双马胜炮双士}),属例胜局。

1.马五进七,炮4进5

如改走将5平4,则马七进八,将4平5,后马进六,红胜。

2.马八进七,将5平4 \qquad 3.后马进八,炮4退4

如改走炮4退5,则马七退六,将4平5,马六进四,红胜。

4.马七退五,将4进1 \qquad 5.帅五进一!炮4进1

6.马五退七

红胜。

\subsection{双马胜马双士}
\begin{wrapfigure}{r}{5cm}
\centering
%\vspace{-1.5cm}
\smallboard
\begin{position}
\rj{e}{1} \rm{a}{7} \rm{f}{7}

\bj{d}{10} \bs{e}{9} \bs{f}{10} \bm{e}{8}
\end{position}
\caption{}\label{双马胜马双士} 
\end{wrapfigure}
(图\ref{双马胜马双士}),属例胜局。

1.马九进八,马5退3

另有两种应法:①将4进1,则帅五进一!士5退4,马四进六,士4进5,马六退七,士5进6,帅五退一!士6退5,马七进八,红胜。②将4平5,则帅五进一!马5退3,马四进六,将5平4,马六退八,红胜。

2.马四退六,将4平5 \qquad 3.帅五进一,马3进4

4.马六进八,马4进5 \qquad 5.马八进七,将5平4

6,马七退五

红胜。

\subsection{双马胜炮双象}
(图\ref{双马胜炮双象}),属例胜局。

1.马六进八!炮4进1

如改走炮4平3,则马七退六,将5平4,帅四平五,炮3进2,马六进四,炮3平5,马八退七抽炮,红胜。

2.马八进七,象9进7 \qquad 3.后马退六,炮4进1

4.马七退八

红胜。

\begin{figure}[!htbp]
\centering
	\begin{minipage}{0.45\textwidth}
	\centering
		\smallboard
		\begin{position}
		\rj{f}{1} \rm{d}{7} \rm{c}{8}
		
		\bj{e}{8} \bx{g}{10} \bx{i}{8} \bp{d}{9}
		\end{position}
		\caption{}\label{双马胜炮双象} 
	\end{minipage}
	\begin{minipage}{0.45\textwidth}
	\centering
		\smallboard
		\begin{position}
		\rj{d}{1} \rm{b}{7} \rm{h}{6}
		
		\bj{e}{10} \bx{g}{10} \bx{e}{8} \bm{d}{5}
		\end{position}
		\caption{}\label{双马胜马双象} 
	\end{minipage}
\end{figure}

\subsection{双马胜马双象}
(图\ref{双马胜马双象}),属例胜局。

1.马八进七,将5进1

如改走将5平6(如将5平4,则马二退四再进五捉死马,红胜),则马二进三,将6进1,马七退五得象,红胜。

2.马二进四,将5平6

如改走将5平4,则马二进三!再退四捉死马,红胜。

3.马七退五

得象,红胜。

\section{马兵类}
\subsection{马底兵胜单象}
(图\ref{马底兵胜单象}),属例胜局。

1.马七进九,将4进1 \qquad 2.马九进七,将4进1

2.帅五进一!象7退9 \qquad 4.马七退五,将4平5

5.帅五平六!将5退1 \qquad 6.马五进三,象9进7

7.帅六退一,象7退5 \qquad 8.帅六平五

红胜。


\begin{figure}[!htbp]
\centering
	\begin{minipage}{0.45\textwidth}
	\centering
		\smallboard
		\begin{position}
		\rj{e}{1} \rm{c}{5} \rb{f}{10}
		
		\bj{d}{10} \bx{g}{6}
		\end{position}
		\caption{}\label{马底兵胜单象} 
	\end{minipage}
	\begin{minipage}{0.45\textwidth}
	\centering
		\smallboard
		\begin{position}
		\rj{f}{1} \rm{c}{7} \rb{f}{10}
		
		\bj{d}{8} \bs{f}{8} \bx{c}{6}
		\end{position}
		\caption{}\label{马底兵胜士象} 
	\end{minipage}
\end{figure}
\subsection{马底兵胜士象}
(图\ref{马底兵胜士象}),属例胜局。

1.帅四平五!士6退5 \qquad 2.兵四平五,士5进6

3.帅五平四!..............

出帅,伏马七退五得士。

3. ..............,士6退5 \qquad 4.马七退五,将4平5

如改走将4退1,则帅四平五,士5退4,马五进七,将4进1,兵五平六,得士红胜。

5.帅四进一,士5进4 \qquad 6.马五进七,将5退1

7.兵五平四,象3退5 \qquad 8.帅四平五

黑必失象,红胜。

\subsection{马底兵胜双士}
(图\ref{马底兵胜双士}),属例胜局。

1.马九退七,将4进1 \qquad 2.马七进八,士5退4

3.兵三平四,土4进5 \qquad 4.兵四平五

黑必丢士,红胜。

\begin{figure}[!htbp]
\centering
	\begin{minipage}{0.45\textwidth}
	\centering
		\smallboard
		\begin{position}
		\rj{e}{1} \rm{a}{9} \rb{g}{10}
		
		\bj{d}{9} \bs{e}{9} \bs{f}{8}
		\end{position}
		\caption{}\label{马底兵胜双士} 
	\end{minipage}
	\begin{minipage}{0.45\textwidth}
	\centering
		\smallboard
		\begin{position}
		\rj{e}{1} \rm{b}{5} \rb{f}{10}
		
		\bj{d}{9} \bx{c}{10} \bx{g}{6}
		\end{position}
		\caption{}\label{马底兵巧胜双象} 
	\end{minipage}
\end{figure}

\subsection{马底兵巧胜双象}
(图\ref{马底兵巧胜双象}),此局黑双象位置不佳,红有巧胜机会。

1.马八进七!将4进1 \qquad 2.马七进八,将4退1

2.兵四平五,象7退5 \qquad 4.兵五平六,将4平5

5.兵六平七,将5平6 \qquad 6.马八进六,将6进1

7.马六退五

红胜。

\subsection{马底兵巧胜单缺象}\label{马底兵巧胜单缺象sec} 
\begin{wrapfigure}{r}{5cm}
\centering
\vspace{-1.5cm}
\smallboard
\begin{position}
\rj{d}{1} \rm{g}{7} \rb{g}{10}

\bj{f}{8} \bs{e}{9} \bs{d}{10} \bx{g}{6}
\end{position}
\caption{}\label{马底兵巧胜单缺象} 
\end{wrapfigure}
(图\ref{马底兵巧胜单缺象}),此局巧胜要诀:控将、兵入九宫破士。

1.帅六平五!士5进4 \qquad 2.兵三平四,士4进5

3.兵四平五,士5退4 \qquad 4.兵五平六,..............

破士,红方胜定。

4. ..............,士4退5 \qquad 5.兵六平五,士5进4

6.马三退五,将6平5 \qquad 7.帅五平六!士4退5

8.帅六进一!士5进6 \qquad 9.马五进三,将5退1

10.兵五平四,象7退5 \qquad 11.帅六平五

黑必失象,红胜。

\subsection{马底兵巧胜士象全}
\begin{wrapfigure}{r}{5cm}
\centering
\vspace{-1.5cm}
\smallboard
\begin{position}
\rj{e}{1} \rm{i}{5} \rb{e}{10}

\bj{d}{9} \bs{e}{9} \bs{d}{8} \bx{c}{10} \bx{a}{8}
\end{position}
\caption{}\label{马底兵巧胜士象全} 
\end{wrapfigure}
(图\ref{马底兵巧胜士象全}),此局黑方将位不佳,红方有巧胜之机。

1.马一进二!象1进3 \qquad 2.马二退四,象3退5

丢象无奈,否则红马四进六再进八,红胜。

3.马四进五,士5进6 \qquad 4.马五退四,士4退5

5.马四退六,将4进1 \qquad 6.马六进七

黑必失士、象,红胜。

\subsection{马低兵巧胜单缺士}
(图\ref{马低兵巧胜单缺士}),此局巧胜要诀:红运马占位,巧破士象而胜。

1.马五退四!..............

以退为进!如改走马五进四,则象7退5,马四进二,士6进5,下着土5退4,则成和局。

1. ..............,象7退5 \qquad 2.马四进六,将4退1

如改走将4进1(或象5进7),则马六进五,再马五进三捉士,红胜。

3.马六进七,\qquad 象5进7

如改走将4平5,则帅五平六!以后马奔卧槽,红胜。

4.兵四进一!象3退5 \qquad 5.帅五进一,象7退9

6.马七进五,将4进1 \qquad 7.马五退七,将4进1

8.马七退五,将4平5 \qquad 9.帅五平六,将5退1

10.马五进三,象9进7 \qquad 11.帅六退一,象7退5

12.帅六平五

黑失象,红胜。

\begin{figure}[!htbp]
\centering
	\begin{minipage}{0.45\textwidth}
	\centering
		\smallboard
		\begin{position}
		\rj{e}{1} \rm{e}{6} \rb{f}{9}
		
		\bj{d}{9} \bs{f}{10} \bx{c}{6} \bx{g}{6}
		\end{position}
		\caption{}\label{马低兵巧胜单缺士} 
	\end{minipage}
	\begin{minipage}{0.45\textwidth}
	\centering
		\smallboard
		\begin{position}
		\rj{e}{1} \rm{d}{5} \rb{g}{8}
		
		\bj{d}{10} \bs{f}{10} \bs{e}{9} \bx{c}{6}
		\end{position}
		\caption{}\label{马低兵巧胜单缺象} 
	\end{minipage}
\end{figure}
\subsection{马低兵巧胜单缺象}
(图\ref{马低兵巧胜单缺象}),此局巧胜要诀:运马控象,小兵横冲深入。

1.马六进七,将4平5 \qquad 2.兵三平四!将5平4

3.马七进八,将4平5

如改走将4进1,则兵四进一,士5进6,兵四进一,形成“马底兵侧胜单士象”的局面。

4.帅五进一!象3退1 \qquad 5.兵四平五,象1进3

6.兵五平六,象3退1 \qquad 7.兵六进一

绝杀,红胜。

\subsection{马低兵巧胜士象全}
\begin{wrapfigure}{r}{5cm}
\centering
\vspace{-1.5cm}
\smallboard
\begin{position}
\rj{f}{1} \rm{e}{6} \rb{d}{9}

\bj{e}{10} \bs{e}{9} \bs{f}{10} \bx{g}{10} \bx{g}{6}
\end{position}
\caption{}\label{马低兵巧胜士象全} 
\end{wrapfigure}
(图\ref{马低兵巧胜士象全}),此局红方已成“左兵右帅”攻势,马位极好,可破士取胜。

1.马五退四!象7退5 \qquad 2.马四退二!士5进6

3.马二进一!士6进5 \qquad 4.马一进二!..............

红四步运马奔槽,形成胜局。

4. ..............,士5退4 \qquad 5.马二进四,将5平6

6.帅四平五,将6进1 \qquad 7.马四退五,象5进7

8.马五进六,象7退5 \qquad 9.马六进八,士4进5

10.马八退七,士5退6 \qquad 11.马七进五,将6进1

12.马五退四!将6退1 \qquad 13.马四退二,将6进1

14.马二进三

黑必丢象,红胜。

\subsection{马低兵巧胜炮士}
\begin{wrapfigure}{r}{5cm}
\centering
\vspace{-1.5cm}
\smallboard
\begin{position}
\rj{e}{1} \rm{c}{7} \rb{d}{9}

\bj{f}{10} \bs{f}{8} \bp{f}{4}
\end{position}
\caption{}\label{马低兵巧胜炮士} 
\end{wrapfigure}
(图\ref{马低兵巧胜炮士}),此局巧胜要诀:用帅“遥控”炮士,形成巧胜。

1.帅五平四!将6进1

如改走炮6退3(如将6平5,则马七退五!),则马七退五,炮6平5,兵六平五,炮5平8,马五进六,红胜。

2.马七退五,炮6平5 \qquad 3.马五进三,将6退1

4.兵六平五,炮5平7 \qquad 5.马三进二,炮7退5

6.帅四进一

红胜。

\subsection{马低兵巧胜炮双象}
(图\ref{马低兵巧胜炮双象}),此局巧胜要诀:形成“左帅右兵”,用帅拴炮,运马从帅后而出获胜。

1.帅五平四!象5进7 \qquad 2.兵六平五,将5平6

3.马七退五,炮6退1 \qquad 4.马五退六,象7进5

5.帅四进一,象5进3 \qquad 6.马六退五,炮6进4

7.马五退四

以后奔马叫将,红胜。

\begin{figure}[!htbp]
\centering
	\begin{minipage}{0.45\textwidth}
	\centering
		\smallboard
		\begin{position}
		\rj{e}{1} \rm{c}{8} \rb{d}{9}
		
		\bj{e}{10} \bx{e}{8} \bx{g}{10} \bp{g}{6}
		\end{position}
		\caption{}\label{马低兵巧胜炮双象} 
	\end{minipage}
	\begin{minipage}{0.45\textwidth}
	\centering
		\smallboard
		\begin{position}
		\rj{e}{1} \rm{b}{9} \rb{c}{7}
		
		\bj{d}{9} \bs{d}{10} \bx{g}{10} \bp{g}{6}
		\end{position}
		\caption{}\label{马高兵胜单缺士} 
	\end{minipage}
\end{figure}
\subsection{马高兵胜单缺士}
(图\ref{马高兵胜单缺士}),属例胜局。

1.兵七进一!士4进5 \qquad 2.兵七进一,将4退1

3.马八退七,象7进5 \qquad 4.兵七平六,将4平5

5.马七退五,象7退9 \qquad 6.帅五平四!..............

形成“左兵右帅”攻势。

6. ..............,土5退4 \qquad 7.马五进三,象9进7

8.马三退一,土4进5 \qquad 9.马一进二,士5进6

10.马二进一,象5退7 \qquad 11.马一退三,将5平6

12.兵六平五!

以后奔马叫将,红胜。

\subsection{马高兵胜单缺象}
\begin{wrapfigure}{r}{5cm}
\centering
\vspace{-1.5cm}
\smallboard
\begin{position}
\rj{e}{1} \rm{e}{5} \rb{e}{7}

\bj{d}{10} \bs{e}{9} \bs{f}{8} \bx{c}{10}
\end{position}
\caption{}\label{马高兵胜单缺象} 
\end{wrapfigure}
(图\ref{马高兵胜单缺象}),属例胜局。

1.马五进七!土5退6 \qquad 2.马七进八!..............

捉象以后平兵捉吃,获胜关键之着。

2. ..............,象3进1 \qquad 3.兵五平六,士6进5

4.兵六平七,士5退6 \qquad 5.兵七平八,土6进5

6.兵八平九

捉死黑象,红胜。

\subsection{马高兵巧胜士象全}
\begin{wrapfigure}{r}{5cm}
\centering
\vspace{-1.5cm}
\smallboard
\begin{position}
\rj{e}{1} \rm{b}{5} \rb{f}{7}

\bj{d}{9} \bs{e}{9} \bs{f}{10} \bx{c}{10} \bx{c}{6}
\end{position}
\caption{}\label{马高兵巧胜士象全} 
\end{wrapfigure}
(图\ref{马高兵巧胜士象全}),马兵控将、得象而胜。

1.马八进七!将4退1 \qquad 2.马七进八,将4进1

如改走将4平5,则马八退六抽象,红胜。

3.兵四平五,士5进6 \qquad 4.兵五平六,士6进5

5.帅五进一!象3退1 \qquad 6.兵六平七,象1进3

7.兵七进一,象1进3 \qquad 8.兵七进一,将4退1

9.兵七进一!

破象后,形成“马底兵巧胜单缺象”,请见第 \ref{马底兵巧胜单缺象sec} 局之胜法。

\subsection{马高兵巧胜卒双士}
(图\ref{马高兵巧胜卒双士}),此局巧胜要诀:进马控将、禁士、限卒制胜。

1.马二进三!将6进1

如改走将6平5,则马三退四,卒5进1,马四退六,捉卒破士,红胜。

2.帅五平四!卒5平6 \qquad 3.帅四进一!卒6平7

4.兵四平五,卒7平6 \qquad 5.兵五进一,士5退4

6.马三进二,将6退1 \qquad 7.兵五平四,士4进5

8.兵四进一,将6平5 \qquad 9.帅四平五!卒6平5

10.马二退三,士5进6 \qquad 11.马三退四,卒5进1

12.帅五平六,卒5平6 \qquad 13.马四进五!

红胜。

\begin{figure}[!htbp]
\centering
	\begin{minipage}{0.45\textwidth}
	\centering
		\smallboard
		\begin{position}
		\rj{e}{1} \rm{h}{6} \rb{f}{7}
		
		\bj{f}{10} \bs{e}{9} \bs{f}{8} \bb{e}{4}
		\end{position}
		\caption{}\label{马高兵巧胜卒双士} 
	\end{minipage}
	\begin{minipage}{0.45\textwidth}
	\centering
		\smallboard
		\begin{position}
		\rj{f}{1} \rs{d}{1} \rx{c}{1} \rm{e}{6} \rb{c}{7}
		
		\bj{e}{10} \bs{e}{9} \bs{f}{10} \bb{h}{4}
		\end{position}
		\caption{}\label{马高兵仕相胜卒双士} 
	\end{minipage}
\end{figure}
\subsection{马高兵仕相胜卒双士}
(图\ref{马高兵仕相胜卒双士}),属例胜局。

1.兵七进一!卒8平7 \qquad 2.兵七进一,卒7平8

如改走将5平4,则帅六进一!卒7平8,马五进七,卒8平7,兵七进一!将4平5,帅四平五!卒7平6,马七进九,红胜。

3.兵七平六,卒8平7 \qquad 4.帅四进一!卒7平8

5.马五退三,士5进6 \qquad 6.马三进二,士6进5

7.马八进七,将5平6 \qquad 8.兵六平五

红胜。

\subsection{马高兵胜炮士}
(图\ref{马高兵胜炮士}),属例胜局。

1.兵四平五!炮4退2

如改走士5退4,则兵五进一,炮4退1,兵五平四,炮4进1(如士4进5,则马六进五再兵四平五,红胜),帅五平四,将6平5,兵四进一,士4进5,帅四平五,红胜。

2.马六进五!炮4平5 \qquad 3.帅五平六,炮5进3

4.马五退三

红胜。

\begin{figure}[!htbp]
\centering
	\begin{minipage}{0.45\textwidth}
	\centering
		\smallboard
		\begin{position}
		\rj{e}{1} \rm{d}{7} \rb{f}{7}
		
		\bj{f}{10} \bs{e}{9} \bp{d}{8}
		\end{position}
		\caption{}\label{马高兵胜炮士} 
	\end{minipage}
	\begin{minipage}{0.45\textwidth}
	\centering
		\smallboard
		\begin{position}
		\rj{e}{1} \rm{b}{5} \rb{e}{7}
		
		\bj{f}{10} \bs{e}{9} \bm{h}{6}
		\end{position}
		\caption{}\label{马高兵胜马士} 
	\end{minipage}
\end{figure}

\subsection{马高兵胜马士}
(图\ref{马高兵胜马士}),属例胜局。

1.兵五进一!士5退4 \qquad 2.马八进六,将6平5

3.马六进七,马8进6 \qquad 4.帅五平六,马6进5

5.帅六进一,马5退4 \qquad 6.马七进六!..............

弃马踩士,妙!拴马定胜。

6. ..............,将5平4 \qquad 7.兵五进一

红胜。

\subsection{马高兵胜炮象}
(图\ref{马高兵胜炮象}),属例胜局。

1.马三进五!炮4平5 \qquad 2.马五进四,将5平6

3.兵五进一,炮5平4 \qquad 4.马四进二,将6退1

5.兵五平四!将6平5 \qquad 6.兵四进一,炮4平3

7.马二退四,炮3平1 \qquad 8.兵四平五,将5平6

9.马四退二

红胜。


\begin{figure}[!htbp]
\centering
	\begin{minipage}{0.45\textwidth}
	\centering
		\smallboard
		\begin{position}
		\rj{d}{1} \rm{g}{4} \rb{e}{7}
		
		\bj{e}{9} \bx{g}{10} \bp{d}{10}
		\end{position}
		\caption{}\label{马高兵胜炮象} 
	\end{minipage}
	\begin{minipage}{0.45\textwidth}
	\centering
		\smallboard
		\begin{position}
		\rj{d}{1} \rm{c}{7} \rb{f}{7}
		
		\bj{f}{10} \bx{g}{10} \bm{c}{6}
		\end{position}
	\caption{}\label{马高兵巧胜马象} 
	\end{minipage}
\end{figure}

\subsection{马高兵巧胜马象}
(图\ref{马高兵巧胜马象}),此局巧胜要诀:红马高兵占位甚好,控黑马、象而胜。

1.帅六平五!..............

紧手!伏马七进六再兵四进一杀。

1. ..............,马3进5 \qquad 2.兵四进一,马5退4

3.马七进六,马4退5 \qquad 4.帅五进一!象7进9

5.兵四平五

捉死黑马,红胜。

\subsection{马双高兵胜单缺象}
(图\ref{马双高兵胜单缺象}),属例胜局。

1.马九进八!将4进1

如改走将4平5,则马八退六抽象,红胜。

2.兵四平五,马7进5 \qquad 3.兵六平七,马5退3

4.兵五进一,马3进5 \qquad 5.兵七进一,马5退6

6.兵五平六,..............

弃兵入杀,精妙之着。

6. ..............,马6退4 \qquad 7.兵七进一,将4退1

8.兵七进一,将4进1 \qquad 9.马八退七

红胜。


\begin{figure}[!htbp]
\centering
	\begin{minipage}{0.45\textwidth}
	\centering
		\smallboard
		\begin{position}
		\rj{e}{1} \rm{a}{7} \rb{d}{7} \rb{f}{7}
		
		\bj{d}{10} \bs{e}{9} \bs{f}{10} \bx{c}{10} \bm{g}{6}
		\end{position}
		\caption{}\label{马双高兵胜单缺象} 
	\end{minipage}
	\begin{minipage}{0.45\textwidth}
	\centering
		\smallboard
		\begin{position}
		\rj{e}{2} \rm{c}{7} \rb{b}{9} \rb{f}{9}
		
		\bj{d}{10} \bs{e}{9} \bs{d}{8} \bx{e}{8} \bx{c}{6} \bm{i}{7}
		\end{position}
	\caption{}\label{马双低兵巧胜马士象全} 
	\end{minipage}
\end{figure}
\subsection{马双低兵巧胜马士象全}
(图\ref{马双低兵巧胜马士象全}),此局红马、双兵已成联攻之势,黑马远离,红方形成巧杀之局。

1.兵八平七,马9进7 \qquad 2.兵七平六,将4平5

3.马七进八!马7进6

如改走马7退8,则兵六进一再马八退六杀,红胜。

4.帅五平四!马6退5 \qquad 5.马八退六!马5退4

6.兵四平五

红胜。

\subsection{双马兵胜马士象全}
\begin{wrapfigure}{r}{5cm}
\centering
\vspace{-1.5cm}
\smallboard
\begin{position}
\rj{f}{1} \rm{a}{6} \rm{d}{7} \rb{d}{9}

\bj{e}{10} \bs{e}{9} \bs{f}{8} \bx{e}{8} \bx{c}{6} \bm{g}{6}
\end{position}
\caption{}\label{双马兵胜马士象全} 
\end{wrapfigure}
(图\ref{双马兵胜马士象全}),属例胜局。

1.马九进八!象3退1 \qquad 2.兵六平五!..............

弃兵换士象,形成胜局。

2. ..............,士6退5 \qquad 3.马六进七,将5平4

4.马七退五,马7退5 \qquad 5.马五退七,象1进3

6.马七进八,将4平5 \qquad 7.前马退六,士5进4

8.马八退七

红胜。
\subsection{双马兵胜炮士象全}
(图\ref{双马兵胜炮士象全}),属例胜局。

1.马七退五!炮4平2 \qquad 2.马五退三,象5进7

3.马三退二!..............

以退为进,从右翼进击奔槽。

3. ..............,象7退5 \qquad 4.马二进一

再马一进二奔槽,红胜。

\begin{figure}[!htbp]
\centering
	\begin{minipage}{0.45\textwidth}
	\centering
		\smallboard
		\begin{position}
		\rj{e}{1} \rm{c}{7} \rm{h}{9} \rb{d}{9}
		
		\bj{e}{10} \bs{e}{9} \bs{f}{8} \bx{e}{8} \bx{g}{10} \bp{d}{8}
		\end{position}
	\caption{} \label{双马兵胜炮士象全} 
	\end{minipage}
	\begin{minipage}{0.45\textwidth}
	\centering
		\smallboard
		\begin{position}
		\rj{e}{1} \rx{e}{3} \rx{g}{1} \rm{d}{7} \rm{f}{6} \rb{g}{7}
		
		\bj{e}{9} \bc{d}{8}
		\end{position}
	\caption{}\label{双马高兵双相胜单车} 
	\end{minipage}
\end{figure}
\subsection{双马高兵双相胜单车}
(图\ref{双马高兵双相胜单车}),属例胜局。

1.兵三进一!将5退1 \qquad 2.兵三乎四,将5平4

3.兵四平五,车4平3 \qquad 4.兵五进一,车3平4

5.马四进二!..............

弃马伏马二进四挂角绝杀,使黑车防不胜防。

5. ..............,车4平3 \qquad 6.马二进四,车3平5

7.帅五进一!车5平4 \qquad 8.兵五进一

红胜。

\section{马炮类}
\subsection{马炮仕胜马双士}
(图\ref{马炮仕胜马双士}),属例胜局。

1.马八进七!马4进2

如改走士5退6,则帅五平六!将4平5,马七退五,士6进5,马五进六兑马,形成“炮仕例胜双士”。

2.马七进八,将4进1

如改走马2退3,则炮五进二,士5退6,炮五平九再进炮打马,红胜。

3.炮五平九,士5退6 \qquad 4.炮九进五!士6进5

5.仕四退五,士5退6 \qquad 6.仕五进六,士6进5

7.炮九退五,士5进4 \qquad 8.炮九平六,士6退5

9.仕六退五,马2进4 \qquad 10.马八退七,将4退1

11.马七退八

红胜。
\begin{figure}[!htbp]
\centering
	\begin{minipage}{0.45\textwidth}
	\centering
		\smallboard
		\begin{position}
		\rj{e}{1} \rs{f}{3} \rp{e}{2} \rm{b}{5}
		
		\bj{d}{10} \bs{e}{9} \bs{f}{8} \bm{d}{8}
		\end{position}
	\caption{} \label{马炮仕胜马双士} 
	\end{minipage}
	\begin{minipage}{0.45\textwidth}
	\centering
		\smallboard
		\begin{position}
		\rj{f}{1} \rs{e}{2} \rx{c}{5} \rp{c}{2} \rm{d}{5}
		
		\bj{e}{10} \bx{e}{8} \bx{g}{6} \bm{c}{4}
		\end{position}
	\caption{}\label{马炮仕相胜马双象} 
	\end{minipage}
\end{figure}
\subsection{马炮仕相胜马双象}
(图\ref{马炮仕相胜马双象}),属例胜局。

1.马六进五,马3退5 \qquad 2.马五进七!..............

高吊车,控制黑将活动。
2. ..............,马5进7 \qquad 3.炮七退一,马7退6

4.炮七平五,马6进7 \qquad 5.仕五进六,象5退7

6.炮五进一,象7退9 \qquad 7.相七退五,象7进5

8.炮五进六

得象,红胜。

\subsection{马炮胜炮双士}
\begin{wrapfigure}{r}{5cm}
\centering
\vspace{-1.5cm}
\smallboard
\begin{position}
\rj{d}{2} \rp{i}{4} \rm{g}{7}

\bj{f}{8} \bs{d}{10} \bs{e}{9} \bp{e}{10}
\end{position}
\caption{}\label{马炮胜炮双士} 
\end{wrapfigure}
(图\ref{马炮胜炮双士}),属例胜局。

1.炮一平九!炮5平7

另有二种着法:①炮5平6,则马三退五,将6退1,炮九进五,士5进6,马五进六,将6平5,马六进八!红得士,胜定。②士5退6,则马三进二,将6退1,炮九平一,将6平5,炮一进五再马二进四得士,胜定。

2.马三退五,将6退1 \qquad 3.炮九进五,士5进6

如改走士5退6,则马五进六,将6平5,马六退四,将5平6(如将5进1,则炮九进一!),帅六平五!炮7进6,炮九退四,红胜。

4.马五进六,将6平5 \qquad 5.马六进八,将5进1

如改走将5退1,则炮九进一,黑丢炮,红胜。

6.马八进六

红得士,胜定。
\subsection{马炮仕相胜炮双象}
\begin{wrapfigure}{r}{5cm}
\centering
\vspace{-1.5cm}
\smallboard
\begin{position}
\rj{d}{1} \rs{f}{3} \rx{e}{3} \rp{e}{4} \rm{g}{8}

\bj{f}{10} \bx{e}{8} \bx{c}{6} \bp{g}{9}
\end{position}
\caption{}\label{马炮仕相胜炮双象} 
\end{wrapfigure}
(图\ref{马炮仕相胜炮双象}),属例胜局。

1.炮五进三!..............

如改走炮五平三,则象5进7拦炮!故进炮伏平三打炮,着法紧凑。

1. ..............,将6进1 \qquad 2.相五进七,象5进7

3.炮五平六!..............

塞象腰,获胜紧要之着。

3. ..............,将6进1

如改走象7退5,则炮六退五,将6退1,炮六平三,象5进7,炮三平四,红胜。

4.炮六退五,将6平5 \qquad 5.炮六平七!象3退1

6.炮七平二,炮7平6 \qquad 7.炮二进六,炮6进1

8.马三退四,将5退1 \qquad 9.炮二平九

红得象,胜定。

\subsection{马炮胜炮士象}
\begin{wrapfigure}{r}{5cm}
\centering
\vspace{-1.5cm}
\smallboard
\begin{position}
\rj{e}{1} \rp{a}{5} \rm{i}{5}

\bj{f}{10} \bs{f}{8} \bx{g}{10} \bp{a}{9}
\end{position}
\caption{}\label{马炮胜炮士象} 
\end{wrapfigure}
(图\ref{马炮胜炮士象}),属例胜局。

1.马一进二!七6退5

如改走将6进1,则炮九平四,士6退5,马二退四,士5进6,马四进六,士6退5,马六进五,红胜。

2.炮九平四,士5退4 \qquad 3.马二退四,炮1平6

4.马四进三,炮6平7 \qquad 5.炮四平三

打死炮,红胜。

\subsection{马炮单缺士胜马单缺象}
(图\ref{马炮单缺士胜马单缺象}),属例胜局。

1.炮九平三!象7进5

如改走象7进9,则马六退五,将6平5,马五进四,马5退6,马四退二,马6退8,炮三平二!打马得象,红胜。

2.炮三平四,将6进1 \qquad 3.仕四退五,马5进6

4.马六退四,象5退3 \qquad 5.马四退二

再马二进三捉死马,红胜。


\begin{figure}[!htbp]
\centering
	\begin{minipage}{0.45\textwidth}
	\centering
		\smallboard
		\begin{position}
		\rj{d}{3} \rs{f}{3} \rx{c}{1} \rx{g}{5} \rp{a}{2} \rm{d}{7}
		
		\bj{f}{10} \bs{e}{9} \bs{f}{8} \bx{g}{10} \bm{e}{7}
		\end{position}
	\caption{} \label{马炮单缺士胜马单缺象} 
	\end{minipage}
	\begin{minipage}{0.45\textwidth}
	\centering
		\smallboard
		\begin{position}
		\rj{e}{1} \rs{f}{3} \rx{c}{5} \rx{g}{5} \rp{e}{2} \rm{b}{7}
		
		\bj{e}{10} \bs{e}{9} \bx{e}{8} \bx{g}{6} \bm{e}{7}
		\end{position}
	\caption{} \label{马炮单缺士胜马单缺士} 
	\end{minipage}
\end{figure}
\subsection{马炮单缺士胜马单缺士}
(图\ref{马炮单缺士胜马单缺士}),属例胜局。

1.帅五平六!士5进6 \qquad 2.马八进七,将5平6

如改走将5进1,则炮五平四,将5平6,马七进五,胜法与正变相同。

3.炮五平四,将6进1 \qquad 4.马七进五,将6平5

5.马五退三,将5平6 \qquad 6.马三退二

得士,形成“马炮单缺仕例胜马双象”之局。

\subsection{马炮单缺士胜炮单缺象}
\begin{wrapfigure}{r}{5cm}
\centering
\vspace{-1.5cm}
\smallboard
\begin{position}
\rj{e}{3} \rs{d}{3} \rx{a}{3} \rx{i}{3} \rp{g}{2} \rm{f}{5}

\bj{e}{10} \bs{e}{9} \bs{f}{10} \bx{g}{10} \bp{f}{6}
\end{position}
\caption{}\label{马炮单缺士胜炮单缺象} 
\end{wrapfigure}
(图\ref{马炮单缺士胜炮单缺象}),属例胜局。

1.马四进六!炮6退2 \qquad 2.相一进三,象7进9

3.马六进八!炮6平2 \qquad 4.炮三平六!象9退7

5.马八进六,将5平4 \qquad 6.仕六退五

红胜。

\subsection{马炮仕相全胜炮单缺士}
(图\ref{马炮仕相全胜炮单缺士}),属例胜局。

1.马七进八!炮6进2

如改走炮6进1,则炮九平五!炮6乎7,马八进六,将5平6,炮五进四,得土红胜。

2.马八进七,将5平6 \qquad 3.炮九平四,炮6平8

4.马七退六,将6平5 \qquad 5.炮四平五,将5平6

6.马六进五

黑失士,红胜。

\begin{figure}[!htbp]
\centering
	\begin{minipage}{0.45\textwidth}
	\centering
		\smallboard
		\begin{position}
		\rj{d}{2} \rs{e}{2} \rs{f}{3} \rx{g}{1} \rx{g}{5} \rp{a}{5} \rm{c}{5}
		
		\bj{e}{10} \bs{e}{9} \bx{e}{8} \bx{g}{6} \bm{f}{10}
		\end{position}
	\caption{} \label{马炮仕相全胜炮单缺士} 
	\end{minipage}
	\begin{minipage}{0.45\textwidth}
	\centering
		\smallboard
		\begin{position}
		\rj{f}{1} \rs{f}{3} \rs{e}{2} \rx{c}{5} \rx{g}{1} \rp{d}{1} \rm{c}{9}
		
		\bj{d}{9} \bs{e}{9} \bs{f}{10} \bx{e}{8} \bx{c}{10} \bm{c}{6}
		\end{position}
	\caption{} \label{马炮仕相全胜马士象全} 
	\end{minipage}
\end{figure}
\subsection{马炮仕相全胜马士象全}
(图\ref{马炮仕相全胜马士象全}),属例胜局,但其攻防变化较为复杂。

1.炮六进一!将4退1

黑另有二种应法:①马3进1,则马七退六,士5进4,马六退五,士4退5,仕五进六,士5进4,马五进四,马1退2,仕六退五,士4退5,马四退六,马2退4,马六进五得象,红胜。②士5退4,则马七退六,马3退4,马六退五,马4进5,仕五进六,将4平5,炮六平五,马5退3,马五进四,将5平6,马四进二,将6进1,炮五平四,红胜。

2.帅四平五,将4进1 \qquad 3.马七退八,士5进6

4.马八进六!将4平5

不能走将4进1,否则仕五进六,红胜。

5.马六退五,将5平6 \qquad 6.马五进三,将6平5

7.马三进二,将5平6 \qquad 8.仕五进六,马3退5

9.炮六平四

黑必失士,红胜。

\section{马炮兵类}
\subsection{马炮兵仕相全胜马双卒士象全}
(图\ref{马炮兵仕相全胜马双卒士象全}),属例胜局。

1.兵六平五!马5退7 \qquad 2.兵五平四,马7进6

3.炮五进一!卒6平7 \qquad 4.马八进七,将6进1

5.炮五平四,士5进6 \qquad 6.马七退六,马6进4

7.马六进四!卒7平6 \qquad 8.马四退二,将6平5

9.马二进三,象5进3 \qquad 10.兵四平五,象3进1

11.炮四平二,将5平4 \qquad 12.炮二平六,士4退5

如改走将4平5,则兵五平六,马4退5,马三退四,将5进1,炮六进二打土,红胜。

13.兵五平六,士5进4 \qquad 14.马三退四,将4退1

15.兵六进一,将4平5 \qquad 16.兵六进一,马4退5

17.兵六平五,将5平6 \qquad 18.马四退二

红胜。

\begin{figure}[!htbp]
\centering
	\begin{minipage}{0.45\textwidth}
	\centering
		\smallboard
		\begin{position}
		\rj{e}{1} \rs{d}{3} \rs{e}{2} \rx{e}{3} \rx{g}{1}
		\rp{e}{5} \rm{b}{7} \rb{d}{7}
		
		\bj{f}{10} \bs{e}{9} \bs{d}{8} \bx{c}{10} \bx{e}{8}
		\bm{e}{6} \bb{e}{4} \bb{f}{4}
		\end{position}
		\caption{}\label{马炮兵仕相全胜马双卒士象全} 
	\end{minipage}
	\begin{minipage}{0.45\textwidth}
	\centering
		\smallboard
		\begin{position}
		\rj{e}{1} \rs{f}{1} \rs{e}{2} \rx{e}{3} \rx{g}{1}
		\rp{a}{10} \rm{b}{10} \rb{d}{7}
		
		\bj{e}{10} \bs{d}{10} \bs{f}{10} \bx{g}{10} \bx{e}{8}
		\bp{i}{8} \bb{e}{4} \bb{d}{4}
		\end{position}
		\caption{}\label{马炮兵仕相全胜炮双卒士象全} 
	\end{minipage}
\end{figure}
\subsection{马炮兵仕相全胜炮双卒士象全}
(图\ref{马炮兵仕相全胜炮双卒士象全}),属例胜局。

1.马八退九!士4进5

如改走将5进1,则马九进七,将5平6,马七退五,红得象胜定。

2.马九进七,将5平4 \qquad 3.炮九退五,将4进1

4.炮九平六,士5进4 \qquad 5.兵六进一!将4平5

不能走将4进1,否则马七退六,红胜。

6.炮六平五!象5进7 \qquad 7.马七退五,将5平6

8.马五退四,炮9平6

黑另有3种应法:①卒5平6,则炮五平四,将6平5,马四进三,将5平6(如将5进1,则兵六进一!),马三退五,将6进1,马五退三,红胜。②将6平5,则马四进三,胜法与①相同。③士6进5,则炮五平四,士5进6,兵六进一!炮9退1,马四进六,士6退5,兵六平五!红胜。

9.马四进三,将6平5 \qquad 10.炮五平四,将5退1

11.炮四进五!炮6退1 \qquad 12.兵六进一

红胜。

\subsection{马炮兵双仕巧胜马炮双士}
(图\ref{马炮兵双仕巧胜马炮双士}),此局巧胜要诀:中兵挺进,抓住黑方兑子,运用等着,形成“单马巧胜双士”局。

1.兵五进一!马2进4

进马兑子(弃子)求和。如改走马2进1,则炮六平五,将5平6,马五进三,炮4进1,仕五进六,亦红胜。

2.兵五平六!炮4进7 \qquad 3.仕五进六,士5进4

4.帅五进一!将5进1 \qquad 5.帅五进一,将5进1

6.仕六退五!将5退1

如改走土4退5,则帅五平六,士5退4,马五进七,将5退1,马七进八得士,红胜。

7.帅五平四!

黑必失士,红胜。

红胜。

\begin{figure}[!htbp]
\centering
	\begin{minipage}{0.45\textwidth}
	\centering
		\smallboard
		\begin{position}
		\rj{e}{1} \rs{d}{1} \rs{e}{2} \rp{d}{3} \rm{e}{6} \rb{e}{7}
		
		\bj{e}{10} \bs{e}{9} \bs{f}{8} \bm{b}{9} \bp{d}{10}
		\end{position}
		\caption{}\label{马炮兵双仕巧胜马炮双士} 
	\end{minipage}
	\begin{minipage}{0.45\textwidth}
	\centering
		\smallboard
		\begin{position}
		\rj{e}{1} \rs{f}{1} \rs{f}{3} \rx{e}{3}
		\rp{i}{7} \rm{g}{9} \rb{f}{8}
		
		\bj{e}{10} \bs{d}{10} \bs{f}{10} \bx{g}{10} \bx{c}{10}
		\bp{f}{9} \bm{f}{7}
		\end{position}
		\caption{}\label{马炮兵单缺相巧胜马炮士象全} 
	\end{minipage}
\end{figure}
\subsection{马炮兵单缺相巧胜马炮士象全}
(图\ref{马炮兵单缺相巧胜马炮士象全}),此局巧胜要诀:首着退马攻马,以兵换炮而胜。

1.马三退二!..............

退马打马,正确!如改走兵四进一,则马6退7,兵四平三,象3进5,兵三平四,士4进5,帅五平六,象7进9,红较难取胜。

1. ..............,马6退4 \qquad 2.兵四进一,马4退6

3.马二进三,土4进5 \qquad 4.帅五平六!..............

如改走炮一进二,则士5进6,炮一平四,将5进1,和局。

4. ..............,士5进6 \qquad 5.炮一平五,象3进1

6.相五进七,象1退3 \qquad 7.炮五退五,象3进1

8.马三退二,马6进8 \qquad 9.马二进四,将5进1

10.马四退五,将5退1 \qquad 11.马六退五,士6进5

12.马五进四

抽马,红胜。

\subsection{马炮兵相巧胜单车}
(图\ref{马炮兵相巧胜单车}),此局巧胜要诀:马炮制车,小兵控将。

1.炮九平六!车2平6 \qquad 2.马四退六,车6平4

3.兵二平三,车4进2 \qquad 4.兵三平四,车4进2

5.兵四平五,车4退4 \qquad 6.兵五平六!..............

横兵献至将口,妙!黑不能走将4退1吃兵,否则马六进八双照杀,红胜。

6. ..............,车4进1 \qquad 7.兵六平七,车6进1

8.帅五进一!..............

上帅弃炮,黑如车4进3杀炮,则马六进四再进五,红胜。

8. ..............,车4退1 \qquad 9.炮六进二,车6进1

10.马六退七,车4平2 \qquad 11.炮六退一,车2平3

12.炮六进四,车3进2 \qquad 13.帅五进一,车3平4

14.马七进八

红胜。

\begin{figure}[!htbp]
\centering
	\begin{minipage}{0.45\textwidth}
	\centering
		\smallboard
		\begin{position}
		\rj{e}{1} \rx{a}{3} \rm{f}{8} \rp{a}{1} \rb{h}{9}
		
		\bj{d}{8} \bc{b}{6}
		\end{position}
		\caption{}\label{马炮兵相巧胜单车} 
	\end{minipage}
	\begin{minipage}{0.45\textwidth}
	\centering
		\smallboard
		\begin{position}
		\rj{e}{1} \rm{d}{7} \rp{e}{3} \rb{f}{9}
		
		\bj{e}{10} \bs{d}{10} \bs{f}{8} \bc{c}{7} \bb{d}{2}
		\end{position}
		\caption{}\label{马炮兵巧胜车卒双士} 
	\end{minipage}
\end{figure}
\subsection{马炮兵巧胜车卒双士}
(图\ref{马炮兵巧胜车卒双士}),此局巧胜要诀:弃兵引将,马炮困车,一鼓作气取胜。

1.兵四平五,将5平6 \qquad 2.兵五进一!将6进1

如改走将6平5,则马六进四,将5进1(如将5平6,则马四进六抽车),马四退五抽车,红胜。

3.炮五平四,士6退5 \qquad 4.马六退四,士5进6

5.马四退六,车3平6 \qquad 6.马六进五,将6平5

7.炮四平五

红胜。

\subsection{马炮兵仕巧胜车双象}
(图\ref{马炮兵仕巧胜车双象}),此局巧胜要诀:马将禁车,调运仕、帅,弃马得车获胜。

1.马六进四!车2平6

如改走将5进1,则兵六平五!将5进1,马四退五,红胜。

2.帅六平五!..............

进帅控局,黑不敢车6进1吃马,否则兵六平五抽车,红胜。

2. ..............,象1退3 \qquad 3.炮五进二,象3进1

4.帅五进一,象1退3 \qquad 5.帅五进一,象3进1

6.仕四进五,象1退3 \qquad 7.仕五进四!象3进1

8.帅五退一,象1退3 \qquad 9.帅五退三,象3进1

10.炮五进一!象1退3 \qquad 11.炮五退三,象3进1

12.马四退五,将5平4 \qquad 13.马五进七,车6平5

14.马七进八!车5平2 \qquad 15炮五平六,车2平4

16.炮六进七

红胜。
\begin{figure}[!htbp]
\centering
	\begin{minipage}{0.45\textwidth}
	\centering
		\smallboard
		\begin{position}
		\rj{d}{1} \rs{f}{1} \rm{d}{7} \rp{e}{2} \rb{d}{8}
		
		\bj{e}{10} \bx{a}{8} \bx{c}{6} \bc{b}{9}
		\end{position}
		\caption{}\label{马炮兵仕巧胜车双象} 
	\end{minipage}
	\begin{minipage}{0.45\textwidth}
	\centering
		\smallboard
		\begin{position}
		\rj{f}{1} \rm{f}{8} \rp{b}{6} \rb{b}{8}
		
		\bj{d}{10} \bs{d}{8} \bs{f}{10}
		\bb{d}{3} \bb{e}{2} \bb{h}{3}
		\end{position}
	\caption{}\label{马炮兵巧胜3卒双士} 
	\end{minipage}
\end{figure}
\subsection{马炮兵巧胜3卒双士}
图(\ref{马炮兵巧胜3卒双士}),此局巧胜要诀:首着平炮三路、限制黑士,平兵入杀,捷足先登。

1.炮八平三!卒8平7 \qquad 2.兵八平七,卒7进1

3.炮三平六,士4退5  \qquad 4.兵七平六

红胜。

\subsection{马炮双兵仕相全巧胜车卒士象全}
(图\ref{马炮双兵仕相全巧胜车卒士象全}),属例胜局。

1.炮七平六,车5平4 \qquad 2.仕六进五,卒9进1

3.兵六平五,车4平5 \qquad 4.兵五进一!..............

弃兵破象,获胜精彩着法。

4. ..............,士6进5

如改走象7进5,则兵七平六,车5平4,马七退五抽车,红胜。

5.兵五平四!..............

弃兵破士,妙手!使黑车防不胜防。

5. ..............,车5平3

如改走士5进6,则马七退六,车5平4,马六进四,车4平6,兵七平六,车6平4,兵六进一,红胜。

6.兵七平六,车3平4 \qquad 7.兵四进一,将4进1

8.马七退九,士5进6 \qquad 9.兵六进一,将4退1

10.马九进七

红胜。

\begin{figure}[!htbp]
\centering
	\begin{minipage}{0.45\textwidth}
	\centering
		\smallboard
		\begin{position}
		\rj{e}{1} \rs{d}{1} \rs{f}{1} \rx{e}{3} \rx{c}{5}
		\rm{c}{9} \rp{c}{3} \rb{c}{7} \rb{d}{7}
		
		\bj{d}{10} \bs{f}{10} \bs{f}{8} \bx{e}{8} \bx{g}{10}
		\bc{e}{6} \bb{i}{5}
		\end{position}
		\caption{}\label{马炮双兵仕相全巧胜车卒士象全} 
	\end{minipage}
	\begin{minipage}{0.45\textwidth}
	\centering
		\smallboard
		\begin{position}
		\rj{d}{1} \rm{b}{9} \rp{e}{7} \rb{f}{9} \rb{g}{9}
		
		\bj{e}{10} \bs{e}{9} \bs{f}{10} \bx{e}{8} \bx{i}{8}
		\bm{g}{2} \bp{h}{8} \bb{e}{2}
		\end{position}
	\caption{}\label{马炮双兵巧胜马炮卒士象全} 
	\end{minipage}
\end{figure}
\subsection{马炮双兵巧胜马炮卒士象全}
(图\ref{马炮双兵巧胜马炮卒士象全}\footnote{原书此局印刷出错,无棋局,整理者根据棋谱推出应为此棋局。}),此局巧胜要诀:首着马入将口,伏弃兵冲士连杀。

1.马八进六!..............

马入将口,绝妙之着。

1. ..............,卒5进1

如改走炮8进6,则兵四进一,将5平6,兵三平四,将6平5,兵四平五,将5平6,兵五平四,红胜。

2.帅六进一,炮8进6 \qquad 3.帅六进一,马7退6

4.炮五退三,马6退5

如改走炮8退2,则马六退五再进七,红胜。

5.马六退伍,士5进4 \qquad 6.马五退六,马5进6

7.马六退四!将5平4 \qquad 8.兵四进一,炮8平6

9.马四进三,将4进1 \qquad 10.兵三平四,象8进7

11.马三退五!..............

献马入杀,获胜妙手。

11. ..............,马6退5 \qquad 12.后兵平五,将4退1

13.兵四平五

红胜。

\section{单车、双车类}
\subsection{单车巧胜士象全(一)}
\begin{wrapfigure}{r}{5cm}
\centering
\vspace{-1.5cm}
\smallboard
\begin{position}
\rj{d}{1} \rc{h}{9}

\bj{d}{9} \bs{d}{8} \bs{e}{9} \bx{g}{10} \bx{g}{6}
\end{position}
\caption{}\label{单车巧胜士象全(一)} 
\end{wrapfigure}
(图\ref{单车巧胜士象全(一)}),此局巧胜要诀:


\end{document}
..............

\begin{wrapfigure}{r}{5cm}
\centering
\vspace{-1.5cm}
\smallboard
\begin{position}
\rj{f}{1} \rb{e}{8}

\bj{f}{10} \bs{f}{8} \bs{d}{8}
\end{position}
\caption{}\label{高兵巧胜双士} 
\end{wrapfigure}

\begin{figure}[!htbp]
\centering
	\begin{minipage}{0.45\textwidth}
	\centering
		\smallboard
		\begin{position}
		\rj{e}{1} \rb{f}{7} \rb{h}{9}
		
		\bj{f}{9} \bx{e}{8} \bx{c}{6}
		\end{position}
	\caption{} \label{高低兵胜双象} 
	\end{minipage}
	\begin{minipage}{0.45\textwidth}
	\centering
		\smallboard
		\begin{position}
		\rj{e}{1} \rb{e}{6} \rb{f}{9}
		
		\bj{e}{10} \bx{e}{8} \bs{e}{9} \bs{d}{10}
		\end{position}
	\caption{}\label{高低兵巧胜单缺象} 
	\end{minipage}
\end{figure}
